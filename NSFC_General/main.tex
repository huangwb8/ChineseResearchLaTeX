%!TEX program = xelatex

% 国家自然科学基金NSFC-面上项目-正文(2025年版)

% 测试环境:[使用VSCode编写LaTeX - 知乎](https://zhuanlan.zhihu.com/p/38178015)

% 编译顺序: xelatex -> bibtex -> xelatex -> xelatex

% 参考 & 鸣谢:
% [Ruzim/NSFC-application-template-latex: 国家自然科学基金申请书正文(面上项目)LaTeX 模板(非官方)](https://github.com/Ruzim/NSFC-application-template-latex)
% [Readon/NSFC-application-template-latex: 国家自然科学基金申请书正文(面上项目)模板(非官方)](https://github.com/Readon/NSFC-application-template-latex)
% [Aligning equations with amsmath - Overleaf, Online LaTeX Editor](https://www.overleaf.com/learn/latex/Aligning_equations_with_amsmath):各种格式的公式写法


\documentclass[12pt,UTF8,AutoFakeBold=2.5,a4paper]{ctexart} %默认小四号字。允许楷体粗体。

%%%%============================系统设置===========================%%%%

\usepackage[english]{babel} %支持混合语言
\usepackage{xcolor}
\usepackage{graphicx} 
\usepackage{amsmath} %更多数学符号
\usepackage{wasysym}
\usepackage{geometry} %改尺寸
\usepackage{fontspec} %字体包
\usepackage{setspace}
\usepackage{amsfonts} % 数学公式增强
\usepackage{xeCJK} % 调整字间距
\usepackage{titlesec} 
\usepackage{ragged2e} % 正文两端对齐所需的包。在需要两端对齐的文字前面添加\justifying即可。
% \usepackage{ctex}
\usepackage{pifont} % 带圈数字
\usepackage{ifplatform} % 检测操作系统

%%%————————————————————————————————边距
% \CJKsetecglue命令定义了CJK字符与其他字符之间的空白,不只限于数字字符,也适用于字母等其他非CJK字符。如果你想要更精细的控制,可能需要借助其它工具或者宏包。
% \hskip0.15em plus0.05em minus 0.05em表示基础间距为0.15em,可增加至0.2em,也可减少至0.1em以便满足排版需求。
% 页边距和字距的某个组合应该可以完美复制Word。
\geometry{left=3.20cm,right=3.14cm,top=2.54cm,bottom=2.54cm} % 页边距
\CJKsetecglue{\hskip 0.3em plus 0.05em minus 0.1em} % 数字与CJK字符的间距
% left=3.20cm,right=3.13-3.14cm

%%%————————————————————————————————颜色
\definecolor{MsBlue}{RGB}{0,112,192} %Ms Word 的蓝色和latex xcolor包预定义的蓝色不一样。通过屏幕取色得到。
\definecolor{headercolor}{RGB}{0,0,0} % 页眉颜色
\definecolor{footercolor}{RGB}{0,0,0} % 页脚颜色

%%%————————————————————————————————对象名
\addto\captionsenglish{
    \renewcommand{\contentsname}{目录}
    \renewcommand{\listfigurename}{插图目录}
    \renewcommand{\listtablename}{表格}
    %\renewcommand{\refname}{\sihao 参考文献}
    \renewcommand{\refname}{\sihao \kaishu \bfseries \leftline{参考文献} \vspace*{5mm}} %这几个字默认字号稍大,改成四号字,楷书,居左(默认居中) 根据喜好自行修改,官方模板未作要求
    \renewcommand{\abstractname}{摘要}
    \renewcommand{\indexname}{索引}
    \renewcommand{\tablename}{表}
    \renewcommand{\figurename}{图}
} %把Figure改成‘图’,reference改成‘参考文献’。如此处理是为了避免和babel包冲突。


%%%————————————————————————————————字号
\newcommand{\chuhao}{\fontsize{42pt}{\baselineskip}\selectfont}
\newcommand{\xiaochuhao}{\fontsize{36pt}{\baselineskip}\selectfont}
\newcommand{\yihao}{\fontsize{26pt}{\baselineskip}\selectfont}
\newcommand{\erhao}{\fontsize{22pt}{\baselineskip}\selectfont}
\newcommand{\xiaoerhao}{\fontsize{18pt}{\baselineskip}\selectfont}
\newcommand{\sanhao}{\fontsize{16pt}{\baselineskip}\selectfont}
\newcommand{\sihao}{\fontsize{14pt}{\baselineskip}\selectfont}
\newcommand{\xiaosihao}{\fontsize{12pt}{\baselineskip}\selectfont}
\newcommand{\yemeizihao}{\fontsize{11pt}{\baselineskip}\selectfont}
\newcommand{\wuhao}{\fontsize{10.5pt}{\baselineskip}\selectfont}
\newcommand{\xiaowuhao}{\fontsize{9pt}{\baselineskip}\selectfont}
\newcommand{\liuhao}{\fontsize{7.875pt}{\baselineskip}\selectfont}
\newcommand{\qihao}{\fontsize{5.25pt}{\baselineskip}\selectfont}
%字号对照表
%二号 21pt
%四号 14
%小四 12
%五号 10.5


%%%————————————————————————————————字体
\ifwindows
  \newfontfamily\kaitichar{KaiTi}[AutoFakeBold] % 利用windows自带字体定义楷体英文和数字
  \NewDocumentCommand \templatefont   { } { \kaishu \kaitichar } %使用模板字体代替直接使用楷书
  \setmainfont{Times New Roman} 
  \setCJKmainfont{KaiTi}[AutoFakeBold=3] 
  % \setCJKmainfont{SimSun}[AutoFakeBold=3] % 可参考[LaTeX 中文字体配置基础指南 - 知乎](https://zhuanlan.zhihu.com/p/538459335) 选择自己喜欢的字体。 个人感觉中宋(SimSun)还可以 (~ ̄▽ ̄)~ 
\else
  \newfontfamily\kaitichar[Path=./fonts/, AutoFakeBold=3]{Kaiti.ttf} % 设置楷体字体,指定字体文件的路径
  \NewDocumentCommand \templatefont   { } { \kaishu \kaitichar } % 使用模板字体代替直接使用楷书
  \setCJKfamilyfont{Times New Roman}[Path=./fonts/, AutoFakeBold=3]{TimesNewRoman.ttf} % 设置Times New Roman字体,指定字体文件的路径
  \setmainfont{Times New Roman}  % 设置主字体
  % \setmainfont[Path=./fonts/]{TimesNewRoman.ttf} % 设置主字体
  \setCJKmainfont[Path=./fonts/, AutoFakeBold=3]{Kaiti.ttf} % 设置CJK主字体
\fi

%%%————————————————————————————————行距
%设置行距 1.5倍
\renewcommand{\baselinestretch}{1.5}
\XeTeXlinebreaklocale "zh" % 中文断行


%%%————————————————————————————————图片与表格

% 对于表格,推荐使用Excel的插件Excel2LaTeX自动生成LaTeX代码。详见: https://www.zhihu.com/question/307970489/answer/2305355098

%图片
\usepackage{graphicx} %插图宏包
\usepackage{float} %设置图片浮动位置的宏包
\usepackage{subfigure} %插入多图时用子图显示的宏包
\usepackage[section]{placeins} %避免浮动体跨过\section
\usepackage{enumerate} %设置列表环境的包
\usepackage{enumitem} % 增强自定义Bullet(即各种List前的那个标记)
\setlist[enumerate]{
  label={\templatefont \bfseries \hspace{1em} \color{MsBlue}(\arabic*)},
  leftmargin=0em, % 列表的左边界
  itemindent=4em, % item首行缩进
  itemsep=0em, % 列表项之间的垂直间距。
  labelsep=0.1pt,  % label与content的距离
  parsep=0em,  % 两个段落之间的垂直间距。效果上与itemsep类似。
  topsep=0em % 列表与上一段对象的垂直间距。
  % after=\vspace{5pt}
}
% \newcommand{\itemtitlefont}[1]{\textbf{\color{MsBlue} #1}} % 定义item小标题字体;
\newcommand{\itemtitlefont}[1]{{\bfseries \color{MsBlue} #1}} % 定义item小标题字体;
\usepackage{caption}
\captionsetup[figure]{name={图},font={footnotesize,stretch=1.25},labelsep=period,labelfont=bf, singlelinecheck=off, justification=centering} 

%表格
\usepackage{booktabs} %表格
\usepackage{tabularx} %表格宽度
\usepackage{multirow} %多行合并
\usepackage{longtable} % latex 多页显示同一表格
\usepackage{tabu} % 使用longtabu(tabularx + longtable)将长文本在单元格多行显示。
% \captionsetup[table]{name={表},font={footnotesize,stretch=1.25},labelsep=period,labelfont=bf, singlelinecheck=off, justification=centering} 


%%%————————————————————————————————超链接
\usepackage[
	colorlinks,
	urlcolor=MsBlue, % 超链接的颜色。默认是芭比粉
	linkcolor=black,
	anchorcolor=black,
	citecolor=black,
	CJKbookmarks=True
]{hyperref}


%%%————————————————————————————————标题计数

% [latex 标题、段落及行距 - 简书](https://www.jianshu.com/p/d7848f815e5f/)
% [LaTeX 中的行间距](https://yinguobing.com/linespace-in-latex/)

\usepackage{titlesec}

\setcounter{secnumdepth}{4} %显示3级目录,然后再具体定制不同的目录
% \setcounter{tocdepth}{4}

% Define section font
\newcommand{\sectionzihao}{\fontsize{14pt}{\baselineskip}\selectfont}
\newcommand{\subsectionzihao}{\fontsize{14pt}{14pt}\selectfont}
\newcommand{\subsubsectionzihao}{\fontsize{13.5pt}{20pt}\selectfont}


% Custom format for section
\titleformat{\section}
  {\color{MsBlue} \sectionzihao \templatefont} % format
  {\hspace{1.45em}  } % label
  {0pt} % separation
  {}   % before-code

% Custom format for subsection
\titleformat{\subsection}
  {\color{MsBlue} \subsectionzihao \templatefont \linespread{1}} % format
  {} % label
  {0pt} % separation
  {}   % before-code

% Custom format for subsubsection
\titleformat{\subsubsection}
  {\color{MsBlue} \subsubsectionzihao \templatefont \bfseries} % format
  {\hspace{1.2em}  \textnormal{\templatefont \arabic{subsection}.\arabic{subsubsection}}} % label
  {0.5em} % separation
  {}   % before-code

% Custom format for subsubsubsection
\titleclass{\subsubsubsection}{straight}[\subsubsection]
\newcounter{subsubsubsection}[subsubsection]
\renewcommand\thesubsubsubsection{(\arabic{subsubsection})}
\titleformat{\subsubsubsection}
  {\templatefont \bfseries} % format \color{MsBlue} 
  {\hspace{1em} (\arabic{subsubsubsection})} % label
  {0.5pt} % separation
  {}   % before-code
\makeatletter
\def\toclevel@subsubsubsection{4}
\def\l@subsubsubsection{\@dottedtocline{4}{7.0em}{4em}}
\makeatother

% Define line stretch for section
\titlespacing*{\section}
  {0pt} % Left indentation
  {0pt plus 0pt minus 0pt} % Space before the title
  {0pt plus 0pt minus 0pt} % Space after the title

% Define line stretch for subsection
\titlespacing*{\subsection}
  {0pt} % Left indentation
  {0pt plus 0pt minus 0pt} % Space before the title
  {0pt plus 0pt minus 0pt} % Space after the title

% Define line stretch for subsubsection
\titlespacing*{\subsubsection}
  {0pt} % Left indentation
  {0pt plus 0pt minus 0pt} % Space before the title
  {0pt plus 0pt minus 0pt} % Space after the title

% Define line stretch for subsubsection
\titlespacing*{\subsubsubsection}
  {0pt} % Left indentation
  {0pt plus 0pt minus 0pt} % Space before the title
  {0pt plus 0pt minus 0pt} % Space after the title


%%%————————————————————————————————代码
\usepackage{listings} %代码块
\usepackage{xcolor}%代码块
\renewcommand{\lstlistingname}{Code}

% 参考:
% https://www.youtube.com/watch?v=qxkQgG1Y0bY
% https://zhuanlan.zhihu.com/p/65441079

% \lstset{
% 	% language=bash,  %代码语言使用的是matlab
% 	frame=shadowbox, %把代码用带有阴影的框圈起来
% 	rulesepcolor=\color{red!20!green!20!blue!20},%代码块边框为淡青色
% 	keywordstyle=\color{blue!90}\bfseries, %代码关键字的颜色为蓝色,粗体
% 	commentstyle=\color{red!10!green!70}\textit,% 设置代码注释的颜色
% 	showstringspaces=true,%不显示代码字符串中间的空格标记
% 	numbers=left, % 显示行号
% 	numberstyle=\tiny,    % 行号字体
% 	stringstyle=\ttfamily, % 代码字符串的特殊格式
% 	breaklines=false, %对过长的代码自动换行
% 	extendedchars=false,  %解决代码跨页时,章节标题,页眉等汉字不显示的问题
% 	escapebegin=\begin{CJK*},escapeend=\end{CJK*},% 代码中出现中文必须加上,否则报错
% 	texcl=true
% }

\lstdefinestyle{codestyle01}{
	backgroundcolor=\color{gray!12},
	%basicstyle=\ttfamily\small,
	basicstyle=\zihao{-5}\ttfamily,
	commentstyle=\color{green!60!black},
	keywordstyle=\color{magenta},
	stringstyle=\color{blue!50!red},
	showstringspaces=false,
	numbers=none,
	numberstyle=\footnotesize\color{gray},
	numbersep=10pt,
	stepnumber=2, %左边数字的步幅
	tabsize=4,
	frame=TB, %顶部和底部有两条竖线。
  breaklines=true % 添加此行以启用自动换行
	% frame=tblr
	% frame=L,
	% framerule=1pt,
	% rulecolor=\color{red}
}


%%%————————————————————————————————其它设置


\makeatletter	
	\setlength{\@fptop}{0pt} % 图片置顶展示(而不是居中)
  % \setlength{\textfloatsep}{1cm plus 1.0pt minus 2.0pt} % 浮动体(如figure)之间的垂直间距
  \setlength{\intextsep}{0.5cm plus 1.0pt minus 2.0pt} % 浮动体(如figure)与文字的垂直间距
\makeatother

% 去除页码
\pagestyle{empty}

% 汉字与标点的间距;
% \punctstyle{banjiao} % 所有标点半角
% \punctstyle{kaiming} % 部分的标点半角
\punctstyle{hangmobanjiao} % 末半角式,仅行末挤压。

% 参考文献中的链接字体用的是texttt
\renewcommand{\ttdefault}{tnr} % 设置默认等宽字体为 Times New Roman

% 自定义第4级小标题
% \newcommand{\ssssubtitle}[1]{\ding{\numexpr171+#1\relax}} % 类似①
\newcommand{\ssssubtitle}[1]{\textcircled{\raisebox{-0.8pt}{\xiaosihao #1}}}
% \newcommand{\ssssubtitle}[1]{#1)} % 类似1)
% \newcommand{\blackding}[1]{\ding{\numexpr181+#1\relax}}
% \newcommand{\whitedingB}[1]{\ding{\numexpr191+#1\relax}}
% \newcommand{\blackdingB}[1]{\ding{\numexpr201+#1\relax}}

%%%%============================正  文=============================%%%%
\begin{document}

\begin{center}
{\sanhao \templatefont \bfseries \hspace{2em} 报告正文}
\end{center}

{\sihao \templatefont 参照以下提纲撰写,要求内容翔实、清晰,层次分明,标题突出。{\color{MsBlue} \bfseries 请勿删除或改动下述提纲标题及括号中的文字。}}
\vskip -5mm

\section{{\bfseries(一)立项依据与研究内容(建议8000字以下)}:}

\subsection{\hspace{1.45em} 1.~{\bfseries 项目的立项依据}(研究意义、国内外研究现状及发展动态分析,需结合科学研究发展趋势来论述科学意义;或结合国民经济和社会发展中迫切需要解决的关键科技问题来论述其应用前景。附主要参考文献目录)\hspace{-14pt} ;}


\subsubsection{一般正文}

\justifying

\indent\setlength{\parindent}{2em}%首行缩进4字符

happy happy happy happy happy happy happy happy happy happy happy happy happy happy happy happy happy happy happy happy happy happy happy happy happy happy happy happy happy happy happy happy happy happy happy happy happy happy happy happy happy happy happy happy happy happy happy happy happy happy happy happy happy happy happy happy happy happy happy happy happy happy happy happy happy happy happy happy happy happy happy happy happy happy happy happy happy happy happy happy happy\cite{Smith1900} happy happy happy happy happy happy happy happy happy happy happy happy happy happy happy happy happy happy happy happy happy happy happy happy happy happy happy happy happy happy happy happy happy happy happy happy happy happy happy happy happy happy happy happy happy happy happy happy happy happy happy \cite{Smith1900,Piter1992, John1997}.

快乐 快乐 快乐(表\ref{table1}) 快乐 快乐 快乐 快乐 快乐 快乐 快乐 快乐 快乐 快乐 快乐 快乐 快乐 快乐 快乐 快乐 快乐 快乐 快乐 快乐 快乐 快乐 快乐 快乐 快乐 快乐 快乐 快乐 快乐 快乐 快乐 快乐 快乐 快乐 快乐 快乐 快乐 快乐 快乐 快乐 快乐 快乐 快乐 快乐 快乐 快乐 快乐 快乐 快乐 快乐 快乐 快乐 快乐 快乐 快乐 快乐 快乐 快乐 快乐 快乐 快乐 快乐 快乐 快乐 快乐 快乐 快乐 快乐 快乐 快乐 快乐 快乐 快乐 快乐 快乐 快乐 快乐 快乐 快乐 快乐 快乐 快乐 快乐 快乐 快乐 快乐 快乐 快乐 快乐 快乐 快乐 快乐 快乐 快乐 快乐 快乐 快乐 快乐 快乐 快乐 快乐 快乐 快乐 快乐 快乐 快乐 快乐 快乐 快乐 快乐 快乐 快乐 快乐 快乐 快乐 快乐 快乐 快乐 快乐 快乐 快乐 快乐 快乐 快乐 快乐 快乐 快乐(图\ref{zzmx-116}) 快乐 快乐 快乐 快乐 快乐 快乐 快乐 快乐 快乐 快乐 快乐 快乐 快乐 快乐 快乐 快乐 快乐 快乐 快乐。最后,在引用了“\hyperref[sec:有序列表]{\color{MsBlue} \ref{sec:有序列表}~有序列表}”的部分之后,我们会变得更加快乐!

\subsubsubsection{自定义的第4层 - subsubsubsection} 
\indent\setlength{\parindent}{2em}

这里是强制换行的。在某些时候可能是有用,比如在立项依据中添加第4层小标题。当然,也可以设置颜色,有需要的可以试试看!

\subsubsection{有序列表(enumerate)}\label{sec:有序列表}
\indent\setlength{\parindent}{2em}

这里的列表格式是特别定制的,在写研究内容、研究目标、拟解决的关键问题等部分时很好用。

\begin{enumerate}
    \item \itemtitlefont{项目1}:\ssssubtitle{1}快乐 快乐。快乐 快乐。快乐 快乐。快乐 快乐。快乐 快乐。快乐 快乐。快乐 快乐。\ssssubtitle{2}快乐 快乐。快乐 快乐。快乐 。快乐 快乐(图\ref{zzmx-mobile-105}) 。
    \item \itemtitlefont{项目2}:快乐 快乐。快乐 快乐。快乐 快乐。快乐 快乐。快乐 快乐。快乐 快乐。快乐 快乐。快乐 快乐。快乐 快乐。快乐 快乐。快乐 快乐。快乐 快乐。
\end{enumerate}


\begin{table}[htbp]
    \centering
    % \footnotesize %字体大小 \tiny;\scriptsize;\footnotesize;\small;\normalsize;\large ;\Large \LARGE \huge \Huge
    \fontsize{11}{11}\selectfont
    \caption{\textbf{我们的爱过了就不再回来}}
    \begin{tabular}{lllll}
    \toprule
            & \textbf{Characteristics} & \textbf{High (n=137)} & \textbf{Low (n=235)} & \textbf{\textit{P} value} \\
    \toprule
    \textbf{Gender (\%)} & Male  & 85 (62.0) & 154 (65.5) & 0.572 \\
            & Female & 52 (38.0) & 81 (34.5) &  \\
    \textbf{Age (mean (SD))} &       & 65.50 (10.22) & 66.06 (10.95) & 0.628 \\
    \textbf{Race (\%)} & White & 98 (81.0) & 139 (69.5) & 0.054 \\
            & Asian & 21 (17.4) & 51 (25.5) &  \\
            & Other & 2 (1.7) & 10 (5.0) &  \\
    \textbf{Tumor position (\%)} & GEJ   & 17 (12.9) & 27 (11.9) & 0.03 \\
            & Cardia & 11 (8.3) & 36 (15.9) &  \\
            & Fundus & 48 (36.4) & 54 (23.8) &  \\
            & Body  & 7 (5.3) & 23 (10.1) &  \\
            & Antrum & 49 (37.1) & 87 (38.3) &  \\
    \textbf{Pathology (\%)} & Intestinal & 65 (52.4) & 167 (78.8) & <0.001 \\
            & Diffuse & 52 (41.9) & 35 (16.5) &  \\
            & Mixed & 7 (5.6) & 10 (4.7) &  \\
    \textbf{Grade (\%)} & G1    & 1 (0.7) & 9 (4.0) & <0.001 \\
            & G2    & 28 (20.6) & 108 (47.6) &  \\
            & G3    & 107 (78.7) & 110 (48.5) &  \\
    \textbf{T stage (\%)} & T1    & 2 (1.6) & 17 (7.3) & 0.009 \\
            & T2    & 24 (18.6) & 42 (17.9) &  \\
            & T3    & 38 (29.5) & 91 (38.9) &  \\
            & T4    & 65 (50.4) & 84 (35.9) &  \\
    \textbf{N stage (\%)} & N0    & 37 (29.1) & 76 (33.2) & 0.504 \\
            & Np    & 90 (70.9) & 153 (66.8) &  \\
    \textbf{M stage (\%)} & M0    & 121 (92.4) & 206 (93.2) & 0.933 \\
            & M1    & 10 (7.6) & 15 (6.8) &  \\
    \textbf{Stage (\%)} & I     & 10 (8.3) & 37 (17.1) & 0.171 \\
            & II    & 40 (33.3) & 66 (30.6) &  \\
            & III   & 60 (50.0) & 98 (45.4) &  \\
            & IV    & 10 (8.3) & 15 (6.9) &  \\
    \textbf{MSI status (\%)} & MSI-H & 21 (21.0) & 26 (19.0) & 0.001 \\
            & MSI-L & 6 (6.0) & 32 (23.4) &  \\
            & MSS   & 73 (73.0) & 79 (57.7) &  \\
    \textbf{EBV infection (\%)} & Positive & 18 (18.0) & 5 (3.6) & 0.001 \\
            & Negative & 82 (82.0) & 132 (96.4) &  \\
    \textbf{Purity (mean (SD))} &       & 0.40 (0.17) & 0.57 (0.19) & <0.001 \\
    \textbf{Ploidy (mean (SD))} &       & 2.37 (0.67) & 2.69 (0.90) & 0.001 \\
    \bottomrule
    \end{tabular}%
    \captionsetup{font={footnotesize,stretch=1.25},justification=raggedright}
    \label{table1}%
\end{table}%

\clearpage

\subsubsection{研究现状}

\indent\setlength{\parindent}{2em}%首行缩进4字符

只是永远 我都放不开只是永远 我都放不开只是永远 我都放不开只是永远 我都放不开只是永远 我都放不开只是永远 我都放不开只是永远 我都放不开只是永远 我都放不开只是永远 我都放不开只是永远 我都放不开只是永远 我都放不开只是永远 我都放不开只是永远 我都放不开只是永远 我都放不开只是永远 我都放不开只是永远 我都放不开只是永远 我都放不开只是永远 我都放不开只是永远 我都放不开

最后的温暖 你给的温暖 最后的温暖 你给的温暖最后的温暖 你给的温暖最后的温暖 你给的温暖最后的温暖 你给的温暖最后的温暖 你给的温暖最后的温暖 你给的温暖最后的温暖 你给的温暖最后的温暖 你给的温暖最后的温暖 你给的温暖

\begin{figure}[!th]
    \begin{center}
    \includegraphics[width=6in]{figures/zzmx-115.jpg}
    \caption{直到现在我还默默地等待。\\
	\raggedright \justifying \noindent
	我们的爱我们的爱我们的爱我们的爱我们的爱我们的爱我们的爱我们的爱我们的爱我们的爱我们的爱我们的爱我们的爱我们的爱我们的爱我们的爱我们的爱我们的爱我们的爱我们的爱我们的爱我们的爱我们的爱我们的爱我们的爱我们的爱
    }
    \label{zzmx-116}
    \end{center}
\end{figure}

\clearpage

\subsubsection{小结}

\indent\setlength{\parindent}{2em}%首行缩进4字符

已变成你的负担已变成你的负担已变成你的负担已变成你的负担已变成你的负担已变成你的负担已变成你的负担已变成你的负担

\begin{figure}[!th]
    \begin{center}
    \includegraphics[width=5in]{figures/zzmx-mobile-105.jpg}
    \caption{我们的爱,我明白。\\
	\raggedright \justifying \noindent
	已变成你的负担已变成你的负担已变成你的负担已变成你的负担已变成你的负担已变成你的负担已变成你的负担已变成你的负担已变成你的负担已变成你的负担已变成你的负担已变成你的负担已变成你的负担已变成你的负担已变成你的负担已变成你的负担已变成你的负担已变成你的负担已变成你的负担已变成你的负担已变成你的负担已变成你的负担已变成你的负担已变成你的负担已变成你的负担已变成你的负担已变成你的负担已变成你的负担
    }
    \label{zzmx-mobile-105}
    \end{center}
\end{figure}

\clearpage

% 参考文献
\input{references/reference.tex}


\subsection{\hspace{1.45em} 2.~{\bfseries 项目的研究内容、研究目标,以及拟解决的关键科学问题}(此部分为重点阐述内容)\hspace{-14pt} {\bfseries ;}}

\input{extraTex/1.2.内容目标问题.tex}

\subsection{\hspace{1.45em}  3.~{\bfseries 拟采取的研究方案及可行性分析}(包括研究方法、技术路线、实验手段、关键技术等说明)\hspace{-14pt} ;}

\input{extraTex/1.3.方案及可行性.tex}

\subsection{\hspace{1.45em}  4.~{\bfseries 本项目的特色与创新之处;}}

\input{extraTex/1.4.特色与创新.tex}


\subsection{\hspace{1.5em}  5.~{\bfseries 年度研究计划及预期研究结果}(包括拟组织的重要学术交流活动、国际合作与交流计划等)\hspace{-14pt} 。}

\input{extraTex/1.5.研究计划.tex}


% \vskip -5mm %可以通过类似的命令微调行距以使得排版美观


\section{{\bfseries(二)研究基础与工作条件}} %2024


\subsection{\hspace{1.45em}  1.~{\bfseries 研究基础}(与本项目相关的研究工作积累和已取得的研究工作成绩)\hspace{-14pt} ;}


\input{extraTex/2.1.研究基础.tex}

\subsection{\hspace{1.45em}  2.~{\bfseries 工作条件}(包括已具备的实验条件,尚缺少的实验条件和拟解决的途径,包括利用国家实验室、全国重点实验室和部门重点实验室等研究基地的计划与落实情况)\hspace{-14pt}  ;}

\input{extraTex/2.2.工作条件.tex}

\subsection{\hspace{1.45em}  3.~{\bfseries 正在承担的与本项目相关的科研项目情况}(申请人和主要参与者正在承担的与本项目相关的科研项目情况,包括国家自然科学基金的项目和国家其他科技计划项目,要注明项目的资助机构、项目类别、批准号、项目名称、获资助金额、起止年月、与本项目的关系及负责的内容等)\hspace{-14pt} ;}

\input{extraTex/2.3.承担项目.tex}

\subsection{\hspace{1.45em}  4.~{\bfseries 完成国家自然科学基金项目情况}(对申请人负责的前一个已资助期满的科学基金项目(项目名称及批准号)完成情况、后续研究进展及与本申请项目的关系加以详细说明。另附该项目的研究工作总结摘要(限500字)和相关成果详细目录)\hspace{-14pt} 。}

\input{extraTex/2.4.项目完成情况.tex}


\section{{\bfseries (三)其他需要说明的情况}} %2024

\subsection{\hspace{1.45em}  1.~申请人同年申请不同类型的国家自然科学基金项目情况(列明同年申请的其他项目的项目类型、项目名称信息,并说明与本项目之间的区别与联系;已收到自然科学基金委不予受理或不予资助决定的,无需列出)\hspace{-14pt} 。}


\input{extraTex/3.1.不同类型国基情况.tex}

\subsection{\hspace{1.45em}  2.~具有高级专业技术职务(职称)的申请人或者主要参与者是否存在同年申请或者参与申请国家自然科学基金项目的单位不一致的情况;如存在上述情况,列明所涉及人员的姓名,申请或参与申请的其他项目的项目类型、项目名称、单位名称、上述人员在该项目中是申请人还是参与者,并说明单位不一致原因。}

\input{extraTex/3.2.同年单位不一致.tex}

\subsection{\hspace{1.45em}  3.~具有高级专业技术职务(职称)的申请人或者主要参与者是否存在与正在承担的国家自然科学基金项目的单位不一致的情况;如存在上述情况,列明所涉及人员的姓名,正在承担项目的批准号、项目类型、项目名称、单位名称、起止年月,并说明单位不一致原因。}

\input{extraTex/3.3.承担中单位不一致.tex}


\subsection{\hspace{1.45em}  4.~同年以不同专业技术职务(职称)申请或参与申请科学基金项目的情况(应详细说明原因)。}

\justifying

\indent\setlength{\parindent}{2em}%首行缩进4字符

无。


\subsection{\hspace{1.45em}  5.~其他。}

\justifying

\indent\setlength{\parindent}{2em}%首行缩进4字符

无。

\clearpage
\end{document}