
\justifying

\subsubsection{研究内容}

\subsubsubsection{职业发展轨迹纵向研究}
\indent 本研究以佐佐木希2002-2025年职业发展为案例,梳理职业路径,划分为五阶段(2002-2008时尚模特期、2008-2010转型期、2010-2017多领域拓展期、2017-2020成熟期、2020-2025重塑期),识别转型触发点。表\ref{tab:career_stages}展示了各阶段的核心特征与关键事件。

\begin{table}[htbp]
  \centering
  \fontsize{11}{11}\selectfont
  \caption{\textbf{佐佐木希职业发展五阶段特征对比}}
  \begin{tabular}{lcccc}
  \toprule
  \textbf{阶段} & \textbf{时间跨度} & \textbf{核心领域} & \textbf{代表作品/事件} & \textbf{转型驱动力} \\
  \midrule
  时尚模特期 & 2002-2008 & 平面模特 & 《Pinky》《Non-no》 & 年龄增长 \\
  转型期 & 2008-2010 & 影视试水 & 《热血街区》 & 职业拓展 \\
  多领域拓展期 & 2010-2017 & 影视+综艺 & 多部电视剧 & 市场需求 \\
  成熟期 & 2017-2020 & 全方位艺人 & 婚姻+事业 & 个人规划 \\
  重塑期 & 2020-2025 & 新媒体转型 & YouTube频道 & 危机应对 \\
  \bottomrule
  \end{tabular}
  \captionsetup{font={footnotesize,stretch=1.25},justification=raggedright}
  \label{tab:career_stages}
\end{table}

\subsubsubsection{跨媒介形象构建机制}
\indent 分析不同媒介平台形象呈现策略,探讨"娃娃形象"的建构演变。

\subsubsubsection{品牌韧性机制}
\indent 以2020年丈夫出轨风波为关键事件,分析个人品牌面临重大负面冲击时的应对策略。

\subsubsubsection{新媒体转型策略}
\indent 研究从传统媒体向YouTube等新媒体平台的转型路径。

\subsubsection{研究目标}

\subsubsubsection{理论目标}
\indent 本项目总体目标是构建基于东亚文化语境的艺人职业发展动态分析框架。提出"艺人职业转型三阶段模型"(探索期-稳定期-重塑期),建立"跨媒介形象一致性测量指标体系"。

\subsubsubsection{方法学目标}
\indent 开发"艺人职业轨迹编码手册",构建"舆情事件影响评估模型"。

\subsubsubsection{应用目标}
\indent 为娱乐产业从业者提供"艺人职业规划决策支持工具"。

\subsubsection{拟解决的关键科学问题}

\subsubsubsection{艺人职业转型的临界点识别问题}
\indent 如何科学识别关键转型节点?哪些内外部因素起决定性作用?

\subsubsubsection{跨媒介形象的一致性维持问题}
\indent 如何在保持差异化形象的同时确保一致性?

\subsubsubsection{个人品牌韧性的边界条件问题}
\indent 何种负面事件可通过特定策略修复?

\subsubsection{研究方法}

\subsubsubsection{纵向案例研究法}
\indent 本研究采用混合研究方法。收集完整职业档案(媒体报道约5000篇、影视作品约30部、商业代言约50个、社交媒体数据等),采用时序分析识别关键转折点。

\subsubsubsection{内容分析法}
\indent 对时尚杂志约200期进行编码分析,提炼形象演变规律;对影视作品角色类型进行系统分类。编码工作由两名研究者独立完成,采用Cohen's Kappa系数评估信度(K>0.8)。

\subsubsubsection{舆情分析法}
\indent 收集2020年出轨风波期间社交媒体舆情数据约10万条,运用情感分析和主题模型评估公众情绪变化。采用LDA模型进行主题分析,使用BERT预训练模型进行情感分类(准确率目标85\%以上)。

\subsubsubsection{比较研究法}
\indent 选取同时期其他日本女性艺人(滨崎步、新垣结衣、石原里美等)作为对照案例,对比中日韩三国艺人发展模式异同。表\ref{tab:research_methods}总结了本研究采用的四种主要研究方法及其应用场景。

\begin{table}[htbp]
  \centering
  \fontsize{11}{11}\selectfont
  \caption{\textbf{研究方法体系与应用场景}}
  \begin{tabular}{lccc}
  \toprule
  \textbf{研究方法} & \textbf{数据来源} & \textbf{分析工具} & \textbf{信度指标} \\
  \midrule
  纵向案例研究法 & 媒体报道5000篇 & 时序分析 & - \\
  内容分析法 & 时尚杂志200期 & 编码分析 & Cohen's Kappa>0.8 \\
  舆情分析法 & 社交媒体10万条 & BERT+LDA & 准确率>85\% \\
  比较研究法 & 多案例对照 & 跨案例分析 & - \\
  \bottomrule
  \end{tabular}
  \captionsetup{font={footnotesize,stretch=1.25},justification=raggedright}
  \label{tab:research_methods}
\end{table}

对于舆情情感分析,本研究采用基于BERT的情感分类模型,其情感得分计算公式为:
\begin{equation}
S(t) = \frac{1}{N} \sum_{i=1}^{N} w_i \cdot s_i(t)
\label{eq:sentiment_score}
\end{equation}
其中,$S(t)$表示时刻$t$的整体情感得分,$N$为样本数量,$w_i$为第$i$条评论的权重(基于转发量和点赞数),$s_i(t)$为第$i$条评论的情感极性得分(范围:-1到+1)。

\subsubsection{技术路线}

\subsubsubsection{数据收集与预处理}
\indent 本研究技术路线分四阶段(图\ref{techroute})。建立多源异构数据库。

\subsubsubsection{职业轨迹建模}
\indent 应用序列分析识别职业发展典型路径,构建"职业状态转移矩阵"。

\subsubsubsection{形象分析}
\indent 采用深度学习方法对影像资料进行特征提取,运用NLP技术分析文本资料。

\subsubsubsection{模型验证与应用}
\indent 通过交叉验证和敏感性分析评估模型稳健性。

\begin{figure}[!th]
\begin{center}
\includegraphics[width=0.8\linewidth]{figures/zzmx-115.jpg}
\caption{佐佐木希职业发展的多阶段转型模型示意图。}
\label{techroute}
\end{center}
\end{figure}

\subsubsection{关键技术}

\subsubsubsection{多模态数据融合}
\indent 整合文本、图像、视频等多模态数据,构建统一知识图谱。

\subsubsubsection{职业轨迹序列挖掘}
\indent 应用频繁序列模式挖掘算法识别职业发展典型模式。

\subsubsubsection{舆情情感分析}
\indent 基于BERT、GPT等预训练模型进行日语文本情感分析(准确率85\%以上)。

\subsubsection{可行性分析}

\subsubsubsection{理论可行}
\indent 建立在成熟的职业发展理论、品牌管理理论、传播学理论基础之上。

\subsubsubsection{数据可行}
\indent 佐佐木希公开资料丰富,Wikipedia、MyDramalist、AsianWiki等数据库提供充足数据源。

\subsubsubsection{技术可行}
\indent 研究团队具备NLP、计算机视觉、数据挖掘技术储备。
