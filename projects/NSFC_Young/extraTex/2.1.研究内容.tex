
\justifying

\subsubsection{研究内容}

\indent\setlength{\parindent}{2em}

本研究以佐佐木希(Nozomi Sasaki)2002-2025年的职业发展为案例,围绕"艺人职业转型的动态机制与形象管理策略"核心主题,开展以下四项研究内容:

\begin{enumerate}
    \item \itemtitlefont{职业发展轨迹的纵向研究}:系统梳理佐佐木希从14岁出道至今的完整职业路径,划分为时尚模特期(2002-2008)、转型期(2008-2010)、多领域拓展期(2010-2017)、成熟期(2017-2020)、重塑期(2020-2025)五个关键阶段,识别各阶段的转型触发点与关键决策节点,构建艺人职业发展的阶段性模型。

    \item \itemtitlefont{跨媒介形象构建机制研究}:分析佐佐木希在不同媒介平台(时尚杂志、影视作品、音乐作品、综艺节目、社交媒体)上的形象呈现策略,探讨"娃娃形象"(Doll-like Image)的建构、演变与解构过程,研究其形象如何在保持核心特征的同时实现适应性调整。

    \item \itemtitlefont{品牌韧性机制研究}:以2020年丈夫出轨风波为关键事件,分析个人品牌在面临重大负面冲击时的应对策略与修复机制,探讨"沉默-原谅-重启"策略在日本文化语境下的有效性以及公众舆论的转变规律。

    \item \itemtitlefont{新媒体转型策略研究}:研究佐佐木希从传统媒体向YouTube(2024年开设个人频道)等新媒体平台的转型路径,分析"Vlog"模式对艺人-粉丝关系重构的影响,探讨"真实性与表演性"的平衡策略。
\end{enumerate}

\subsubsection{研究目标}

\indent\setlength{\parindent}{2em}

本项目的总体目标是构建基于东亚文化语境的艺人职业发展动态分析框架。具体目标包括:

\begin{enumerate}
    \item \itemtitlefont{理论目标}:提出"艺人职业转型的三阶段模型"(探索期-稳定期-重塑期),填补现有理论对东亚娱乐产业适用性的空白;建立"跨媒介形象一致性测量指标体系",为艺人品牌管理提供量化评估工具。

    \item \itemtitlefont{方法学目标}:开发"艺人职业轨迹编码手册",实现职业发展过程的系统化分析;构建"舆情事件影响评估模型",定量评估负面事件对艺人品牌价值的影响程度与恢复周期。

    \item \itemtitlefont{应用目标}:为娱乐产业从业者提供"艺人职业规划决策支持工具",为女性艺人应对职业生涯中的性别化挑战提供策略指导。
\end{enumerate}

\subsubsection{拟解决的关键科学问题}

\indent\setlength{\parindent}{2em}

\begin{enumerate}
    \item \itemtitlefont{艺人职业转型的临界点识别问题}:如何科学识别艺人职业生涯中的关键转型节点?哪些内部因素(个人决策、能力储备)和外部因素(产业变迁、市场机会)在转型决策中起决定性作用?

    \item \itemtitlefont{跨媒介形象的一致性维持问题}:艺人在不同媒介平台需要呈现差异化形象以满足受众期待,但如何确保这些差异化形象之间的一致性,避免品牌价值稀释?

    \item \itemtitlefont{个人品牌韧性的边界条件问题}:何种类型的负面事件可以通过特定策略修复?哪些因素决定了品牌修复的成功率?"原谅策略"在不同文化语境下的适用性边界在哪里?
\end{enumerate}

\subsubsection{研究方法}

\indent\setlength{\parindent}{2em}

本研究采用混合研究方法(Mixed Methods Approach),结合定量与定性分析:

\begin{enumerate}
    \item \itemtitlefont{纵向案例研究法}:收集佐佐木希2002-2025年的完整职业档案,包括媒体报道、影视作品、商业代言、社交媒体数据等,采用"时序分析"(Temporal Analysis)识别职业发展的关键转折点与转型模式。

    \item \itemtitlefont{内容分析法}:对佐佐木希在《Pinky》、《Non-no》、《Oggi》、《With》等时尚杂志的封面与内页报道进行编码分析,提炼其形象呈现的演变规律;对其主要影视作品中的角色类型进行系统分类。

    \item \itemtitlefont{舆情分析法}:收集2020年出轨风波期间及后续的社交媒体舆情数据(Twitter、Instagram等),运用情感分析法(Sentiment Analysis)和主题模型(Topic Modeling)评估公众情绪的变化轨迹。

    \item \itemtitlefont{比较研究法}:选取同时期其他日本女性艺人作为对照案例,比较不同职业路径的差异;对比中、日、韩三国娱乐产业中艺人职业发展模式的异同。
\end{enumerate}

\subsubsection{技术路线}

\indent\setlength{\parindent}{2em}

本研究的技术路线分为四个阶段(图\ref{techroute}):

\begin{enumerate}
    \item \itemtitlefont{数据收集与预处理阶段}:通过合法渠道收集佐佐木希的公开资料,建立多源异构数据库,实现文本、图像、视频等多种模态数据的整合存储。

    \item \itemtitlefont{职业轨迹建模阶段}:应用序列分析(Sequence Analysis)方法,识别职业发展的典型路径与转型模式;构建"职业状态转移矩阵",量化不同职业状态之间的转移概率。

    \item \itemtitlefont{形象分析阶段}:采用深度学习方法(CNN、Transformer)对影像资料进行特征提取,量化形象呈现的演变;运用自然语言处理技术(BERT、GPT)分析文本资料中的形象建构策略。

    \item \itemtitlefont{模型验证与应用阶段}:通过交叉验证(Cross-Validation)和敏感性分析(Sensitivity Analysis)评估模型的稳健性,开发"艺人职业发展决策支持系统"的原型并进行案例测试。
\end{enumerate}

\begin{figure}[!th]
    \begin{center}
    \includegraphics[width=0.9\linewidth]{figures/zzmx-115.jpg}
    \caption{佐佐木希职业发展的多阶段转型模型示意图。}
    \label{techroute}
    \end{center}
\end{figure}

\subsubsection{关键技术}

\indent\setlength{\parindent}{2em}

\begin{enumerate}
    \item \itemtitlefont{多模态数据融合技术}:整合文本、图像、视频、时间序列等多模态数据,构建统一的艺人职业发展知识图谱,采用注意力机制(Attention Mechanism)实现不同模态特征的自动加权融合。

    \item \itemtitlefont{职业轨迹序列挖掘技术}:应用频繁序列模式挖掘(Frequent Sequence Mining)算法,识别职业发展的典型模式;采用马尔可夫链(Markov Chain)预测职业转型的概率与方向。

    \item \itemtitlefont{舆情情感分析技术}:基于预训练语言模型(如BERT、GPT)进行日语文本的情感分析,准确率超过85\%;结合时间序列分析(Time Series Analysis)追踪舆情情绪的动态变化。
\end{enumerate}

\subsubsection{可行性分析}

\indent\setlength{\parindent}{2em}

\begin{enumerate}
    \item \itemtitlefont{理论可行性}:本研究建立在成熟的职业发展理论、品牌管理理论、传播学理论基础之上,理论框架清晰;前期文献调研已确认研究切入点的创新性与科学价值。

    \item \itemtitlefont{数据可行性}:佐佐木希作为公众人物,其公开资料丰富且易于获取,包括Wikipedia、MyDramalist、AsianWiki等权威数据库;日本娱乐产业的媒体报道体系完善,为纵向研究提供了充足的数据源。

    \item \itemtitlefont{技术可行性}:研究团队具备自然语言处理、计算机视觉、数据挖掘等相关技术储备;所需的开源工具(NLTK、OpenCV、TensorFlow、PyTorch等)成熟稳定,技术风险可控。
\end{enumerate}

\clearpage
