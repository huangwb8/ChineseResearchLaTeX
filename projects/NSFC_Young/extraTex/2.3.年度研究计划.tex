
\justifying

\subsubsection{年度研究计划}

\subsubsubsection{第一年研究计划}

\noindent\textbf{研究重点}

\indent 数据收集与预处理、职业轨迹初步建模。

\noindent\textbf{主要工作}

\indent 完成佐佐木希职业发展档案数据库建立(媒体报道、影视作品、商业代言、杂志封面、社交媒体舆情数据),开发数据清洗与预处理流程。完成相关研究文献综述,构建"艺人职业发展三阶段动态模型"理论框架,开发"艺人职业轨迹编码手册"初稿。

\noindent\textbf{年度预期成果}

\indent 建立完整研究数据库;完成文献综述论文1-2篇;构建理论模型框架。

\subsubsubsection{第二年研究计划}

\noindent\textbf{研究重点}

\indent 深度数据分析、模型构建与验证。

\noindent\textbf{主要工作}

\indent 应用序列分析方法识别职业发展关键转型节点与典型路径,构建"职业状态转移矩阵",开发"职业转型预测模型"原型。采用深度学习方法对影像资料进行特征提取,运用NLP技术分析文本资料中的形象建构策略,构建"跨媒介形象一致性测量指标体系"。

\noindent\textbf{年度预期成果}

\indent 完成职业轨迹建模分析;发表实证研究论文2-3篇;开发模型原型系统。

\subsubsubsection{第三年研究计划}

\noindent\textbf{研究重点}

\indent 模型验证、案例比较、成果总结与转化。

\noindent\textbf{主要工作}

\indent 通过交叉验证和敏感性分析评估模型稳健性,选取同时期其他日本女性艺人作为对照案例检验外部效度。撰写研究总报告,开发"艺人职业规划决策支持工具"原型,组织学术研讨会,撰写专著《艺人职业发展的动态机制:基于佐佐木希案例的纵向研究》。

\noindent\textbf{年度预期成果}

\indent 完成模型验证与优化;发表高水平研究论文3-4篇;完成研究总报告;开发决策支持工具原型。

\subsubsubsection{预期研究结果}

\noindent\textbf{理论贡献}

\indent 提出"艺人职业发展三阶段动态模型";建立"品牌韧性"理论框架;构建"跨媒介形象一致性测量指标体系";开发"艺人职业轨迹编码手册";提出多模态数据融合分析框架。

\noindent\textbf{实践价值}

\indent 为经纪公司提供"艺人职业规划决策支持工具";为艺人本人提供"个人品牌管理指南";为政府文化部门制定政策提供实证依据。

\noindent\textbf{学术产出}

\indent 计划发表高水平学术论文6-8篇(SSCI/CSSCI不少于4篇),目标期刊包括Journal of Business Research、Tourism Management、《新闻与传播研究》《现代传播》等;撰写专著1部;参加国际学术会议2-3次;组织专题学术研讨会1-2次。
