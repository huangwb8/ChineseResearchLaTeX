
\justifying

\subsubsection{第一年研究计划}

\indent\setlength{\parindent}{2em}

\textbf{研究重点:数据收集与预处理、职业轨迹初步建模}

完成佐佐木希职业发展档案数据库建立,收集2002-2025年的媒体报道、影视作品目录、商业代言记录、杂志封面数据及2020年出轨风波期间的社交媒体舆情数据,开发数据清洗与预处理流程。完成国内外相关研究的系统文献综述,构建"艺人职业发展三阶段动态模型"的理论框架,开发"艺人职业轨迹编码手册"(Career Trajectory Coding Manual)初稿。

\textbf{年度预期成果:}建立完整的研究数据库;完成文献综述论文1-2篇;构建理论模型框架。

\subsubsection{第二年研究计划}

\indent\setlength{\parindent}{2em}

\textbf{研究重点:深度数据分析、模型构建与验证}

应用序列分析方法(Sequence Analysis)识别佐佐木希职业发展的关键转型节点与典型路径,构建"职业状态转移矩阵"(Career State Transition Matrix),开发"职业转型预测模型"原型。采用深度学习方法(CNN、Vision Transformer)对影像资料进行特征提取,运用自然语言处理技术(BERT、GPT)分析文本资料中的形象建构策略,构建"跨媒介形象一致性测量指标体系"。

\textbf{年度预期成果:}完成职业轨迹建模分析;发表实证研究论文2-3篇;开发模型原型系统。

\subsubsection{第三年研究计划}

\indent\setlength{\parindent}{2em}

\textbf{研究重点:模型验证、案例比较、成果总结与转化}

通过交叉验证(Cross-Validation)和敏感性分析(Sensitivity Analysis)评估"职业转型预测模型"的稳健性,选取同时期其他日本女性艺人作为对照案例检验模型的外部效度。撰写研究总报告,开发"艺人职业规划决策支持工具"原型,组织学术研讨会,撰写专著《艺人职业发展的动态机制:基于佐佐木希案例的纵向研究》。

\textbf{年度预期成果:}完成模型验证与优化;发表高水平研究论文3-4篇;完成研究总报告;开发决策支持工具原型。

\subsubsection{预期研究结果}

\indent\setlength{\parindent}{2em}

\textbf{理论贡献:}提出"艺人职业发展三阶段动态模型"(Exploration-Stabilization-Reinvention),为职业发展理论提供新的理论视角;建立"品牌韧性"(Brand Resilience)理论框架,揭示负面事件后品牌修复的内在机制;构建"跨媒介形象一致性测量指标体系",实现形象管理的量化评估;开发"艺人职业轨迹编码手册",为相关研究提供标准化的研究工具;提出多模态数据融合分析框架,推动文化研究方法的数字化革新。

\textbf{实践价值:}为经纪公司提供"艺人职业规划决策支持工具",提高艺人管理的科学性;为艺人本人提供"个人品牌管理指南",增强职业发展的可持续性;为政府文化部门制定艺人保护政策、完善娱乐产业规范提供实证依据。

\textbf{学术产出:}计划发表高水平学术论文6-8篇(其中SSCI/CSSCI检索论文不少于4篇),目标期刊包括《Journal of Business Research》、《Tourism Management》、《新闻与传播研究》、《现代传播》等;撰写专著1部;参加国际学术会议2-3次(包括ICA、IAMCR等顶级会议);组织专题学术研讨会1-2次。

\clearpage
