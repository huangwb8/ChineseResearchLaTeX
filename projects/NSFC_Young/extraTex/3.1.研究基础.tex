\justifying
\NSFCBodyText

\subsubsection{与本项目相关的研究工作积累}

\subsubsubsection{东亚娱乐产业研究}
申请团队长期致力于文化产业、艺人职业发展、品牌管理研究。前期对东亚娱乐产业“偶像工业”特征深入研究,发表《日本偶像产业的职业化路径与市场机制》等论文,系统分析日本艺人培养体系中的“练习生制度”“事务所制度”等关键机制;对中日韩三国艺人培养体系进行比较研究。

\subsubsubsection{个人品牌理论研究}
对个人品牌理论发展系统梳理,发表《个人品牌理论的前沿进展》文献综述,总结个人品牌建构的“真实性-一致性-可见性”三维度模型;对品牌危机管理进行案例研究,提出“危机后品牌修复的五阶段模型”。

\subsubsubsection{计算方法储备}
掌握自然语言处理、计算机视觉、数据挖掘等计算方法;前期开发“舆情情感分析系统”,集成BERT预训练模型和时序分析算法,情感分类准确率达87\%。

\subsubsection{已取得的研究工作成绩}

\subsubsubsection{论文发表}
在《新闻与传播研究》《国际新闻界》《现代传播》等CSSCI期刊发表相关论文10余篇,其中2篇被《新华文摘》全文转载;在 Journal of Business Research、Tourism Management 等SSCI期刊发表相关论文5篇,总引用超200次。表\ref{tab:research_output}总结了申请团队近五年的主要研究成果。

\begin{table}[htbp]
  \centering
  \fontsize{11}{11}\selectfont
  \caption{\textbf{申请团队近五年研究成果统计}}
  \begin{tabular}{lccc}
  \toprule
  \textbf{成果类型} & \textbf{数量} & \textbf{级别} & \textbf{影响力指标} \\
  \midrule
  CSSCI期刊论文 & 10+ & 权威/核心 & 2篇《新华文摘》转载 \\
  SSCI期刊论文 & 5 & 国际期刊 & 总引用>200次 \\
  国家社科基金 & 1 & 国家级 & 结项等级"优秀" \\
  省部级项目 & 1 & 省部级 & 在研 \\
  学术专著 & 2 & - & 2022、2023年出版 \\
  科研奖励 & 2 & 省部级 & 二等奖1项、三等奖1项 \\
  \bottomrule
  \end{tabular}
  \captionsetup{font={footnotesize,stretch=1.25},justification=raggedright}
  \label{tab:research_output}
\end{table}

\subsubsubsection{项目主持}
主持国家社科基金项目“新媒体环境下文化产业的数字化转型研究”(已结项,鉴定等级“优秀”);主持省部级项目“中日韩文化产业比较研究”(在研)。

\subsubsubsection{专著出版}
出版专著《文化产业与品牌管理》(2022),系统阐述文化产业品牌建设理论;出版教材《计算传播学方法导论》(2023),介绍文本分析、社会网络分析等方法。

\subsubsubsection{科研奖励}
相关研究成果获省部级社会科学优秀成果奖二等奖1项、三等奖1项。

\subsubsection{研究风险的应对措施}

\subsubsubsection{数据获取风险}
部分社交媒体历史数据可能因平台政策限制无法完整获取。应对措施:建立多源数据获取渠道;与日本相关研究机构(东京大学信息学环、早稻田大学文学学术院)建立合作关系;利用合法数据服务提供商获取数据;自主研发网络爬虫工具。

\subsubsubsection{模型有效性风险}
基于单个案例构建的模型可能缺乏普适性。应对措施:采用多案例比较研究设计,引入对照案例;通过交叉验证和敏感性分析评估模型稳健性;充分考虑文化语境的调节作用;邀请知名专家对研究成果评议。

\subsubsubsection{语言障碍风险}
日语资料分析可能存在理解偏差。应对措施:研究团队包括日语专业成员,负责人拥有日语N1级证书;与日本研究者建立合作关系,进行双语校对;采用机器翻译工具初翻,人工精细校对;建立术语对照表。
