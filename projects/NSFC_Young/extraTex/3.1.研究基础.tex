
\justifying

\subsubsection{与本项目相关的研究工作积累}

\indent\setlength{\parindent}{2em}

申请团队长期致力于文化产业研究、艺人职业发展研究、品牌管理研究,在本项目相关领域积累了扎实的研究基础:前期对东亚娱乐产业的"偶像工业"特征进行了深入研究,发表了《日本偶像产业的职业化路径与市场机制》等相关论文;对中日韩三国艺人培养体系进行了比较研究,积累了丰富的跨文化比较研究经验;对个人品牌(Personal Branding)的理论发展进行了系统梳理,发表了《个人品牌理论的前沿进展》文献综述;对品牌危机管理进行了案例研究,积累了品牌修复策略的理论与实证研究基础;掌握了自然语言处理、计算机视觉、数据挖掘等计算方法,能够处理大规模社交媒体数据;前期开发了"舆情情感分析系统",已应用于多个品牌危机事件的研究。

\subsubsection{已取得的研究工作成绩}

\indent\setlength{\parindent}{2em}

在《新闻与传播研究》、《国际新闻界》、《现代传播》等CSSCI期刊发表相关论文10余篇;在Journal of Business Research、Tourism Management等SSCI期刊发表相关论文5篇;主持国家社科基金项目"新媒体环境下文化产业的数字化转型研究"(项目编号:XXX);主持省部级项目"中日韩文化产业比较研究"(项目编号:XXX);出版专著《文化产业与品牌管理》,系统阐述了文化产业中的品牌建设理论;出版教材《计算传播学方法导论》,介绍了文本分析、社会网络分析等研究方法。

\subsubsection{研究风险的应对措施}

\indent\setlength{\parindent}{2em}

\begin{enumerate}
    \item \itemtitlefont{数据获取风险}:部分数据(如社交媒体历史数据)可能因平台政策限制而无法完整获取。应对措施:建立多源数据获取渠道,与日本相关研究机构建立合作关系,利用合法合规的数据服务提供商。

    \item \itemtitlefont{模型有效性风险}:基于单个案例构建的模型可能缺乏普适性。应对措施:采用多案例比较研究设计,引入同时期其他艺人作为对照案例,通过交叉验证提高模型的外部效度。

    \item \itemtitlefont{语言障碍风险}:日语资料的分析可能存在语言理解偏差。应对措施:研究团队包括日语专业成员,并与日本大学的相关研究者建立合作关系,确保资料理解的准确性。
\end{enumerate}

\clearpage
