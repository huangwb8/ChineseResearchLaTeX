
\justifying

\subsubsection{已具备的实验条件}

\indent\setlength{\parindent}{2em}

\textbf{硬件设施}:

依托单位文化产业研究中心配备了完善的科研设施,能够满足本项目的需求:(1)高性能计算集群,配备NVIDIA A100 GPU 4张,总内存512GB,可支持深度学习模型的训练与推理;(2)大容量数据存储系统,总容量500TB,支持多模态研究数据的长期存储与管理;(3)专业级影像处理工作站,配备高精度显示器,用于影像资料的标注与分析;(4)专业录音与视频编辑设备,用于多媒体资料的采集与处理。

\textbf{软件资源}:

实验室已部署完整的数据分析工具链:(1)自然语言处理工具包(NLTK、spaCy、jieba等),支持多语言文本分析;(2)深度学习框架(TensorFlow、PyTorch、Keras),用于构建和训练神经网络模型;(3)计算机视觉工具(OpenCV、PIL、scikit-image),用于影像资料的特征提取;(4)数据可视化工具(Tableau、D3.js、Matplotlib),用于研究结果的呈现;(5)社会科学统计软件(SPSS、Stata、R),用于传统统计分析。

\textbf{数据库资源}:

依托单位购买了以下学术数据库的使用权限:(1)外文数据库:Web of Science、Scopus、JSTOR、SpringerLink、ScienceDirect、EBSCO等;(2)中文数据库:中国知网(CNKI)、万方数据、维普资讯、人大复印报刊资料等;(3)专业数据库:MyDramalist(影视作品数据库)、AsianWiki(亚洲艺人数据库)等。这些数据库为本项目的文献调研与数据收集提供了坚实基础。

\subsubsection{尚缺少的实验条件及拟解决途径}

\indent\setlength{\parindent}{2em}

\textbf{缺少的实验条件}:

\begin{enumerate}
    \item \itemtitlefont{日本本土数据资源访问受限}:
    \ssssubtitle{1}部分日本本土的社交媒体数据(如Twitter日本区历史数据、5ch论坛历史帖子)因地域限制和平台政策,无法完整获取。
    \ssssubtitle{2}日本娱乐产业的一手资料(如经纪公司内部数据、艺人合约模板等)属于商业机密,难以直接获取。

    \item \itemtitlefont{跨文化比较研究资源不足}:
    \ssssubtitle{1}韩国娱乐产业的相关数据(如K-pop艺人职业发展数据)需要与韩国研究机构合作获取。
    \ssssubtitle{2}欧洲与北美娱乐产业的数据资源相对分散,缺乏系统性的获取渠道。
\end{enumerate}

\textbf{拟解决途径}:

\begin{enumerate}
    \item \itemtitlefont{建立国际合作网络}:
    \ssssubtitle{1}与日本东京大学、早稻田大学的相关研究机构建立合作关系,通过学术交流获取日本本土数据资源。
    \ssssubtitle{2}与韩国首尔国立大学、汉阳大学的媒体研究中心建立合作关系,开展中、日、韩三国娱乐产业比较研究。
    \ssssubtitle{3}与欧洲的阿姆斯特丹大学、北美的南加州大学的相关研究者建立联系,拓展跨文化比较研究网络。

    \item \itemtitlefont{利用合法合规的数据服务提供商}:
    \ssssubtitle{1}通过与合法的数据服务提供商(如Brandwatch、Talkwalker等)合作,获取社交媒体历史数据。
    \ssssubtitle{2}利用公开数据集(如Twitter Academic Research Product Track)获取研究所需的基础数据。

    \item \itemtitlefont{开发数据采集工具}:
    \ssssubtitle{1}自主研发网络爬虫工具,在遵守平台服务条款的前提下,公开采集可用数据。
    \ssssubtitle{2}开发多语言情感分析工具,提高对日语、韩语等小语种数据的处理能力。
\end{enumerate}

\subsubsection{利用国家重点实验室等研究基地的计划}

\indent\setlength{\parindent}{2em}

\textbf{计划利用的国家重点实验室}:

\begin{enumerate}
    \item \itemtitlefont{中国人民大学新闻学院社会发展研究基地}:
    \ssssubtitle{1}该基地在传播学研究领域具有深厚积累,可为本项目提供理论指导与方法支持。
    \ssssubtitle{2}计划于第一年派遣研究团队成员赴该基地进行为期3个月的访学交流,学习先进的研究方法。

    \item \itemtitlefont{清华大学新闻与传播学院新媒体研究中心}:
    \ssssubtitle{1}该中心在新媒体传播研究方面处于国内领先地位,可为本项目提供计算传播学方法支持。
    \ssssubtitle{2}计划于第二年与该中心联合举办"文化产业数字化转型"学术研讨会。

    \item \itemtitlefont{中国传媒大学文化产业管理学院}:
    \ssssubtitle{1}该学院在文化产业管理研究方面具有丰富经验,可为本项目提供政策解读与实践指导。
    \ssssubtitle{2}计划于第三年与该学院合作开展"艺人职业发展"专题研究。
\end{enumerate}

\textbf{计划利用的部门重点实验室}:

\begin{enumerate}
    \item \itemtitlefont{文化和旅游部文化产业研究中心}:
    \ssssubtitle{1}该中心掌握文化产业的宏观政策数据,可为项目提供政策环境分析支持。
    \ssssubtitle{2}计划邀请中心专家参与项目咨询,确保研究成果符合政策导向。

    \item \itemtitlefont{国家广播电视总局广播电视规划院}:
    \ssssubtitle{1}该院拥有影视产业的一手统计数据,可为项目提供权威数据支持。
    \ssssubtitle{2}计划于项目执行期与该院建立数据共享机制。
\end{enumerate}

\textbf{落实情况}:

申请团队已与上述研究基地的负责人进行了初步沟通,均表示愿意支持本项目的开展。具体合作计划将在项目获批后进一步细化落实。

\clearpage
