
\justifying

\subsubsection{已具备的实验条件}

\indent\setlength{\parindent}{2em}

\textbf{硬件设施}:依托单位文化产业研究中心配备了完善的科研设施,包括高性能计算集群(配备NVIDIA A100 GPU 4张,总内存512GB)、大容量数据存储系统(总容量500TB)、专业级影像处理工作站、专业录音与视频编辑设备等,能够满足本项目的需求。

\textbf{软件资源}:实验室已部署完整的数据分析工具链,包括自然语言处理工具包(NLTK、spaCy、jieba等)、深度学习框架(TensorFlow、PyTorch、Keras)、计算机视觉工具(OpenCV、PIL、scikit-image)、数据可视化工具(Tableau、D3.js、Matplotlib)、社会科学统计软件(SPSS、Stata、R)等。

\textbf{数据库资源}:依托单位购买了Web of Science、Scopus、JSTOR、SpringerLink、ScienceDirect、EBSCO等外文数据库,中国知网(CNKI)、万方数据、维普资讯等中文数据库,以及MyDramalist、AsianWiki等专业数据库的使用权限,为本项目的文献调研与数据收集提供了坚实基础。

\subsubsection{尚缺少的实验条件及拟解决途径}

\indent\setlength{\parindent}{2em}

\textbf{缺少的实验条件}:部分日本本土的社交媒体数据(如Twitter日本区历史数据、5ch论坛历史帖子)因地域限制和平台政策无法完整获取;日本娱乐产业的一手资料(如经纪公司内部数据、艺人合约模板等)属于商业机密,难以直接获取;韩国、欧洲与北美娱乐产业的数据资源相对分散,缺乏系统性的获取渠道。

\textbf{拟解决途径}:与日本东京大学、早稻田大学,韩国首尔国立大学、汉阳大学,以及欧洲的阿姆斯特丹大学、北美的南加州大学等相关研究机构建立合作关系;通过与合法的数据服务提供商(如Brandwatch、Talkwalker等)合作获取社交媒体历史数据;利用公开数据集(如Twitter Academic Research Product Track)获取研究所需的基础数据;自主研发网络爬虫工具和多语言情感分析工具。

\subsubsection{利用国家重点实验室等研究基地的计划}

\indent\setlength{\parindent}{2em}

计划与中国人民大学新闻学院社会发展研究基地、清华大学新闻与传播学院新媒体研究中心、中国传媒大学文化产业管理学院、文化和旅游部文化产业研究中心、国家广播电视总局广播电视规划院等研究基地建立合作关系。申请团队已与上述研究基地的负责人进行了初步沟通,均表示愿意支持本项目的开展。具体合作计划将在项目获批后进一步细化落实。

\clearpage
