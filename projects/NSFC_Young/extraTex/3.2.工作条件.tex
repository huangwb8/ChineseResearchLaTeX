\justifying
\NSFCBodyText

\subsubsection{已具备的实验条件}

\subsubsubsection{硬件设施}
依托单位文化产业研究中心配备完善科研设施:(1)高性能计算集群:NVIDIA A100 GPU 4张(每张40GB显存),总内存512GB;(2)大容量数据存储系统:总容量500TB,支持多用户并发访问和自动备份;(3)专业级影像处理工作站:高色域显示器(99\% Adobe RGB)和专业级图形卡;(4)专业录音与视频编辑设备。表\ref{tab:hardware_config}详细列出了主要硬件设施配置。

\begin{table}[htbp]
  \centering
  \fontsize{11}{11}\selectfont
  \caption{\textbf{主要硬件设施配置}}
  \begin{tabular}{lccc}
  \toprule
  \textbf{设备类型} & \textbf{配置规格} & \textbf{数量} & \textbf{主要用途} \\
  \midrule
  GPU计算卡 & NVIDIA A100 40GB & 4张 & 深度学习训练 \\
  服务器内存 & DDR4 ECC & 512GB & 大规模数据处理 \\
  存储系统 & 企业级NAS & 500TB & 多模态数据存储 \\
  工作站 & 高色域显示器 & 5台 & 影像处理与分析 \\
  录音设备 & 专业级麦克风 & 2套 & 访谈录制 \\
  \bottomrule
  \end{tabular}
  \captionsetup{font={footnotesize,stretch=1.25},justification=raggedright}
  \label{tab:hardware_config}
\end{table}

\subsubsubsection{软件资源}
实验室部署完整数据分析工具链:(1)自然语言处理:NLTK、spaCy、jieba、HanLP等;(2)深度学习框架:TensorFlow、PyTorch、Keras,支持BERT、GPT等预训练模型;(3)计算机视觉:OpenCV、PIL、scikit-image、detectron2;(4)数据可视化:Tableau、D3.js、Matplotlib、Seaborn;(5)社会科学统计:SPSS、Stata、R;(6)项目管理:Git、GitHub、Slack、Trello。

\subsubsubsection{数据库资源}
依托单位图书馆购买丰富学术数据库:(1)外文数据库:Web of Science、Scopus、JSTOR、SpringerLink、ScienceDirect、EBSCO、SAGE、Wiley等;(2)中文数据库:中国知网(CNKI)、万方数据、维普资讯等;(3)专业数据库:MyDramalist、AsianWiki、Oricon等;(4)统计年鉴:中国统计年鉴、日本统计年鉴等。

\subsubsection{尚缺少的实验条件及拟解决途径}

\subsubsubsection{缺少的实验条件}
(1)日本本土社交媒体数据因地域限制和平台政策无法完整获取;(2)日本娱乐产业一手资料属商业机密;(3)韩国、欧美娱乐产业数据资源分散;(4)大规模视频数据存储与处理需更专业设备;(5)日语情感分析缺乏高质量标注语料。

\subsubsubsection{拟解决途径}
(1)国际合作:与东京大学信息学环、早稻田大学文学学术院、首尔国立大学言论情报学研究所、阿姆斯特丹大学传媒系、南加州大学新闻与传播学院等建立合作关系;(2)数据采购:通过 Brandwatch、Talkwalker、Cision 等获取社交媒体历史数据(预算3-5万元/年);(3)公开数据集:利用 Twitter Academic Research Product Track 等;(4)自主研发:已开发网络爬虫工具和多语言情感分析工具原型;(5)语料库建设:与日本合作机构联合建设日语情感标注语料库(约10万条);(6)设备升级:申请经费升级视频存储和处理设备(预算5-8万元)。

\subsubsection{利用国家重点实验室等研究基地的计划}

\subsubsubsection{合作单位}
申请团队计划与国内高水平研究基地建立紧密合作关系:中国人民大学新闻学院社会发展研究基地、清华大学新闻与传播学院新媒体研究中心、中国传媒大学文化产业管理学院、文化和旅游部文化产业研究中心、国家广播电视总局广播电视规划院。

\subsubsubsection{合作方式}
学者互访(每年邀请合作基地知名学者来校讲学1-2次,项目负责人赴合作基地交流访问1-2次);联合研究(在重大理论问题上开展联合研究,共同发表高水平论文);资源共享(共享部分研究数据和设备资源);人才培养(共同指导研究生,开展学生访学交流)。
