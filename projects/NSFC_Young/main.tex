%!TEX program = xelatex

% 国家自然科学基金NSFC-青年科学基金项目(C类)-正文(2026年版)

% 测试环境:[使用VSCode编写LaTeX - 知乎](https://zhuanlan.zhihu.com/p/38178015)

% 编译顺序: xelatex -> bibtex -> xelatex -> xelatex

% 参考 & 鸣谢:
% [Ruzim/NSFC-application-template-latex: 国家自然科学基金申请书正文(面上项目)LaTeX 模板(非官方)](https://github.com/Ruzim/NSFC-application-template-latex)
% [Readon/NSFC-application-template-latex: 国家自然科学基金申请书正文(面上项目)模板(非官方)](https://github.com/Readon/NSFC-application-template-latex)
% [Aligning equations with amsmath - Overleaf, Online LaTeX Editor](https://www.overleaf.com/learn/latex/Aligning_equations_with_amsmath):各种格式的公式写法


\documentclass[12pt,UTF8,AutoFakeBold=2.5,a4paper,fontset=none]{ctexart} %默认小四号字。允许楷体粗体。

%%%%============================系统设置===========================%%%%

\usepackage[english]{babel} %支持混合语言
\usepackage{xcolor}
\usepackage{graphicx} 
\usepackage{amsmath} %更多数学符号
\usepackage{bm} % 粗体数学符号(\bm 命令)
\usepackage{wasysym}
\usepackage{geometry} %改尺寸
\usepackage{fontspec} %字体包
\usepackage{setspace}
\usepackage{amsfonts} % 数学公式增强
\usepackage{xeCJK} % 调整字间距
\usepackage{titlesec} 
\usepackage{ragged2e} % 正文两端对齐所需的包。在需要两端对齐的文字前面添加\justifying即可。
\usepackage{indentfirst} % 中文写作通常要求标题后的首段也缩进
% \usepackage{ctex}
\usepackage{pifont} % 带圈数字
\usepackage{etoolbox} % 轻量补丁工具(用于可选参考文献逻辑)

% 操作系统检测(避免 ifplatform 在未开启 shell escape 时产生警告)
\newif\ifwindows
\IfFileExists{/dev/null}{\windowsfalse}{\windowstrue}

%%%————————————————————————————————边距
% \CJKsetecglue命令定义了CJK字符与其他字符之间的空白,不只限于数字字符,也适用于字母等其他非CJK字符。如果你想要更精细的控制,可能需要借助其它工具或者宏包。
% \hskip0.15em plus0.05em minus 0.05em表示基础间距为0.15em,可增加至0.2em,也可减少至0.1em以便满足排版需求。
% 页边距和字距的某个组合应该可以完美复制Word。
\geometry{left=3.00cm,right=2.75cm,top=2.63cm,bottom=2.96cm} % 页边距(对齐 2026 Word 模板 v2 PDF)
\CJKsetecglue{\hskip 0.15em plus 0.05em minus 0.05em} % 数字与CJK字符的间距(对齐 2026 Word 模板)
% left=3.20cm,right=3.13-3.14cm



%%%————————————————————————————————颜色
% 说明:在 XeLaTeX + xcolor 组合下,RGB(0,112,192) 在 PDF 中可能被量化为 (0,111,192);
% 为了让最终 PDF 提取/显示为 Word 的 MsBlue:RGB(0,112,192),这里将 G 分量 +1。
\definecolor{MsBlue}{RGB}{0,113,192} % 目标:PDF 输出等效 RGB(0,112,192)
\definecolor{headercolor}{RGB}{0,0,0} % 页眉颜色
\definecolor{footercolor}{RGB}{0,0,0} % 页脚颜色

%%%————————————————————————————————对象名
\addto\captionsenglish{
    \renewcommand{\contentsname}{目录}
    \renewcommand{\listfigurename}{插图目录}
    \renewcommand{\listtablename}{表格}
    \renewcommand{\refname}{\sihao \templatefont \bfseries \leftline{参考文献}} % 参考文献标题(与 Local/Young 保持一致)
    \renewcommand{\abstractname}{摘要}
    \renewcommand{\indexname}{索引}
    \renewcommand{\tablename}{表}
    \renewcommand{\figurename}{图}
} %把Figure改成‘图’,reference改成‘参考文献’。如此处理是为了避免和babel包冲突。


%%%————————————————————————————————字号
\newcommand{\chuhao}{\fontsize{42pt}{\baselineskip}\selectfont}
\newcommand{\xiaochuhao}{\fontsize{36pt}{\baselineskip}\selectfont}
\newcommand{\yihao}{\fontsize{26pt}{\baselineskip}\selectfont}
\newcommand{\erhao}{\fontsize{22pt}{\baselineskip}\selectfont}
\newcommand{\xiaoerhao}{\fontsize{18pt}{\baselineskip}\selectfont}
\newcommand{\sanhao}{\fontsize{16pt}{22pt}\selectfont}
\newcommand{\sihao}{\fontsize{14pt}{22pt}\selectfont}
\newcommand{\xiaosihao}{\fontsize{12pt}{22pt}\selectfont}
\newcommand{\yemeizihao}{\fontsize{11pt}{\baselineskip}\selectfont}
\newcommand{\wuhao}{\fontsize{10.5pt}{\baselineskip}\selectfont}
\newcommand{\xiaowuhao}{\fontsize{9pt}{\baselineskip}\selectfont}
\newcommand{\liuhao}{\fontsize{7.875pt}{\baselineskip}\selectfont}
\newcommand{\qihao}{\fontsize{5.25pt}{\baselineskip}\selectfont}
%字号对照表
%二号 21pt
%四号 14
%小四 12
%五号 10.5


%%%————————————————————————————————字体
\ifwindows
  % 可参考[LaTeX 中文字体配置基础指南 - 知乎](https://zhuanlan.zhihu.com/p/538459335) 选择自己喜欢的字体。 个人感觉中宋(SimSun)还可以 (~ ̄▽ ̄)~ 
  \setCJKfamilyfont{sectionzhfont}{KaiTi}[AutoFakeBold=3] % 定义模板中文字体为楷体
  \newcommand{\sectionzhfont}{\CJKfamily{sectionzhfont}} % 定义模板中文字体为新命令\sectionzhfont
  \newfontfamily\sectionenfont{KaiTi}[AutoFakeBold] % 定义模板中英文字体为楷体,并定义新命令\sectionzhfont
  \NewDocumentCommand \templatefont { } {\sectionzhfont \sectionenfont} % 模板中英文字体
  \setmainfont{Times New Roman} % 定义主体内容的英文字体为Times New Roman
  \setCJKmainfont{KaiTi}[AutoFakeBold=3]  % 定义主体内容的字体为Kaiti
  \setCJKmonofont{KaiTi}[AutoFakeBold=3]  % 定义 CJK 等宽字体(避免 xeCJK tt 字体警告)
  % \setCJKmainfont{SimSun}[AutoFakeBold=3]  % 定义主体内容的字体为SimSun
  \xeCJKsetup{PunctStyle=quanjiao} % 强制中文标点符号使用中文字体
\else
  % 使用仓库内置字体,避免不同系统的字体名差异影响可复现性
  \setCJKfamilyfont{sectionzhfont}[Path=./fonts/, Extension=.ttf, AutoFakeBold=3]{Kaiti} % 定义模板中文字体为楷体
  \newcommand{\sectionzhfont}{\CJKfamily{sectionzhfont}} % 定义模板中文字体为新命令\sectionzhfont
  \newfontfamily\sectionenfont[Path=./fonts/, Extension=.ttf, AutoFakeBold=3]{Kaiti} % 定义模板中英文字体为楷体,并定义新命令\sectionzhfont
  \NewDocumentCommand \templatefont {} {\sectionzhfont \sectionenfont} % 模板中英文字体
  \setCJKmainfont[Path=./fonts/, Extension=.ttf, AutoFakeBold=3]{Kaiti} % 定义主体内容的字体为Kaiti
  \setCJKmonofont[Path=./fonts/, Extension=.ttf, AutoFakeBold=3]{Kaiti}
  % \setmainfont[Path=./fonts/, Extension=.ttf]{TimesNewRoman} % 定义主体内容的英文字体为Times New Roman。由于是外挂字体,无法自由使用英文斜体和粗体。
  \setmainfont[BoldFont=Times New Roman, AutoFakeBold=5]{Times New Roman} %使用MacOS自带的Times New Roman,可以自由使用英文斜体和粗体。
  % \setCJKmainfont[Path=./fonts/, Extension=.ttf, AutoFakeBold=3]{SimSun} % 定义主体内容的字体为SimSun
  \xeCJKsetup{PunctStyle=quanjiao} % 强制中文标点符号使用中文字体
\fi


%%%————————————————————————————————行距
% Word 2026 模板使用固定行距:22 pt(“固定值 22 磅”)
\renewcommand{\baselinestretch}{1.0}
\AtBeginDocument{\fontsize{12pt}{22pt}\selectfont}
\setlength{\parindent}{0pt} % 保持模板:main.tex 中已有必要的手工缩进控制
\setlength{\parskip}{7.8pt} % Word 模板“段后 7.8pt”
\XeTeXlinebreaklocale "zh" % 中文断行

% Word 空白段落的“视觉高度”对齐:在已有 \parskip 的情况下补足到 1 行行距
\newcommand{\NSFCBlankPara}{\vspace*{\dimexpr\baselineskip-\parskip\relax}}

% 正文段首缩进:仅在 extraTex 正文中启用,避免与 main.tex 的提示语缩进叠加
\newcommand{\NSFCBodyText}{%
  \setlength{\parindent}{2em}%
  % 正文段间距:用户常希望“段间距和行间距一样紧凑”,因此在正文区块内关闭段后距
  % (不影响 main.tex 提示语区域的 Word 风格段后距设置)。
  \setlength{\parskip}{0pt}%
}


%%%————————————————————————————————图片与表格

% 对于表格,推荐使用Excel的插件Excel2LaTeX自动生成LaTeX代码。详见: https://www.zhihu.com/question/307970489/answer/2305355098

%图片
\usepackage{graphicx} %插图宏包
\usepackage{float} %设置图片浮动位置的宏包
\usepackage{subfigure} %插入多图时用子图显示的宏包
\usepackage[section]{placeins} %避免浮动体跨过\section
\usepackage{enumerate} %设置列表环境的包
\usepackage{enumitem} % 增强自定义Bullet(即各种List前的那个标记)
\setlist[enumerate]{
  label={\templatefont \bfseries \hspace{1em} \color{MsBlue}(\arabic*)},
  leftmargin=0em, % 列表的左边界
  itemindent=4em, % item首行缩进
  itemsep=0em, % 列表项之间的垂直间距。
  labelsep=0.1pt,  % label与content的距离
  parsep=0em,  % 两个段落之间的垂直间距。效果上与itemsep类似。
  topsep=0em % 列表与上一段对象的垂直间距。
  % after=\vspace{5pt}
}
% \newcommand{\itemtitlefont}[1]{\textbf{\color{MsBlue} #1}} % 定义item小标题字体;
\newcommand{\itemtitlefont}[1]{{\bfseries \color{MsBlue} #1}} % 定义item小标题字体;
\usepackage{caption}
\captionsetup{font={footnotesize,stretch=1.25},labelsep=period,labelfont=bf, singlelinecheck=off, justification=centering}

%表格
\usepackage{booktabs} %表格
\usepackage{tabularx} %表格宽度
\usepackage{multirow} %多行合并
\usepackage{longtable} % latex 多页显示同一表格
\usepackage{tabu} % 使用longtabu(tabularx + longtable)将长文本在单元格多行显示。
% \captionsetup[table]{name={表},font={footnotesize,stretch=1.25},labelsep=period,labelfont=bf, singlelinecheck=off, justification=centering} 


%%%————————————————————————————————超链接
\usepackage[
	colorlinks,
	urlcolor=MsBlue, % 超链接的颜色。默认是芭比粉
	linkcolor=black,
	anchorcolor=black,
	citecolor=black,
	CJKbookmarks=True
]{hyperref}
\pdfstringdefDisableCommands{%
  \def\linebreak{}%
}

%%%————————————————————————————————参考文献(兼容“xe->bib->xe->xe”)
% 目标:
% 1) 即使正文暂时没有任何 \cite,也能顺利执行 bibtex(不报“no \\bibdata/no \\bibstyle/no \\citation”)。
% 2) 只有在正文实际使用了引用时,才在文末输出“参考文献”,避免空参考文献页影响版式对齐。
\newif\ifNSFCHasCite
\NSFCHasCitefalse
\AtBeginDocument{%
  \pretocmd{\cite}{\global\NSFCHasCitetrue}{}{}%
  \pretocmd{\nocite}{\global\NSFCHasCitetrue}{}{}%
}
\makeatletter
% 参考文献由 main.tex 中的 \clearpage
\bibliographystyle{bibtex-style/gbt7714-nsfc.bst}
\bibliography{references/myexample} 手动控制
% \AtBeginDocument{%
%   \immediate\write\@auxout{\string\bibdata{references/myexample}}%
%   \immediate\write\@auxout{\string\bibstyle{bibtex-style/gbt7714-nsfc}}%
% }
% \AtEndDocument{%
%   \ifNSFCHasCite
%     \bibliographystyle{bibtex-style/gbt7714-nsfc}
%     \bibliography{references/myexample}
%   \else
%     % 若当前文档未使用任何引用,为了让"xe->bib->xe->xe"流程不报错,
%     % 这里写入一个"哑引用"到 aux(bibtex 能正常运行,但不会在 PDF 中输出参考文献)。
%     \immediate\write\@auxout{\string\citation{Smith1900}}%
%   \fi
% }
\makeatother

%%%————————————————————————————————参考文献间距(可调,且不影响其它样式)
% 间距调节(与 Local/Young 完全一致):
% - 标题与上文:\NSFCBibTitleAboveSkip(进入参考文献块前的额外垂直距离)
% - 标题与条目:\NSFCBibTitleBelowSkip(在标题后额外增加/减少的垂直距离)
% - 条目间距:\NSFCBibItemSep(thebibliography 的 \itemsep)
% - 条目行宽:\NSFCBibTextWidth(用于跨项目消除换行差异)
%
% 用法:直接改下面四个 length 的值即可(允许负值微调)。
\newlength{\NSFCBibTitleAboveSkip}
\newlength{\NSFCBibTitleBelowSkip}
\newlength{\NSFCBibItemSep}
\newlength{\NSFCBibTextWidth}
% 默认值:保持当前模板视觉效果(对应旧版 reference.tex 的 \vspace{15pt})
\setlength{\NSFCBibTitleAboveSkip}{15pt}
\setlength{\NSFCBibTitleBelowSkip}{0pt}
\setlength{\NSFCBibItemSep}{0pt}
\setlength{\NSFCBibTextWidth}{397.16727pt} % 推荐默认值:按 Young 的"参考文献条目有效行宽"对齐,用于降低跨项目换行差异;需要更宽/更窄可自行调整
% 说明:上述参数会在本文件末尾的 thebibliography 环境重定义中生效(仅影响参考文献块)。


%%%————————————————————————————————标题计数

% [latex 标题、段落及行距 - 简书](https://www.jianshu.com/p/d7848f815e5f/)
% [LaTeX 中的行间距](https://yinguobing.com/linespace-in-latex/)

\usepackage{titlesec}

\setcounter{secnumdepth}{4} %显示3级目录,然后再具体定制不同的目录
% \setcounter{tocdepth}{4}

% Define section font
\newcommand{\sectionzihao}{\fontsize{14pt}{22pt}\selectfont}
\newcommand{\subsectionzihao}{\fontsize{14pt}{22pt}\selectfont}
\newcommand{\subsubsectionzihao}{\fontsize{13.5pt}{20pt}\selectfont}

% Custom format for section
\titleformat{\section}
  {\color{MsBlue} \sectionzihao \templatefont \bfseries} % format
  {\hspace*{2em}} % label(用于缩进,不影响标题文字内容)
  {0pt} % separation
  {}   % before-code

% Custom format for subsection
\titleformat{\subsection}
  {\color{MsBlue} \subsectionzihao \templatefont} % format
  {} % label
  {0pt} % separation
  {\hspace*{2em}}   % before-code(用于首行缩进,续行自然顶格)

% Custom format for subsubsection
\titleformat{\subsubsection}
  {\color{MsBlue} \subsubsectionzihao \templatefont \bfseries} % format
  {\hspace{1.1em}  \textnormal{\templatefont \arabic{subsection}.\arabic{subsubsection}}} % label
  {0.5em} % separation
  {}   % before-code
\renewcommand\thesubsubsection{\arabic{subsection}.\arabic{subsubsection}} % Reference rule for subsubsection

% Custom format for subsubsubsection
\titleclass{\subsubsubsection}{straight}[\subsubsection]
\newcounter{subsubsubsection}[subsubsection]
\renewcommand\thesubsubsubsection{(\arabic{subsubsubsection})}
\titleformat{\subsubsubsection}
  {\templatefont \bfseries} % format \color{MsBlue} 
  {\hspace{1em} (\arabic{subsubsubsection})} % label
  {0.5pt} % separation
  {}   % before-code
\makeatletter
\def\toclevel@subsubsubsection{4}
\def\l@subsubsubsection{\@dottedtocline{4}{7.0em}{4em}}
\makeatother

% Define line stretch for section
\titlespacing*{\section}
  {0pt} % Left indentation
  {0pt plus 0pt minus 0pt} % Space before the title
  {-7pt} % Space after the title(大幅减小与 subsection 的距离)

% Define line stretch for subsection
\titlespacing*{\subsection}
  {0pt} % Left indentation
  {-8pt plus 0pt minus 0pt} % Space before the title(大幅减小与 section 的距离)
  {0pt} % Space after the title(由 \parskip 控制段后)

% Define line stretch for subsubsection
\titlespacing*{\subsubsection}
  {0pt} % Left indentation
  {0pt plus 0pt minus 0pt} % Space before the title
  {0pt plus 0pt minus 0pt} % Space after the title

% Define line stretch for subsubsection
\titlespacing*{\subsubsubsection}
  {0pt} % Left indentation
  {0pt plus 0pt minus 0pt} % Space before the title
  {0pt plus 0pt minus 0pt} % Space after the title


%%%————————————————————————————————代码
\usepackage{listings} %代码块
\usepackage{xcolor}%代码块
\renewcommand{\lstlistingname}{Code}

% 参考:
% https://www.youtube.com/watch?v=qxkQgG1Y0bY
% https://zhuanlan.zhihu.com/p/65441079

% \lstset{
% 	% language=bash,  %代码语言使用的是matlab
% 	frame=shadowbox, %把代码用带有阴影的框圈起来
% 	rulesepcolor=\color{red!20!green!20!blue!20},%代码块边框为淡青色
% 	keywordstyle=\color{blue!90}\bfseries, %代码关键字的颜色为蓝色,粗体
% 	commentstyle=\color{red!10!green!70}\textit,% 设置代码注释的颜色
% 	showstringspaces=true,%不显示代码字符串中间的空格标记
% 	numbers=left, % 显示行号
% 	numberstyle=\tiny,    % 行号字体
% 	stringstyle=\ttfamily, % 代码字符串的特殊格式
% 	breaklines=false, %对过长的代码自动换行
% 	extendedchars=false,  %解决代码跨页时,章节标题,页眉等汉字不显示的问题
% 	escapebegin=\begin{CJK*},escapeend=\end{CJK*},% 代码中出现中文必须加上,否则报错
% 	texcl=true
% }

\lstdefinestyle{codestyle01}{
	backgroundcolor=\color{gray!12},
	%basicstyle=\ttfamily\small,
	basicstyle=\zihao{-5}\ttfamily,
	commentstyle=\color{green!60!black},
	keywordstyle=\color{magenta},
	stringstyle=\color{blue!50!red},
	showstringspaces=false,
	numbers=left,
	numberstyle=\footnotesize\color{gray},
	numbersep=10pt,
	stepnumber=1,
	tabsize=4,
	frame=single,
	framerule=0.3pt,
	rulecolor=\color{black!20},
	xleftmargin=2em,
	xrightmargin=0.5em,
	breaklines=true,
	breakatwhitespace=true,
	captionpos=b
	% frame=tblr
	% frame=L,
	% framerule=1pt,
	% rulecolor=\color{red}
}
\lstset{style=codestyle01}


%%%————————————————————————————————其它设置


\makeatletter	
	\setlength{\@fptop}{0pt} % 图片置顶展示(而不是居中)
  % \setlength{\textfloatsep}{1cm plus 1.0pt minus 2.0pt} % 浮动体(如figure)之间的垂直间距
  \setlength{\intextsep}{0.5cm plus 1.0pt minus 2.0pt} % 浮动体(如figure)与文字的垂直间距
\makeatother

% 去除页码
\pagestyle{empty}

% 汉字与标点的间距;
% \punctstyle{banjiao} % 所有标点半角
% \punctstyle{kaiming} % 部分的标点半角
\punctstyle{hangmobanjiao} % 末半角式,仅行末挤压。

% 参考文献间距(可调):标题与上文/标题与条目/条目间距(见上方 \NSFCBib* 三个参数)
% 说明:不使用 \section*{\refname}(避免 titlesec 对 \section 的 spacing 差异影响参考文献),
% 仅在 thebibliography 内部手工排标题与列表参数,从而保证三套项目在“同样参考文献参数”下表现一致。
\makeatletter
\renewenvironment{thebibliography}[1]{%
  \par\addvspace{\NSFCBibTitleAboveSkip}%
  \noindent{\refname}\par%
  \vspace{\NSFCBibTitleBelowSkip}%
  \frenchspacing%
  \list{\@biblabel{\@arabic\c@enumiv}}{%
    \settowidth\labelwidth{\@biblabel{#1}}%
    \leftmargin\labelwidth
    \advance\leftmargin\labelsep
    % 统一条目行宽(跨项目像素级一致性依赖此项)
    \rightmargin=\dimexpr\textwidth-\NSFCBibTextWidth-\leftmargin\relax
    \ifdim\rightmargin<0pt\relax \rightmargin=0pt\relax \fi
    \setlength{\itemsep}{\NSFCBibItemSep}%
    \setlength{\parsep}{0pt}%
    \setlength{\parskip}{0pt}%
    \setlength{\topsep}{0pt}%
    \setlength{\partopsep}{0pt}%
    \usecounter{enumiv}%
    \let\p@enumiv\@empty
    \renewcommand\theenumiv{\@arabic\c@enumiv}%
  }%
  \sloppy\clubpenalty4000\widowpenalty4000%
  \sfcode`\.=1000\relax%
}{%
  \def\@noitemerr{\@latex@warning{Empty `thebibliography' environment}}%
  \endlist%
}
\makeatother

% 参考文献中的链接字体用的是texttt
\renewcommand{\ttdefault}{tnr} % 设置默认等宽字体为 Times New Roman

% 自定义第4级小标题
% \newcommand{\ssssubtitle}[1]{\ding{\numexpr171+#1\relax}} % 类似①
\newcommand{\ssssubtitle}[1]{\textcircled{\raisebox{-0.8pt}{\xiaosihao #1}}}
% \newcommand{\ssssubtitle}[1]{#1)} % 类似1)
% \newcommand{\blackding}[1]{\ding{\numexpr181+#1\relax}}
% \newcommand{\whitedingB}[1]{\ding{\numexpr191+#1\relax}}
% \newcommand{\blackdingB}[1]{\ding{\numexpr201+#1\relax}}


%%%%============================正  文=============================%%%%
\begin{document}

{\centering
\vspace{-0.5em}
{\sanhao \templatefont \bfseries 报告正文(2026版)}\par
\vspace{1em}
}

{\sihao \templatefont \hspace*{2em}参照以下提纲撰写,要求内容翔实、清晰,层次分明,标题突出。\linebreak{}申请书正文原则上不超过30页,鼓励简洁表达。{\bfseries \color{MsBlue}请勿删除或改动下述提纲标题及括号中的文字。}}

\vspace{-2pt} % 微调“提纲提示语”与首个 \section 之间的垂直间距
\section{(一)立项依据}

\subsection{(为什么要开展此项研究,研究的科学技术价值如何)}

\justifying
\NSFCBodyText

\subsubsection{研究背景}

\subsubsubsection{案例概述}
\indent 刘亦菲(1987年生)作为中国演艺产业代表性人物,其职业生涯从2002年持续至今,跨越童星、电视剧演员、电影演员、国际巨星等多个阶段。2002年出演《金粉世家》进入演艺圈,2005年凭借《仙剑奇侠传》中"赵灵儿"一角成为现象级偶像。2006-2008年主演《神雕侠侣》《天龙八部》等金庸武侠剧,奠定"古装女神"地位。2012年起向电影转型,主演《铜雀台》《露水红颜》等作品。2020年主演迪士尼真人版《花木兰》,成为首位担纲好莱坞A级制作女主角的华人演员,实现国际化突破。

\subsubsubsection{研究价值}
\indent 刘亦菲的职业发展路径为理解中国娱乐产业变迁、女性艺人国际化策略及跨文化形象管理提供了极具价值的案例。其职业生涯跨越中国影视产业从电视剧主导向电影主导、从国内市场向国际市场转型的关键期(2005-2025)。作为女性艺人,她在"古装女神"到"国际巨星"的转型中面临的挑战具有典型性。《花木兰》项目为研究华人演员好莱坞突破机制提供了珍贵实证。

\subsubsubsection{研究问题}

\subsubsection{国内外研究现状}

\subsubsubsection{职业转型理论}
\indent 目前研究主要集中三个方向:Smith(2020)提出"职业锚理论"\cite{Smith2020};Super(1957)划分职业发展五阶段\cite{Super1957},但缺乏中国演艺产业语境研究。

\subsubsubsection{国际化路径研究}
\indent Johnson(2019)提出"本土-区域-全球"三阶段国际化模型\cite{Johnson2019};Lee(2018)分析亚洲演员好莱坞突破策略\cite{Lee2018},但对华人女演员的文化适应机制研究不足。

\subsubsubsection{品牌形象管理}

\subsubsection{现有研究的局限性}

\subsubsubsection{理论层面}
\indent 现有研究存在以下不足:
\begin{enumerate}[label=\ssssubtitle{\arabic*}, leftmargin=*, itemsep=0.2em]
  \item 缺乏完整职业生命周期的纵向案例研究。
  \item 对华人演员好莱坞突破机制探讨不足。
  \item 忽视女性艺人在跨文化转型中的特殊挑战。
\end{enumerate}

\subsubsubsection{方法层面}

\subsubsection{研究切入点}

\subsubsubsection{案例独特性}
\indent 刘亦菲的职业轨迹具有独特研究价值:(1)转型涵盖"童星→古装女神→电影演员→国际巨星"完整链条;(2)成功突破好莱坞A级制作,成为华人女演员国际化标杆;(3)职业发展与中国影视产业国际化进程高度同步。

\subsubsubsection{研究代表性}


% 参考文献
\clearpage
\bibliographystyle{bibtex-style/gbt7714-nsfc.bst}
\bibliography{references/myexample}

\section{(二)研究内容}

\subsection{\texorpdfstring{\hspace{0.1em}}{}(提纲不做限制,请按照研究工作的自身逻辑撰写。应提炼出特\linebreak{}色与创新点、年度研究计划)}


\justifying

\subsubsection{研究内容}

\indent

本研究以佐佐木希2002-2025年职业发展为案例,开展以下研究:(1)\textbf{职业发展轨迹纵向研究}:梳理职业路径,划分为五阶段(2002-2008时尚模特期、2008-2010转型期、2010-2017多领域拓展期、2017-2020成熟期、2020-2025重塑期),识别转型触发点;(2)\textbf{跨媒介形象构建机制}:分析不同媒介平台形象呈现策略,探讨"娃娃形象"的建构演变;(3)\textbf{品牌韧性机制}:以2020年丈夫出轨风波为关键事件,分析个人品牌面临重大负面冲击时的应对策略;(4)\textbf{新媒体转型策略}:研究从传统媒体向YouTube等新媒体平台的转型路径。

\subsubsection{研究目标}

\indent

本项目总体目标是构建基于东亚文化语境的艺人职业发展动态分析框架。具体目标:(1)\textbf{理论目标}:提出"艺人职业转型三阶段模型"(探索期-稳定期-重塑期),建立"跨媒介形象一致性测量指标体系";(2)\textbf{方法学目标}:开发"艺人职业轨迹编码手册",构建"舆情事件影响评估模型";(3)\textbf{应用目标}:为娱乐产业从业者提供"艺人职业规划决策支持工具"。

\subsubsection{拟解决的关键科学问题}

\indent

(1)\textbf{艺人职业转型的临界点识别问题}:如何科学识别关键转型节点?哪些内外部因素起决定性作用?(2)\textbf{跨媒介形象的一致性维持问题}:如何在保持差异化形象的同时确保一致性?(3)\textbf{个人品牌韧性的边界条件问题}:何种负面事件可通过特定策略修复?

\subsubsection{研究方法}

\indent

本研究采用混合研究方法:(1)\textbf{纵向案例研究法}:收集完整职业档案(媒体报道约5000篇、影视作品约30部、商业代言约50个、社交媒体数据等),采用时序分析识别关键转折点;(2)\textbf{内容分析法}:对时尚杂志约200期进行编码分析,提炼形象演变规律;对影视作品角色类型进行系统分类。编码工作由两名研究者独立完成,采用Cohen's Kappa系数评估信度(K>0.8);(3)\textbf{舆情分析法}:收集2020年出轨风波期间社交媒体舆情数据约10万条,运用情感分析和主题模型评估公众情绪变化。采用LDA模型进行主题分析,使用BERT预训练模型进行情感分类(准确率目标85\%以上);(4)\textbf{比较研究法}:选取同时期其他日本女性艺人(滨崎步、新垣结衣、石原里美等)作为对照案例,对比中日韩三国艺人发展模式异同。

\subsubsection{技术路线}

\indent

本研究技术路线分四阶段(图\ref{techroute}):(1)\textbf{数据收集与预处理}:建立多源异构数据库;(2)\textbf{职业轨迹建模}:应用序列分析识别职业发展典型路径,构建"职业状态转移矩阵";(3)\textbf{形象分析}:采用深度学习方法对影像资料进行特征提取,运用NLP技术分析文本资料;(4)\textbf{模型验证与应用}:通过交叉验证和敏感性分析评估模型稳健性。

\begin{figure}[!th]
\begin{center}
\includegraphics[width=0.8\linewidth]{figures/zzmx-115.jpg}
\caption{佐佐木希职业发展的多阶段转型模型示意图。}
\label{techroute}
\end{center}
\end{figure}

\subsubsection{关键技术}

\indent

(1)\textbf{多模态数据融合}:整合文本、图像、视频等多模态数据,构建统一知识图谱;(2)\textbf{职业轨迹序列挖掘}:应用频繁序列模式挖掘算法识别职业发展典型模式;(3)\textbf{舆情情感分析}:基于BERT、GPT等预训练模型进行日语文本情感分析(准确率85\%以上)。

\subsubsection{可行性分析}

\indent

(1)\textbf{理论可行}:建立在成熟的职业发展理论、品牌管理理论、传播学理论基础之上;(2)\textbf{数据可行}:佐佐木希公开资料丰富,Wikipedia、MyDramalist、AsianWiki等数据库提供充足数据源;(3)\textbf{技术可行}:研究团队具备NLP、计算机视觉、数据挖掘技术储备。


\begingroup
\justifying
\setlength{\parindent}{2em}% 首行缩进 2 字符

\subsubsection{学术思想的创新}

\textbf{【示例内容】}

本项目的核心创新在于提出"艺人职业发展三阶段动态模型"(Exploration-Stabilization-Reinvention),突破了传统职业发展理论对线性路径的假设。与Super(1957)的职业发展阶段理论相比,本项目强调艺人职业发展的非线性特征和多阶段循环性\cite{Super1957}。

具体而言,本项目首次将"品牌韧性"(Brand Resilience)概念引入艺人职业发展研究,提出"形象修复三机制"理论框架:(1)情感机制-公众同情心的调动;(2)认知机制-品牌联想的重构;(3)行为机制-职业行动的调整。这一框架填补了现有研究对负面事件后品牌修复机制的理论空白。\NSFCBlankPara

\subsubsection{研究方法的创新}

\textbf{【示例内容】}

本研究在方法学上的创新体现在以下三个方面:

\begin{enumerate}
  \item \itemtitlefont{多模态数据融合分析}:\par
  (1)首次将文本分析、图像分析、视频分析整合于统一的艺人职业发展研究框架中。\par
  (2)开发基于注意力机制的多模态特征融合算法,实现不同数据源信息的自动加权与整合。\NSFCBlankPara

  \item \itemtitlefont{纵向序列分析方法的跨学科应用}:\par
  (1)将社会科学中的生命历程研究(Life Course Research)方法与数据科学中的序列挖掘(Sequence Mining)技术相结合。\par
  (2)提出"职业状态转移矩阵"(Career State Transition Matrix)概念,量化职业转型的概率与方向。\NSFCBlankPara

  \item \itemtitlefont{混合研究方法的设计创新}:\par
  (1)采用"定量引导-定性深化"(Quantitative-to-Qualitative)的混合设计,先用大数据方法识别模式,再用深度案例访谈理解机制。\par
  (2)开发"艺人职业轨迹编码手册"(Career Trajectory Coding Manual),实现研究过程的可重复性与可验证性。\NSFCBlankPara
\end{enumerate}

\subsubsection{研究视角的创新}

\textbf{【示例内容】}

本项目的独特之处在于其跨学科的研究视角,整合传播学、管理学、社会学、计算机科学等多个学科的理论与方法:

\begin{enumerate}
  \item \itemtitlefont{性别视角的引入}:\par
  (1)关注女性艺人在职业发展中的特殊挑战,如年龄歧视、婚姻对职业的影响、"母亲角色"与"艺人身份"的冲突等。\par
  (2)分析佐佐木希在结婚生子后如何调整职业策略,为女性艺人的职业可持续发展提供理论指导。\NSFCBlankPara

  \item \itemtitlefont{文化比较视角}:\par
  (1)对比东亚文化语境(日本、韩国、中国)与西方文化语境下艺人职业发展模式的差异。\par
  (2)探讨集体主义文化、高语境文化对艺人-粉丝关系、公众期待、形象管理策略的影响。\NSFCBlankPara

  \item \itemtitlefont{产业生态视角}:\par
  (1)将艺人个体置于娱乐产业生态系统中考察,分析其与经纪公司、媒体平台、粉丝群体、广告商等多主体的互动关系。\par
  (2)研究传统媒体(电视、杂志)与新媒体(YouTube、Instagram)的生态位竞争对艺人职业策略的影响。\NSFCBlankPara
\end{enumerate}

\subsubsection{应用价值的创新}

\textbf{【示例内容】}

本项目的研究成果将具有重要的实践价值:

\begin{enumerate}
  \item \itemtitlefont{对娱乐产业从业者的指导价值}:\par
  (1)为经纪公司提供"艺人职业规划决策支持工具",帮助识别最佳转型时机与方向。\par
  (2)为艺人本人提供"个人品牌管理指南",提高职业发展的可持续性。\NSFCBlankPara

  \item \itemtitlefont{对文化产业政策的参考价值}:\par
  (1)为政府文化部门制定艺人保护政策、完善娱乐产业规范提供实证依据。\par
  (2)为文化产业的可持续发展提供理论指导。\NSFCBlankPara

  \item \itemtitlefont{对职业发展理论的拓展价值}:\par
  (1)本项目的理论框架可迁移至其他高风险、高不确定性职业(如体育、创业等)的职业发展研究。\par
  (2)为个人品牌理论在数字化时代的发展提供新的理论视角。\NSFCBlankPara
\end{enumerate}

\endgroup



\justifying

\subsubsection{第一年研究计划}

\indent\setlength{\parindent}{2em}

\textbf{研究重点:数据收集与预处理、职业轨迹初步建模}

\begin{enumerate}
    \item \itemtitlefont{数据收集阶段(第1-6个月)}:
    \ssssubtitle{1}建立佐佐木希职业发展档案数据库,收集2002-2025年的媒体报道、影视作品目录、商业代言记录、杂志封面数据等。
    \ssssubtitle{2}收集2020年出轨风波期间的社交媒体舆情数据(Twitter、Instagram、5ch等),建立时间序列数据集。
    \ssssubtitle{3}开发数据清洗与预处理流程,实现多源异构数据的标准化存储。

    \item \itemtitlefont{文献综述与理论构建(第7-12个月)}:
    \ssssubtitle{1}完成国内外相关研究的系统文献综述,撰写文献综述报告。
    \ssssubtitle{2}构建"艺人职业发展三阶段动态模型"的理论框架,撰写理论模型论文。
    \ssssubtitle{3}开发"艺人职业轨迹编码手册"(Career Trajectory Coding Manual)初稿。
\end{enumerate}

\textbf{年度预期成果}:建立完整的研究数据库;完成文献综述论文1-2篇;构建理论模型框架。

\subsubsection{第二年研究计划}

\indent\setlength{\parindent}{2em}

\textbf{研究重点:深度数据分析、模型构建与验证}

\begin{enumerate}
    \item \itemtitlefont{职业轨迹建模阶段(第13-18个月)}:
    \ssssubtitle{1}应用序列分析方法(Sequence Analysis)识别佐佐木希职业发展的关键转型节点与典型路径。
    \ssssubtitle{2}构建"职业状态转移矩阵"(Career State Transition Matrix),量化不同职业状态之间的转移概率。
    \ssssubtitle{3}开发"职业转型预测模型"原型,实现转型时机与方向的初步预测。

    \item \itemtitlefont{形象分析阶段(第19-24个月)}:
    \ssssubtitle{1}采用深度学习方法(CNN、Vision Transformer)对佐佐木希的影像资料进行特征提取,量化形象呈现的演变规律。
    \ssssubtitle{2}运用自然语言处理技术(BERT、GPT)分析文本资料中的形象建构策略,识别关键词汇与主题演变。
    \ssssubtitle{3}构建"跨媒介形象一致性测量指标体系",实现形象一致性的量化评估。
\end{enumerate}

\textbf{年度预期成果}:完成职业轨迹建模分析;发表实证研究论文2-3篇;开发模型原型系统。

\subsubsection{第三年研究计划}

\indent\setlength{\parindent}{2em}

\textbf{研究重点:模型验证、案例比较、成果总结与转化}

\begin{enumerate}
    \item \itemtitlefont{模型验证与优化阶段(第25-30个月)}:
    \ssssubtitle{1}通过交叉验证(Cross-Validation)和敏感性分析(Sensitivity Analysis)评估"职业转型预测模型"的稳健性。
    \ssssubtitle{2}选取同时期其他日本女性艺人(如坛蜜、木下优树菜、水原希子等)作为对照案例,检验模型的外部效度。
    \ssssubtitle{3}根据验证结果优化模型参数,提高预测准确性。

    \item \itemtitlefont{成果总结与转化阶段(第31-36个月)}:
    \ssssubtitle{1}撰写研究总报告,系统总结研究发现与理论贡献。
    \ssssubtitle{2}开发"艺人职业规划决策支持工具"原型,进行用户测试与优化。
    \ssssubtitle{3}组织学术研讨会,向娱乐产业从业者、政策制定者、学术界传播研究成果。
    \ssssubtitle{4}撰写专著《艺人职业发展的动态机制:基于佐佐木希案例的纵向研究》。
\end{enumerate}

\textbf{年度预期成果}:完成模型验证与优化;发表高水平研究论文3-4篇;完成研究总报告;开发决策支持工具原型。

\subsubsection{预期研究结果}

\indent\setlength{\parindent}{2em}

\textbf{理论贡献}:

\begin{enumerate}
    \item \itemtitlefont{理论创新}:
    \ssssubtitle{1}提出"艺人职业发展三阶段动态模型"(Exploration-Stabilization-Reinvention),为职业发展理论提供新的理论视角。
    \ssssubtitle{2}建立"品牌韧性"(Brand Resilience)理论框架,揭示负面事件后品牌修复的内在机制。
    \ssssubtitle{3}构建"跨媒介形象一致性测量指标体系",实现形象管理的量化评估。

    \item \itemtitlefont{方法学贡献}:
    \ssssubtitle{1}开发"艺人职业轨迹编码手册",为相关研究提供标准化的研究工具。
    \ssssubtitle{2}提出多模态数据融合分析框架,推动文化研究方法的数字化革新。
    \ssssubtitle{3}建立"舆情事件影响评估模型",为品牌危机管理提供量化评估工具。
\end{enumerate}

\textbf{实践价值}:

\begin{enumerate}
    \item \itemtitlefont{对娱乐产业}:
    \ssssubtitle{1}为经纪公司提供"艺人职业规划决策支持工具",提高艺人管理的科学性。
    \ssssubtitle{2}为艺人本人提供"个人品牌管理指南",增强职业发展的可持续性。

    \item \itemtitlefont{对文化产业政策}:
    \ssssubtitle{1}为政府文化部门制定艺人保护政策、完善娱乐产业规范提供实证依据。
    \ssssubtitle{2}为文化产业的可持续发展提供理论指导。
\end{enumerate}

\textbf{学术产出}:

\begin{enumerate}
    \item \itemtitlefont{论文发表}:
    \ssssubtitle{1}计划发表高水平学术论文6-8篇,其中SSCI/CSSCI检索论文不少于4篇。
    \ssssubtitle{2}目标期刊包括《Journal of Business Research》、《Tourism Management》、《新闻与传播研究》、《现代传播》等。

    \item \itemtitlefont{学术专著}:
    \ssssubtitle{1}撰写专著1部,系统呈现研究成果。

    \item \itemtitlefont{学术交流}:
    \ssssubtitle{1}参加国际学术会议2-3次,包括ICA、IAMCR等顶级会议。
    \ssssubtitle{2}组织专题学术研讨会1-2次,推动学术交流与合作。
\end{enumerate}

\clearpage


\section{(三)研究基础}

{\subsection{\punctstyle{banjiao}1. \textbf{研究基础与可行性分析}(与本项目相关的研究工作积累和已\linebreak{}取得的研究工作成绩,研究风险的应对措施等);}}


\justifying

\subsubsection{研究基础与可行性分析}

\indent\setlength{\parindent}{2em}%首行缩进4字符

happy happy happy.

快乐 快乐 快乐。

\subsubsection{工作条件}

\indent\setlength{\parindent}{2em}%首行缩进4字符

happy happy happy.

快乐 快乐 快乐。


\subsection{{\punctstyle{banjiao}2. \textbf{工作条件}(包括已具备的实验条件,尚缺少的实验条件和拟\linebreak{}解决的途径,包括利用国家实验室、全国重点实验室和部门重点实验\linebreak{}室等研究基地的计划与落实情况);}}


\justifying

\subsubsection{已具备的实验条件}

\indent\setlength{\parindent}{2em}

\textbf{硬件设施}:

依托单位文化产业研究中心配备了完善的科研设施,能够满足本项目的需求:(1)高性能计算集群,配备NVIDIA A100 GPU 4张,总内存512GB,可支持深度学习模型的训练与推理;(2)大容量数据存储系统,总容量500TB,支持多模态研究数据的长期存储与管理;(3)专业级影像处理工作站,配备高精度显示器,用于影像资料的标注与分析;(4)专业录音与视频编辑设备,用于多媒体资料的采集与处理。

\textbf{软件资源}:

实验室已部署完整的数据分析工具链:(1)自然语言处理工具包(NLTK、spaCy、jieba等),支持多语言文本分析;(2)深度学习框架(TensorFlow、PyTorch、Keras),用于构建和训练神经网络模型;(3)计算机视觉工具(OpenCV、PIL、scikit-image),用于影像资料的特征提取;(4)数据可视化工具(Tableau、D3.js、Matplotlib),用于研究结果的呈现;(5)社会科学统计软件(SPSS、Stata、R),用于传统统计分析。

\textbf{数据库资源}:

依托单位购买了以下学术数据库的使用权限:(1)外文数据库:Web of Science、Scopus、JSTOR、SpringerLink、ScienceDirect、EBSCO等;(2)中文数据库:中国知网(CNKI)、万方数据、维普资讯、人大复印报刊资料等;(3)专业数据库:MyDramalist(影视作品数据库)、AsianWiki(亚洲艺人数据库)等。这些数据库为本项目的文献调研与数据收集提供了坚实基础。

\subsubsection{尚缺少的实验条件及拟解决途径}

\indent\setlength{\parindent}{2em}

\textbf{缺少的实验条件}:

\begin{enumerate}
    \item \itemtitlefont{日本本土数据资源访问受限}:
    \ssssubtitle{1}部分日本本土的社交媒体数据(如Twitter日本区历史数据、5ch论坛历史帖子)因地域限制和平台政策,无法完整获取。
    \ssssubtitle{2}日本娱乐产业的一手资料(如经纪公司内部数据、艺人合约模板等)属于商业机密,难以直接获取。

    \item \itemtitlefont{跨文化比较研究资源不足}:
    \ssssubtitle{1}韩国娱乐产业的相关数据(如K-pop艺人职业发展数据)需要与韩国研究机构合作获取。
    \ssssubtitle{2}欧洲与北美娱乐产业的数据资源相对分散,缺乏系统性的获取渠道。
\end{enumerate}

\textbf{拟解决途径}:

\begin{enumerate}
    \item \itemtitlefont{建立国际合作网络}:
    \ssssubtitle{1}与日本东京大学、早稻田大学的相关研究机构建立合作关系,通过学术交流获取日本本土数据资源。
    \ssssubtitle{2}与韩国首尔国立大学、汉阳大学的媒体研究中心建立合作关系,开展中、日、韩三国娱乐产业比较研究。
    \ssssubtitle{3}与欧洲的阿姆斯特丹大学、北美的南加州大学的相关研究者建立联系,拓展跨文化比较研究网络。

    \item \itemtitlefont{利用合法合规的数据服务提供商}:
    \ssssubtitle{1}通过与合法的数据服务提供商(如Brandwatch、Talkwalker等)合作,获取社交媒体历史数据。
    \ssssubtitle{2}利用公开数据集(如Twitter Academic Research Product Track)获取研究所需的基础数据。

    \item \itemtitlefont{开发数据采集工具}:
    \ssssubtitle{1}自主研发网络爬虫工具,在遵守平台服务条款的前提下,公开采集可用数据。
    \ssssubtitle{2}开发多语言情感分析工具,提高对日语、韩语等小语种数据的处理能力。
\end{enumerate}

\subsubsection{利用国家重点实验室等研究基地的计划}

\indent\setlength{\parindent}{2em}

\textbf{计划利用的国家重点实验室}:

\begin{enumerate}
    \item \itemtitlefont{中国人民大学新闻学院社会发展研究基地}:
    \ssssubtitle{1}该基地在传播学研究领域具有深厚积累,可为本项目提供理论指导与方法支持。
    \ssssubtitle{2}计划于第一年派遣研究团队成员赴该基地进行为期3个月的访学交流,学习先进的研究方法。

    \item \itemtitlefont{清华大学新闻与传播学院新媒体研究中心}:
    \ssssubtitle{1}该中心在新媒体传播研究方面处于国内领先地位,可为本项目提供计算传播学方法支持。
    \ssssubtitle{2}计划于第二年与该中心联合举办"文化产业数字化转型"学术研讨会。

    \item \itemtitlefont{中国传媒大学文化产业管理学院}:
    \ssssubtitle{1}该学院在文化产业管理研究方面具有丰富经验,可为本项目提供政策解读与实践指导。
    \ssssubtitle{2}计划于第三年与该学院合作开展"艺人职业发展"专题研究。
\end{enumerate}

\textbf{计划利用的部门重点实验室}:

\begin{enumerate}
    \item \itemtitlefont{文化和旅游部文化产业研究中心}:
    \ssssubtitle{1}该中心掌握文化产业的宏观政策数据,可为项目提供政策环境分析支持。
    \ssssubtitle{2}计划邀请中心专家参与项目咨询,确保研究成果符合政策导向。

    \item \itemtitlefont{国家广播电视总局广播电视规划院}:
    \ssssubtitle{1}该院拥有影视产业的一手统计数据,可为项目提供权威数据支持。
    \ssssubtitle{2}计划于项目执行期与该院建立数据共享机制。
\end{enumerate}

\textbf{落实情况}:

申请团队已与上述研究基地的负责人进行了初步沟通,均表示愿意支持本项目的开展。具体合作计划将在项目获批后进一步细化落实。

\clearpage


\subsection{{\punctstyle{banjiao}3. \textbf{正在承担的与本项目相关的科研项目情况}(申请人正在承担\linebreak{}的与本项目相关的科研项目情况,包括国家自然科学基金的项目和国\linebreak{}家其他科技计划项目,要注明项目的资助机构、项目类别、批准号、\linebreak{}项目名称、获资助金额、起止年月、与本项目的关系及负责的内容等);}}

\begingroup
\justifying
\setlength{\parindent}{2em}% 首行缩进 2 字符

\textbf{【示例内容】}

申请人正在承担以下科研项目:

(1)国家社科基金一般项目:项目名称XXX研究,项目编号XXX,起止时间2020年9月—2024年12月,项目金额20万元,本人角色主持人,与本项目关系为本项目的研究成果将为面上项目提供理论基础和方法支撑。

(2)教育部人文社科研究项目:项目名称XXX研究,项目编号XXX,起止时间2019年7月—2023年12月,项目金额8万元,本人角色主持人,与本项目关系为本项目积累了XXX研究经验,为面上项目提供研究基础。

(3)省部级项目:项目名称XXX研究,项目编号XXX,起止时间2018年1月—2022年12月,项目金额5万元,本人角色主持人,与本项目关系为本项目对XXX的研究成果,将直接支撑面上项目。

\textbf{说明:}以上项目均按计划顺利实施,与面上项目在研究方向上有机衔接,但研究内容不重复,不存在时间冲突。

\endgroup


\subsection{{\punctstyle{banjiao}4. \textbf{完成国家自然科学基金项目情况}(对申请人负责的前一个已\linebreak{}资助期满的科学基金项目(项目名称及批准号)完成情况、后续研究\linebreak{}进展及与本申请项目的关系加以详细说明。另附该项目的研究工作总\linebreak{}结摘要(限500字)和相关成果详细目录)。}}

\begingroup
\justifying
\setlength{\parindent}{2em}% 首行缩进 2 字符

\textbf{【示例内容】}

\subsubsection{完成国家自然科学基金项目情况}

申请人曾主持并完成以下国家自然科学基金项目:

\begin{enumerate}
  \item \itemtitlefont{青年科学基金项目}:\par
  项目名称:XXX研究\par
  项目编号:XXX\par
  起止时间:2016年1月—2018年12月\par
  项目金额:18万元\par
  本人角色:主持人\par
  完成情况:项目按期完成,发表SCI/SSCI论文3篇,CSSCI论文5篇,出版专著1部,获得良好的学术评价。\NSFCBlankPara
\end{enumerate}

\textbf{说明:}以上项目已按计划顺利完成,研究成果丰富,为本面上项目的申请奠定了坚实的学术基础。

\endgroup


\section{(四)其他需要说明的情况}

\subsection{1. 申请人同年申请不同类型的国家自然科学基金项目情况(列\linebreak{}明同年申请的其他项目的项目类型、项目名称信息,并说明与本项目\linebreak{}之间的区别与联系;已收到自然科学基金委不予受理或不予资助决定\linebreak{}的,无需列出)。}

\input{extraTex/4.1.不同类型国基情况.tex}

\subsection{2. 具有高级专业技术职务(职称)的申请人是否存在同年申请\linebreak{}或者参与申请国家自然科学基金项目的单位不一致的情况;如存在上\linebreak{}述情况,列明所涉及人员的姓名,申请或参与申请的其他项目的项目\linebreak{}类型、项目名称、单位名称、上述人员在该项目中是申请人还是参与\linebreak{}者,并说明单位不一致原因。}

\justifying
\NSFCBodyText

% 在此处撰写“2. 同年单位不一致”的说明内容(写正文后可删除下行占位空白)


\subsection{3. 具有高级专业技术职务(职称)的申请人是否存在与正在承\linebreak{}担的国家自然科学基金项目的单位不一致的情况;如存在上述情况,\linebreak{}列明所涉及人员的姓名,正在承担项目的批准号、项目类型、项目名\linebreak{}称、单位名称、起止年月,并说明单位不一致原因。}

% 在此处撰写“3. 承担中单位不一致”的说明内容(写正文后可删除下行占位空白)
\NSFCBlankPara


\subsection{4. 同年以不同专业技术职务(职称)申请或参与申请科学基金\linebreak{}项目的情况(应详细说明原因)。}

{\fontsize{14pt}{14pt}\selectfont \templatefont \color{MsBlue}
\setlength{\parindent}{2em}
\setlength{\parskip}{0pt}
\indent 4. 申请人和主要参与者同年以不同专业技术职务(职称)申请或参与申请科学基金项目的情况(应详细说明原因)。
\par
}
\vspace{0.3\baselineskip}


\subsection{5. 其他。}

{\fontsize{14pt}{14pt}\selectfont \templatefont \color{MsBlue}
\setlength{\parindent}{2em}
\setlength{\parskip}{0pt}
\indent 6. 其他(包括但不限于使用以他人名义申报过的申请书;如有,请详细说明)。
\par
}
\vspace{0pt}


\clearpage
\end{document}
