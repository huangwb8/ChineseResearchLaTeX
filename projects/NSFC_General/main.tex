%!TEX program = xelatex

% 国家自然科学基金NSFC-面上项目-正文(2026年版)

% 测试环境:[使用VSCode编写LaTeX - 知乎](https://zhuanlan.zhihu.com/p/38178015)

% 编译顺序: xelatex -> bibtex -> xelatex -> xelatex

% 参考 & 鸣谢:
% [Ruzim/NSFC-application-template-latex: 国家自然科学基金申请书正文(面上项目)LaTeX 模板(非官方)](https://github.com/Ruzim/NSFC-application-template-latex)
% [Readon/NSFC-application-template-latex: 国家自然科学基金申请书正文(面上项目)模板(非官方)](https://github.com/Readon/NSFC-application-template-latex)
% [Aligning equations with amsmath - Overleaf, Online LaTeX Editor](https://www.overleaf.com/learn/latex/Aligning_equations_with_amsmath):各种格式的公式写法


\documentclass[12pt,UTF8,AutoFakeBold=2.5,a4paper]{ctexart} %默认小四号字。允许楷体粗体。

%%%%============================系统设置===========================%%%%

\usepackage[english]{babel} %支持混合语言
\usepackage{xcolor}
\usepackage{graphicx} 
\usepackage{amsmath} %更多数学符号
\usepackage{bm} % 粗体数学符号(\bm 命令)
\usepackage{wasysym}
\usepackage{geometry} %改尺寸
\usepackage{fontspec} %字体包
\usepackage{setspace}
\usepackage{amsfonts} % 数学公式增强
\usepackage{xeCJK} % 调整字间距
\usepackage{titlesec} 
\usepackage{ragged2e} % 正文两端对齐所需的包。在需要两端对齐的文字前面添加\justifying即可。
\usepackage{indentfirst} % 中文写作通常要求标题后的首段也缩进
% \usepackage{ctex}
\usepackage{pifont} % 带圈数字
\usepackage{etoolbox} % 轻量补丁工具(用于可选参考文献逻辑)

% 操作系统检测(避免 ifplatform 在未开启 shell escape 时产生警告)
\newif\ifwindows
\IfFileExists{/dev/null}{\windowsfalse}{\windowstrue}

%%%————————————————————————————————边距
% \CJKsetecglue命令定义了CJK字符与其他字符之间的空白,不只限于数字字符,也适用于字母等其他非CJK字符。如果你想要更精细的控制,可能需要借助其它工具或者宏包。
% \hskip0.15em plus0.05em minus 0.05em表示基础间距为0.15em,可增加至0.2em,也可减少至0.1em以便满足排版需求。
% 页边距和字距的某个组合应该可以完美复制Word。
\geometry{left=3.00cm,right=2.75cm,top=2.63cm,bottom=2.96cm} % 页边距(对齐 2026 Word 模板 v2 PDF)
\CJKsetecglue{\hskip 0.15em plus 0.05em minus 0.05em} % 数字与CJK字符的间距(对齐 2026 Word 模板)
% left=3.20cm,right=3.13-3.14cm



%%%————————————————————————————————颜色
% 说明:在 XeLaTeX + xcolor 组合下,RGB(0,112,192) 在 PDF 中可能被量化为 (0,111,192);
% 为了让最终 PDF 提取/显示为 Word 的 MsBlue:RGB(0,112,192),这里将 G 分量 +1。
\definecolor{MsBlue}{RGB}{0,113,192} % 目标:PDF 输出等效 RGB(0,112,192)
\definecolor{headercolor}{RGB}{0,0,0} % 页眉颜色
\definecolor{footercolor}{RGB}{0,0,0} % 页脚颜色

%%%————————————————————————————————对象名
\addto\captionsenglish{
    \renewcommand{\contentsname}{目录}
    \renewcommand{\listfigurename}{插图目录}
    \renewcommand{\listtablename}{表格}
    \renewcommand{\refname}{\sihao \templatefont \bfseries \leftline{参考文献}} % 参考文献标题(与 Local/Young 保持一致)
    \renewcommand{\abstractname}{摘要}
    \renewcommand{\indexname}{索引}
    \renewcommand{\tablename}{表}
    \renewcommand{\figurename}{图}
} %把Figure改成‘图’,reference改成‘参考文献’。如此处理是为了避免和babel包冲突。


%%%————————————————————————————————字号
\newcommand{\chuhao}{\fontsize{42pt}{\baselineskip}\selectfont}
\newcommand{\xiaochuhao}{\fontsize{36pt}{\baselineskip}\selectfont}
\newcommand{\yihao}{\fontsize{26pt}{\baselineskip}\selectfont}
\newcommand{\erhao}{\fontsize{22pt}{\baselineskip}\selectfont}
\newcommand{\xiaoerhao}{\fontsize{18pt}{\baselineskip}\selectfont}
\newcommand{\sanhao}{\fontsize{16pt}{22pt}\selectfont}
\newcommand{\sihao}{\fontsize{14pt}{22pt}\selectfont}
\newcommand{\xiaosihao}{\fontsize{12pt}{22pt}\selectfont}
\newcommand{\yemeizihao}{\fontsize{11pt}{\baselineskip}\selectfont}
\newcommand{\wuhao}{\fontsize{10.5pt}{\baselineskip}\selectfont}
\newcommand{\xiaowuhao}{\fontsize{9pt}{\baselineskip}\selectfont}
\newcommand{\liuhao}{\fontsize{7.875pt}{\baselineskip}\selectfont}
\newcommand{\qihao}{\fontsize{5.25pt}{\baselineskip}\selectfont}
%字号对照表
%二号 21pt
%四号 14
%小四 12
%五号 10.5


%%%————————————————————————————————字体
\ifwindows
  % 可参考[LaTeX 中文字体配置基础指南 - 知乎](https://zhuanlan.zhihu.com/p/538459335) 选择自己喜欢的字体。 个人感觉中宋(SimSun)还可以 (~ ̄▽ ̄)~ 
  \setCJKfamilyfont{sectionzhfont}{KaiTi}[AutoFakeBold=3] % 定义模板中文字体为楷体
  \newcommand{\sectionzhfont}{\CJKfamily{sectionzhfont}} % 定义模板中文字体为新命令\sectionzhfont
  \newfontfamily\sectionenfont{KaiTi}[AutoFakeBold] % 定义模板中英文字体为楷体,并定义新命令\sectionzhfont
  \NewDocumentCommand \templatefont { } {\sectionzhfont \sectionenfont} % 模板中英文字体
  \setmainfont{Times New Roman} % 定义主体内容的英文字体为Times New Roman
  \setCJKmainfont{KaiTi}[AutoFakeBold=3]  % 定义主体内容的字体为Kaiti
  \setCJKmonofont{KaiTi}[AutoFakeBold=3]  % 定义 CJK 等宽字体(避免 xeCJK tt 字体警告)
  % \setCJKmainfont{SimSun}[AutoFakeBold=3]  % 定义主体内容的字体为SimSun
  \xeCJKsetup{PunctStyle=quanjiao} % 强制中文标点符号使用中文字体
\else
  % 使用仓库内置字体,避免不同系统的字体名差异影响可复现性
  \setCJKfamilyfont{sectionzhfont}[Path=./fonts/, Extension=.ttf, AutoFakeBold=3]{Kaiti} % 定义模板中文字体为楷体
  \newcommand{\sectionzhfont}{\CJKfamily{sectionzhfont}} % 定义模板中文字体为新命令\sectionzhfont
  \newfontfamily\sectionenfont[Path=./fonts/, Extension=.ttf, AutoFakeBold=3]{Kaiti} % 定义模板中英文字体为楷体,并定义新命令\sectionzhfont
  \NewDocumentCommand \templatefont {} {\sectionzhfont \sectionenfont} % 模板中英文字体
  \setCJKmainfont[Path=./fonts/, Extension=.ttf, AutoFakeBold=3]{Kaiti} % 定义主体内容的字体为Kaiti
  \setCJKmonofont[Path=./fonts/, Extension=.ttf, AutoFakeBold=3]{Kaiti}
  % \setmainfont[Path=./fonts/, Extension=.ttf]{TimesNewRoman} % 定义主体内容的英文字体为Times New Roman。由于是外挂字体,无法自由使用英文斜体和粗体。
  \setmainfont[BoldFont=Times New Roman, AutoFakeBold=5]{Times New Roman} %使用MacOS自带的Times New Roman,可以自由使用英文斜体和粗体。
  % \setCJKmainfont[Path=./fonts/, Extension=.ttf, AutoFakeBold=3]{SimSun} % 定义主体内容的字体为SimSun
  \xeCJKsetup{PunctStyle=quanjiao} % 强制中文标点符号使用中文字体
\fi


%%%————————————————————————————————行距
% Word 2026 模板使用固定行距:22 pt(“固定值 22 磅”)
\renewcommand{\baselinestretch}{1.0}
\AtBeginDocument{\fontsize{12pt}{22pt}\selectfont}
\setlength{\parindent}{0pt} % 保持模板:main.tex 中已有必要的手工缩进控制
\setlength{\parskip}{7.8pt} % Word 模板“段后 7.8pt”
\XeTeXlinebreaklocale "zh" % 中文断行

% Word 空白段落的“视觉高度”对齐:在已有 \parskip 的情况下补足到 1 行行距
\newcommand{\NSFCBlankPara}{\vspace*{\dimexpr\baselineskip-\parskip\relax}}

% 正文段首缩进:仅在 extraTex 正文中启用,避免与 main.tex 的提示语缩进叠加
\newcommand{\NSFCBodyText}{%
  \setlength{\parindent}{2em}%
  % 正文段间距:用户常希望“段间距和行间距一样紧凑”,因此在正文区块内关闭段后距
  % (不影响 main.tex 提示语区域的 Word 风格段后距设置)。
  \setlength{\parskip}{0pt}%
}


%%%————————————————————————————————图片与表格

% 对于表格,推荐使用Excel的插件Excel2LaTeX自动生成LaTeX代码。详见: https://www.zhihu.com/question/307970489/answer/2305355098

%图片
\usepackage{graphicx} %插图宏包
\usepackage{float} %设置图片浮动位置的宏包
\usepackage{subfigure} %插入多图时用子图显示的宏包
\usepackage[section]{placeins} %避免浮动体跨过\section
\usepackage{enumerate} %设置列表环境的包
\usepackage{enumitem} % 增强自定义Bullet(即各种List前的那个标记)
\setlist[enumerate]{
  label={\templatefont \bfseries \hspace{1em} \color{MsBlue}(\arabic*)},
  leftmargin=0em, % 列表的左边界
  itemindent=4em, % item首行缩进
  itemsep=0em, % 列表项之间的垂直间距。
  labelsep=0.1pt,  % label与content的距离
  parsep=0em,  % 两个段落之间的垂直间距。效果上与itemsep类似。
  topsep=0em % 列表与上一段对象的垂直间距。
  % after=\vspace{5pt}
}
% \newcommand{\itemtitlefont}[1]{\textbf{\color{MsBlue} #1}} % 定义item小标题字体;
\newcommand{\itemtitlefont}[1]{{\bfseries \color{MsBlue} #1}} % 定义item小标题字体;
\usepackage{caption}
\captionsetup{font={footnotesize,stretch=1.25},labelsep=period,labelfont=bf, singlelinecheck=off, justification=centering}

%表格
\usepackage{booktabs} %表格
\usepackage{tabularx} %表格宽度
\usepackage{multirow} %多行合并
\usepackage{longtable} % latex 多页显示同一表格
\usepackage{tabu} % 使用longtabu(tabularx + longtable)将长文本在单元格多行显示。
% \captionsetup[table]{name={表},font={footnotesize,stretch=1.25},labelsep=period,labelfont=bf, singlelinecheck=off, justification=centering} 


%%%————————————————————————————————超链接
\usepackage[
	colorlinks,
	urlcolor=MsBlue, % 超链接的颜色。默认是芭比粉
	linkcolor=black,
	anchorcolor=black,
	citecolor=black,
	CJKbookmarks=True
]{hyperref}
\pdfstringdefDisableCommands{%
  \def\linebreak{}%
}

%%%————————————————————————————————参考文献(兼容“xe->bib->xe->xe”)
% 目标:
% 1) 即使正文暂时没有任何 \cite,也能顺利执行 bibtex(不报“no \\bibdata/no \\bibstyle/no \\citation”)。
% 2) 只有在正文实际使用了引用时,才在文末输出“参考文献”,避免空参考文献页影响版式对齐。
\newif\ifNSFCHasCite
\NSFCHasCitefalse
\AtBeginDocument{%
  \pretocmd{\cite}{\global\NSFCHasCitetrue}{}{}%
  \pretocmd{\nocite}{\global\NSFCHasCitetrue}{}{}%
}
\makeatletter
% 参考文献由 main.tex 中的 \clearpage
\bibliographystyle{bibtex-style/gbt7714-nsfc.bst}
\bibliography{references/myexample} 手动控制
% \AtBeginDocument{%
%   \immediate\write\@auxout{\string\bibdata{references/myexample}}%
%   \immediate\write\@auxout{\string\bibstyle{bibtex-style/gbt7714-nsfc}}%
% }
% \AtEndDocument{%
%   \ifNSFCHasCite
%     \bibliographystyle{bibtex-style/gbt7714-nsfc}
%     \bibliography{references/myexample}
%   \else
%     % 若当前文档未使用任何引用,为了让"xe->bib->xe->xe"流程不报错,
%     % 这里写入一个"哑引用"到 aux(bibtex 能正常运行,但不会在 PDF 中输出参考文献)。
%     \immediate\write\@auxout{\string\citation{Smith1900}}%
%   \fi
% }
\makeatother

%%%————————————————————————————————参考文献间距(可调,且不影响其它样式)
% 间距调节(与 Local/Young 完全一致):
% - 标题与上文:\NSFCBibTitleAboveSkip(进入参考文献块前的额外垂直距离)
% - 标题与条目:\NSFCBibTitleBelowSkip(在标题后额外增加/减少的垂直距离)
% - 条目间距:\NSFCBibItemSep(thebibliography 的 \itemsep)
% - 条目行宽:\NSFCBibTextWidth(用于跨项目消除换行差异)
%
% 用法:直接改下面四个 length 的值即可(允许负值微调)。
\newlength{\NSFCBibTitleAboveSkip}
\newlength{\NSFCBibTitleBelowSkip}
\newlength{\NSFCBibItemSep}
\newlength{\NSFCBibTextWidth}
% 默认值:保持当前模板视觉效果(对应旧版 reference.tex 的 \vspace{15pt})
\setlength{\NSFCBibTitleAboveSkip}{15pt}
\setlength{\NSFCBibTitleBelowSkip}{0pt}
\setlength{\NSFCBibItemSep}{0pt}
\setlength{\NSFCBibTextWidth}{397.16727pt} % 推荐默认值:按 Young 的"参考文献条目有效行宽"对齐,用于降低跨项目换行差异;需要更宽/更窄可自行调整
% 说明:上述参数会在本文件末尾的 thebibliography 环境重定义中生效(仅影响参考文献块)。


%%%————————————————————————————————标题计数

% [latex 标题、段落及行距 - 简书](https://www.jianshu.com/p/d7848f815e5f/)
% [LaTeX 中的行间距](https://yinguobing.com/linespace-in-latex/)

\usepackage{titlesec}

\setcounter{secnumdepth}{4} %显示3级目录,然后再具体定制不同的目录
% \setcounter{tocdepth}{4}

% Define section font
\newcommand{\sectionzihao}{\fontsize{14pt}{22pt}\selectfont}
\newcommand{\subsectionzihao}{\fontsize{14pt}{22pt}\selectfont}
\newcommand{\subsubsectionzihao}{\fontsize{13.5pt}{20pt}\selectfont}

% Custom format for section
\titleformat{\section}
  {\color{MsBlue} \sectionzihao \templatefont \bfseries} % format
  {\hspace*{2em}} % label(用于缩进,不影响标题文字内容)
  {0pt} % separation
  {}   % before-code

% Custom format for subsection
\titleformat{\subsection}
  {\color{MsBlue} \subsectionzihao \templatefont} % format
  {} % label
  {0pt} % separation
  {\hspace*{2em}}   % before-code(用于首行缩进,续行自然顶格)

% Custom format for subsubsection
\titleformat{\subsubsection}
  {\color{MsBlue} \subsubsectionzihao \templatefont \bfseries} % format
  {\hspace{1.1em}  \textnormal{\templatefont \arabic{subsection}.\arabic{subsubsection}}} % label
  {0.5em} % separation
  {}   % before-code
\renewcommand\thesubsubsection{\arabic{subsection}.\arabic{subsubsection}} % Reference rule for subsubsection

% Custom format for subsubsubsection
\titleclass{\subsubsubsection}{straight}[\subsubsection]
\newcounter{subsubsubsection}[subsubsection]
\renewcommand\thesubsubsubsection{(\arabic{subsubsubsection})}
\titleformat{\subsubsubsection}
  {\templatefont \bfseries} % format \color{MsBlue} 
  {\hspace{1em} (\arabic{subsubsubsection})} % label
  {0.5pt} % separation
  {}   % before-code
\makeatletter
\def\toclevel@subsubsubsection{4}
\def\l@subsubsubsection{\@dottedtocline{4}{7.0em}{4em}}
\makeatother

% Define line stretch for section
\titlespacing*{\section}
  {0pt} % Left indentation
  {0pt plus 0pt minus 0pt} % Space before the title
  {-7pt} % Space after the title(大幅减小与 subsection 的距离)

% Define line stretch for subsection
\titlespacing*{\subsection}
  {0pt} % Left indentation
  {-8pt plus 0pt minus 0pt} % Space before the title(大幅减小与 section 的距离)
  {0pt} % Space after the title(由 \parskip 控制段后)

% Define line stretch for subsubsection
\titlespacing*{\subsubsection}
  {0pt} % Left indentation
  {0pt plus 0pt minus 0pt} % Space before the title
  {0pt plus 0pt minus 0pt} % Space after the title

% Define line stretch for subsubsection
\titlespacing*{\subsubsubsection}
  {0pt} % Left indentation
  {0pt plus 0pt minus 0pt} % Space before the title
  {0pt plus 0pt minus 0pt} % Space after the title


%%%————————————————————————————————代码
\usepackage{listings} %代码块
\usepackage{xcolor}%代码块
\renewcommand{\lstlistingname}{Code}

% 参考:
% https://www.youtube.com/watch?v=qxkQgG1Y0bY
% https://zhuanlan.zhihu.com/p/65441079

% \lstset{
% 	% language=bash,  %代码语言使用的是matlab
% 	frame=shadowbox, %把代码用带有阴影的框圈起来
% 	rulesepcolor=\color{red!20!green!20!blue!20},%代码块边框为淡青色
% 	keywordstyle=\color{blue!90}\bfseries, %代码关键字的颜色为蓝色,粗体
% 	commentstyle=\color{red!10!green!70}\textit,% 设置代码注释的颜色
% 	showstringspaces=true,%不显示代码字符串中间的空格标记
% 	numbers=left, % 显示行号
% 	numberstyle=\tiny,    % 行号字体
% 	stringstyle=\ttfamily, % 代码字符串的特殊格式
% 	breaklines=false, %对过长的代码自动换行
% 	extendedchars=false,  %解决代码跨页时,章节标题,页眉等汉字不显示的问题
% 	escapebegin=\begin{CJK*},escapeend=\end{CJK*},% 代码中出现中文必须加上,否则报错
% 	texcl=true
% }

\lstdefinestyle{codestyle01}{
	backgroundcolor=\color{gray!12},
	%basicstyle=\ttfamily\small,
	basicstyle=\zihao{-5}\ttfamily,
	commentstyle=\color{green!60!black},
	keywordstyle=\color{magenta},
	stringstyle=\color{blue!50!red},
	showstringspaces=false,
	numbers=left,
	numberstyle=\footnotesize\color{gray},
	numbersep=10pt,
	stepnumber=1,
	tabsize=4,
	frame=single,
	framerule=0.3pt,
	rulecolor=\color{black!20},
	xleftmargin=2em,
	xrightmargin=0.5em,
	breaklines=true,
	breakatwhitespace=true,
	captionpos=b
	% frame=tblr
	% frame=L,
	% framerule=1pt,
	% rulecolor=\color{red}
}
\lstset{style=codestyle01}


%%%————————————————————————————————其它设置


\makeatletter	
	\setlength{\@fptop}{0pt} % 图片置顶展示(而不是居中)
  % \setlength{\textfloatsep}{1cm plus 1.0pt minus 2.0pt} % 浮动体(如figure)之间的垂直间距
  \setlength{\intextsep}{0.5cm plus 1.0pt minus 2.0pt} % 浮动体(如figure)与文字的垂直间距
\makeatother

% 去除页码
\pagestyle{empty}

% 汉字与标点的间距;
% \punctstyle{banjiao} % 所有标点半角
% \punctstyle{kaiming} % 部分的标点半角
\punctstyle{hangmobanjiao} % 末半角式,仅行末挤压。

% 参考文献间距(可调):标题与上文/标题与条目/条目间距(见上方 \NSFCBib* 三个参数)
% 说明:不使用 \section*{\refname}(避免 titlesec 对 \section 的 spacing 差异影响参考文献),
% 仅在 thebibliography 内部手工排标题与列表参数,从而保证三套项目在“同样参考文献参数”下表现一致。
\makeatletter
\renewenvironment{thebibliography}[1]{%
  \par\addvspace{\NSFCBibTitleAboveSkip}%
  \noindent{\refname}\par%
  \vspace{\NSFCBibTitleBelowSkip}%
  \frenchspacing%
  \list{\@biblabel{\@arabic\c@enumiv}}{%
    \settowidth\labelwidth{\@biblabel{#1}}%
    \leftmargin\labelwidth
    \advance\leftmargin\labelsep
    % 统一条目行宽(跨项目像素级一致性依赖此项)
    \rightmargin=\dimexpr\textwidth-\NSFCBibTextWidth-\leftmargin\relax
    \ifdim\rightmargin<0pt\relax \rightmargin=0pt\relax \fi
    \setlength{\itemsep}{\NSFCBibItemSep}%
    \setlength{\parsep}{0pt}%
    \setlength{\parskip}{0pt}%
    \setlength{\topsep}{0pt}%
    \setlength{\partopsep}{0pt}%
    \usecounter{enumiv}%
    \let\p@enumiv\@empty
    \renewcommand\theenumiv{\@arabic\c@enumiv}%
  }%
  \sloppy\clubpenalty4000\widowpenalty4000%
  \sfcode`\.=1000\relax%
}{%
  \def\@noitemerr{\@latex@warning{Empty `thebibliography' environment}}%
  \endlist%
}
\makeatother

% 参考文献中的链接字体用的是texttt
\renewcommand{\ttdefault}{tnr} % 设置默认等宽字体为 Times New Roman

% 自定义第4级小标题
% \newcommand{\ssssubtitle}[1]{\ding{\numexpr171+#1\relax}} % 类似①
\newcommand{\ssssubtitle}[1]{\textcircled{\raisebox{-0.8pt}{\xiaosihao #1}}}
% \newcommand{\ssssubtitle}[1]{#1)} % 类似1)
% \newcommand{\blackding}[1]{\ding{\numexpr181+#1\relax}}
% \newcommand{\whitedingB}[1]{\ding{\numexpr191+#1\relax}}
% \newcommand{\blackdingB}[1]{\ding{\numexpr201+#1\relax}}


%%%%============================正  文=============================%%%%
\begin{document}

{\centering
{\sanhao \templatefont \hspace{2em} 报告正文(2026 版)}\par
}

{\fontsize{14pt}{14pt}\selectfont \templatefont
\setlength{\parindent}{1.45em}
\setlength{\parskip}{0pt}
\indent 参照以下提纲撰写,要求内容翔实、清晰,层次分明,标题突出
\par
}

{\fontsize{14pt}{14pt}\selectfont \templatefont
\setlength{\parindent}{0pt}
\setlength{\parskip}{0pt}
\noindent {\color{red}申请书正文原则上不超过 30 页,鼓励简洁表达。}{\color{MsBlue}请勿删除或改动下述提纲标题及括号中的文字。}
\par
}

\section{(一)立项依据:}

{\fontsize{14pt}{14pt}\selectfont \templatefont \color{MsBlue}
\setlength{\parindent}{2em}
\setlength{\parskip}{0pt}
\indent (为什么要开展此项研究,研究的科学技术价值如何)
\par
}

% “立项依据”(2026 版)建议写作顺序:
% 背景/意义 → 国内外现状与不足 → 关键科学问题 → 研究思路与预期贡献
\justifying
\NSFCBodyText

\subsubsection{研究背景}

\subsubsubsection{案例概述}
\indent 刘亦菲(1987年生)作为中国演艺产业代表性人物,其职业生涯从2002年持续至今,跨越童星、电视剧演员、电影演员、国际巨星等多个阶段。2002年出演《金粉世家》进入演艺圈,2005年凭借《仙剑奇侠传》中"赵灵儿"一角成为现象级偶像。2006-2008年主演《神雕侠侣》《天龙八部》等金庸武侠剧,奠定"古装女神"地位。2012年起向电影转型,主演《铜雀台》《露水红颜》等作品。2020年主演迪士尼真人版《花木兰》,成为首位担纲好莱坞A级制作女主角的华人演员,实现国际化突破。

\subsubsubsection{研究价值}
\indent 刘亦菲的职业发展路径为理解中国娱乐产业变迁、女性艺人国际化策略及跨文化形象管理提供了极具价值的案例。其职业生涯跨越中国影视产业从电视剧主导向电影主导、从国内市场向国际市场转型的关键期(2005-2025)。作为女性艺人,她在"古装女神"到"国际巨星"的转型中面临的挑战具有典型性。《花木兰》项目为研究华人演员好莱坞突破机制提供了珍贵实证。

\subsubsubsection{研究问题}

\subsubsection{国内外研究现状}

\subsubsubsection{职业转型理论}
\indent 目前研究主要集中三个方向:Smith(2020)提出"职业锚理论"\cite{Smith2020};Super(1957)划分职业发展五阶段\cite{Super1957},但缺乏中国演艺产业语境研究。

\subsubsubsection{国际化路径研究}
\indent Johnson(2019)提出"本土-区域-全球"三阶段国际化模型\cite{Johnson2019};Lee(2018)分析亚洲演员好莱坞突破策略\cite{Lee2018},但对华人女演员的文化适应机制研究不足。

\subsubsubsection{品牌形象管理}

\subsubsection{现有研究的局限性}

\subsubsubsection{理论层面}
\indent 现有研究存在以下不足:
\begin{enumerate}[label=\ssssubtitle{\arabic*}, leftmargin=*, itemsep=0.2em]
  \item 缺乏完整职业生命周期的纵向案例研究。
  \item 对华人演员好莱坞突破机制探讨不足。
  \item 忽视女性艺人在跨文化转型中的特殊挑战。
\end{enumerate}

\subsubsubsection{方法层面}

\subsubsection{研究切入点}

\subsubsubsection{案例独特性}
\indent 刘亦菲的职业轨迹具有独特研究价值:(1)转型涵盖"童星→古装女神→电影演员→国际巨星"完整链条;(2)成功突破好莱坞A级制作,成为华人女演员国际化标杆;(3)职业发展与中国影视产业国际化进程高度同步。

\subsubsubsection{研究代表性}



\section{(二)研究内容:}

{\fontsize{14pt}{14pt}\selectfont \templatefont \color{MsBlue}
\setlength{\parindent}{2em}
\setlength{\parskip}{0pt}
\indent (提纲不做限制,请按照研究工作的自身逻辑撰写。应提炼出特色与创新点、年度研究计划)
\par
}

\justifying
\indent\setlength{\parindent}{2em}%首行缩进4字符

% 在此处撰写“研究内容”正文内容
\vspace{0.8\baselineskip}


\section{(三)研究基础:}

\subsection{\hspace{1.45em} 1.研究基础与可行性分析(与本项目相关的研究工作积累和已取得的研究工作成绩,研究风险的应对措施等);}

\begingroup
\justifying
\setlength{\parindent}{2em}% 首行缩进 2 字符

% "研究基础与可行性分析"(2026 版)建议写作顺序:
% 与本项目相关的研究工作积累 → 已取得的研究工作成绩 → 研究风险的应对措施

\subsubsection{研究基础}
% 建议:梳理前期工作,说明本项目延续和深化的基础。

\textbf{【示例内容】}

申请团队长期致力于该领域的基础研究和应用探索,积累了丰富的研究经验。近年来,团队主持完成了国家自然科学基金项目"XXX理论及其应用研究"(批准号:XXXXXXXX,资助金额:XX万元,起止年月:20XX年XX月—20XX年XX月),在该项目的支持下,团队在理论建模、算法设计和系统实现三个方面取得了系统性进展,为本项目的开展奠定了坚实基础。

同时,团队还参与了国家重点研发计划"YYY关键技术研究"(项目编号:YYYYYYYY),负责其中的算法优化模块。通过该项目,团队积累了大规模工程化实践经验,培养了技术攻关能力,建立了完整的研发流程和质量管理体系。

在学术交流方面,团队与美国XXX大学、英国YYY研究所等国际一流研究机构建立了长期合作关系,定期开展学术访问和联合研究,为本项目的国际化视野和前沿性提供了保障。

\NSFCBlankPara

\subsubsection{已取得的研究工作成绩}
% 建议:列出相关论文、专利、预实验结果等,证明团队能力。

\textbf{【示例内容】}

\textbf{学术论文:}团队在该领域顶级期刊和会议上发表了系列研究成果。近5年来,发表SCI收录论文XX篇,其中TOP期刊论文XX篇,包括IEEE Transactions on XXX(2篇)、ACM XXX(1篇)、中国科学(1篇)等。这些论文被国内外同行广泛引用(Google Scholar引用次数:XXX次),产生了重要的学术影响。

\textbf{发明专利:}申请国家发明专利XX项,其中已授权XX项。核心专利"一种基于XXX的高效算法"(专利号:ZLXXXXXXXXXX.X)已成功转化应用,产生了显著的经济效益。

\textbf{软件著作权:}获得软件著作权XX项,开发的"XXX数据分析系统"已在多家单位和科研机构得到应用。

\textbf{获奖情况:}研究成果获得XXX科学技术奖一等奖(20XX年)、YYY协会优秀论文奖(20XX年)等多项奖励。

\textbf{预实验结果:}团队已完成初步的概念验证实验,实验结果表明新方法在效率上较现有方法提升XX%,精度提升XX%,验证了核心思路的可行性和有效性。

\NSFCBlankPara

\subsubsection{可行性分析}
% 建议:按"理论/技术/团队/条件"四维写可行性;并给出 2–3 条风险与备选方案。

\textbf{【示例内容】}

\begin{enumerate}
  \item \itemtitlefont{理论可行性:}\par
  本项目提出的理论框架建立在坚实的数学基础之上,相关理论已经过严格的数学推导和证明。团队前期的理论研究表明,该框架在理论上是完备的,能够覆盖现有理论的特殊情况,并在关键假设上有所突破。通过大量的文献调研和专家咨询,我们确认该理论方向是可行的,具有明确的研究价值和应用前景。

  \item \itemtitlefont{技术可行性:}\par
  团队在相关技术领域积累了丰富的经验,掌握了核心算法的关键技术。前期的预实验结果表明,技术路线是可行的,主要技术难点已经找到解决方案。团队配备了先进的实验设备和计算资源,包括高性能计算集群(XX核XX内存)、GPU加速卡(XX块)等,能够满足项目的技术需求。

  \item \itemtitlefont{团队与条件可行性:}\par
  \textbf{团队构成:}项目负责人为XXX教授/研究员,长期从事该领域研究,发表SCI论文XX篇,主持国家级项目XX项。研究团队包括教授/研究员X名,副教授/副研究员X名,博士/硕士研究生X名,形成了合理的人才梯队。团队成员在理论、算法、系统等方面各有所长,能够协同攻关。\par
  \textbf{实验条件:}依托XXX国家重点实验室/XXX教育部重点实验室,拥有完备的实验设备和先进的计算平台,能够满足项目的研究需求。\par
  \textbf{合作基础:}与国内外多个顶尖研究机构建立了长期合作关系,能够提供必要的技术支持和学术交流。

  \item \itemtitlefont{主要风险与对策:}\par
  \textbf{风险1:}理论模型可能存在未预见的局限性。\par
  \textbf{对策:}采用渐进式研究策略,先在简化模型上验证核心思想,再逐步扩展到一般情况;建立定期专家咨询机制,及时发现和解决问题。\par

  \textbf{风险2:}算法性能可能达不到预期指标。\par
  \textbf{对策:}设计多套备选方案,在不同技术路线并行推进;引入最新的优化技术,定期进行性能评估和调整。\par

  \textbf{风险3:}实验数据获取可能存在困难。\par
  \textbf{对策:}提前与数据提供方建立合作关系,制定数据共享协议;开发数据合成和增强技术,减少对真实数据的依赖。
\end{enumerate}

\endgroup


\subsection{\hspace{1.45em} 2.工作条件(包括已具备的实验条件,尚缺少的实验条件和拟解决的途径,包括利用国家实验室、全国重点实验室和部门重点实验室等研究基地的计划与落实情况);}


\justifying

\subsubsection{已具备的实验条件}

\indent\setlength{\parindent}{2em}

\textbf{硬件设施}:

依托单位文化产业研究中心配备了完善的科研设施,能够满足本项目的需求:(1)高性能计算集群,配备NVIDIA A100 GPU 4张,总内存512GB,可支持深度学习模型的训练与推理;(2)大容量数据存储系统,总容量500TB,支持多模态研究数据的长期存储与管理;(3)专业级影像处理工作站,配备高精度显示器,用于影像资料的标注与分析;(4)专业录音与视频编辑设备,用于多媒体资料的采集与处理。

\textbf{软件资源}:

实验室已部署完整的数据分析工具链:(1)自然语言处理工具包(NLTK、spaCy、jieba等),支持多语言文本分析;(2)深度学习框架(TensorFlow、PyTorch、Keras),用于构建和训练神经网络模型;(3)计算机视觉工具(OpenCV、PIL、scikit-image),用于影像资料的特征提取;(4)数据可视化工具(Tableau、D3.js、Matplotlib),用于研究结果的呈现;(5)社会科学统计软件(SPSS、Stata、R),用于传统统计分析。

\textbf{数据库资源}:

依托单位购买了以下学术数据库的使用权限:(1)外文数据库:Web of Science、Scopus、JSTOR、SpringerLink、ScienceDirect、EBSCO等;(2)中文数据库:中国知网(CNKI)、万方数据、维普资讯、人大复印报刊资料等;(3)专业数据库:MyDramalist(影视作品数据库)、AsianWiki(亚洲艺人数据库)等。这些数据库为本项目的文献调研与数据收集提供了坚实基础。

\subsubsection{尚缺少的实验条件及拟解决途径}

\indent\setlength{\parindent}{2em}

\textbf{缺少的实验条件}:

\begin{enumerate}
    \item \itemtitlefont{日本本土数据资源访问受限}:
    \ssssubtitle{1}部分日本本土的社交媒体数据(如Twitter日本区历史数据、5ch论坛历史帖子)因地域限制和平台政策,无法完整获取。
    \ssssubtitle{2}日本娱乐产业的一手资料(如经纪公司内部数据、艺人合约模板等)属于商业机密,难以直接获取。

    \item \itemtitlefont{跨文化比较研究资源不足}:
    \ssssubtitle{1}韩国娱乐产业的相关数据(如K-pop艺人职业发展数据)需要与韩国研究机构合作获取。
    \ssssubtitle{2}欧洲与北美娱乐产业的数据资源相对分散,缺乏系统性的获取渠道。
\end{enumerate}

\textbf{拟解决途径}:

\begin{enumerate}
    \item \itemtitlefont{建立国际合作网络}:
    \ssssubtitle{1}与日本东京大学、早稻田大学的相关研究机构建立合作关系,通过学术交流获取日本本土数据资源。
    \ssssubtitle{2}与韩国首尔国立大学、汉阳大学的媒体研究中心建立合作关系,开展中、日、韩三国娱乐产业比较研究。
    \ssssubtitle{3}与欧洲的阿姆斯特丹大学、北美的南加州大学的相关研究者建立联系,拓展跨文化比较研究网络。

    \item \itemtitlefont{利用合法合规的数据服务提供商}:
    \ssssubtitle{1}通过与合法的数据服务提供商(如Brandwatch、Talkwalker等)合作,获取社交媒体历史数据。
    \ssssubtitle{2}利用公开数据集(如Twitter Academic Research Product Track)获取研究所需的基础数据。

    \item \itemtitlefont{开发数据采集工具}:
    \ssssubtitle{1}自主研发网络爬虫工具,在遵守平台服务条款的前提下,公开采集可用数据。
    \ssssubtitle{2}开发多语言情感分析工具,提高对日语、韩语等小语种数据的处理能力。
\end{enumerate}

\subsubsection{利用国家重点实验室等研究基地的计划}

\indent\setlength{\parindent}{2em}

\textbf{计划利用的国家重点实验室}:

\begin{enumerate}
    \item \itemtitlefont{中国人民大学新闻学院社会发展研究基地}:
    \ssssubtitle{1}该基地在传播学研究领域具有深厚积累,可为本项目提供理论指导与方法支持。
    \ssssubtitle{2}计划于第一年派遣研究团队成员赴该基地进行为期3个月的访学交流,学习先进的研究方法。

    \item \itemtitlefont{清华大学新闻与传播学院新媒体研究中心}:
    \ssssubtitle{1}该中心在新媒体传播研究方面处于国内领先地位,可为本项目提供计算传播学方法支持。
    \ssssubtitle{2}计划于第二年与该中心联合举办"文化产业数字化转型"学术研讨会。

    \item \itemtitlefont{中国传媒大学文化产业管理学院}:
    \ssssubtitle{1}该学院在文化产业管理研究方面具有丰富经验,可为本项目提供政策解读与实践指导。
    \ssssubtitle{2}计划于第三年与该学院合作开展"艺人职业发展"专题研究。
\end{enumerate}

\textbf{计划利用的部门重点实验室}:

\begin{enumerate}
    \item \itemtitlefont{文化和旅游部文化产业研究中心}:
    \ssssubtitle{1}该中心掌握文化产业的宏观政策数据,可为项目提供政策环境分析支持。
    \ssssubtitle{2}计划邀请中心专家参与项目咨询,确保研究成果符合政策导向。

    \item \itemtitlefont{国家广播电视总局广播电视规划院}:
    \ssssubtitle{1}该院拥有影视产业的一手统计数据,可为项目提供权威数据支持。
    \ssssubtitle{2}计划于项目执行期与该院建立数据共享机制。
\end{enumerate}

\textbf{落实情况}:

申请团队已与上述研究基地的负责人进行了初步沟通,均表示愿意支持本项目的开展。具体合作计划将在项目获批后进一步细化落实。

\clearpage


\subsection{\hspace{1.45em} 3.正在承担的与本项目相关的科研项目情况(申请人和主要参与者正在承担的与本项目相关的科研项目情况,包括国家自然科学基金的项目和国家其他科技计划项目,要注明项目的资助机构、项目类别、批准号、项目名称、获资助金额、起止年月、与本项目的关系及负责的内容等);}

% 在此处撰写“正在承担的与本项目相关的科研项目情况”正文内容(写正文后可删除下行占位空白)
\NSFCBlankPara


\subsection{\hspace{1.45em} 4.完成国家自然科学基金项目情况(对申请人负责的前一个已资助期满的科学基金项目(项目名称及批准号)完成情况、后续研究进展及与本申请项目的关系加以详细说明。另附该项目的研究工作总结摘要(限 500 字)和相关成果详细目录)。}

% 在此处撰写“完成国家自然科学基金项目情况”正文内容(写正文后可删除下行占位空白)
\NSFCBlankPara



\section{(四)其他需要说明的情况:}

{\fontsize{14pt}{14pt}\selectfont \templatefont \color{MsBlue}
\setlength{\parindent}{2em}
\setlength{\parskip}{0pt}
\indent 1. 申请人同年申请不同类型的国家自然科学基金项目情况(列明同年申请的其他项目的项目类型、项目名称信息,并说明与本项目之间的区别与联系;已收到自然科学基金委不予受理或不予资助决定的无需列出)。
\par
}
\vspace{0.3\baselineskip}

\justifying
\NSFCBodyText

% 在此处撰写“2. 同年单位不一致”的说明内容(写正文后可删除下行占位空白)

% 在此处撰写“3. 承担中单位不一致”的说明内容(写正文后可删除下行占位空白)
\NSFCBlankPara

{\fontsize{14pt}{14pt}\selectfont \templatefont \color{MsBlue}
\setlength{\parindent}{2em}
\setlength{\parskip}{0pt}
\indent 4. 申请人和主要参与者同年以不同专业技术职务(职称)申请或参与申请科学基金项目的情况(应详细说明原因)。
\par
}
\vspace{0.3\baselineskip}

{\fontsize{14pt}{14pt}\selectfont \templatefont \color{MsBlue}
\setlength{\parindent}{2em}
\setlength{\parskip}{0pt}
\indent 5. 申请人在撰写本申请书时使用生成式人工智能的情况,请详细说明申请书中使用的位置和内容。
\par
}
\vspace{0.3\baselineskip}

{\fontsize{14pt}{14pt}\selectfont \templatefont \color{MsBlue}
\setlength{\parindent}{2em}
\setlength{\parskip}{0pt}
\indent 6. 其他(包括但不限于使用以他人名义申报过的申请书;如有,请详细说明)。
\par
}
\vspace{0pt}


\clearpage
\end{document}
