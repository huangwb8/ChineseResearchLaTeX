\begingroup
\justifying
\setlength{\parindent}{2em}% 首行缩进 2 字符

\subsubsection{研究背景与科学意义}
% 建议:1–2 段。需求/痛点 → 科学问题缺口 → 本项目意义(科学价值/方法学价值/潜在应用)。
示例:参考文献引用展示 \cite{Smith1900},多项研究 \cite{Smith1900,Piter1992,John1997} 证实了该方向的重要性。
\NSFCBlankPara

\subsubsection{国内外研究现状与不足}
% 建议:2–3 段。现状梳理(可分方向/代表性工作)→ 核心不足(方法/数据/理论/转化/工程)。

【示例:图片插入】图 \ref{fig:research-status} 展示了研究现状的梳理结果。

\begin{figure}[!th]
    \begin{center}
        \includegraphics[width=0.8\linewidth]{figures/zzmx-115.jpg}
        \caption{研究现状示意图。\\
        \raggedright \justifying \noindent
        本图展示了该领域的发展历程和主要研究方向。从图中可以看出,现有研究主要集中在方法A和方法B两个方向,但都存在一定局限性。}
        \label{fig:research-status}
    \end{center}
\end{figure}

\clearpage

【示例:表格插入】表 \ref{table1} 展示了临床特征数据的统计分析结果。

\begin{table}[htbp]
    \centering
    \fontsize{10}{10}\selectfont
    \caption{\textbf{临床特征统计分析表}}
    \begin{tabular}{lllll}
    \toprule
            & \textbf{Characteristics} & \textbf{High (n=137)} & \textbf{Low (n=235)} & \textbf{\textit{P} value} \\
    \toprule
    \textbf{Gender (\%)} & Male  & 85 (62.0) & 154 (65.5) & 0.572 \\
            & Female & 52 (38.0) & 81 (34.5) &  \\
    \textbf{Age (mean (SD))} &       & 65.50 (10.22) & 66.06 (10.95) & 0.628 \\
    \textbf{Race (\%)} & White & 98 (81.0) & 139 (69.5) & 0.054 \\
            & Asian & 21 (17.4) & 51 (25.5) &  \\
            & Other & 2 (1.7) & 10 (5.0) &  \\
    \textbf{Tumor position (\%)} & GEJ   & 17 (12.9) & 27 (11.9) & 0.03 \\
            & Cardia & 11 (8.3) & 36 (15.9) &  \\
            & Fundus & 48 (36.4) & 54 (23.8) &  \\
            & Body  & 7 (5.3) & 23 (10.1) &  \\
            & Antrum & 49 (37.1) & 87 (38.3) &  \\
    \textbf{Pathology (\%)} & Intestinal & 65 (52.4) & 167 (78.8) & <0.001 \\
            & Diffuse & 52 (41.9) & 35 (16.5) &  \\
            & Mixed & 7 (5.6) & 10 (4.7) &  \\
    \textbf{Grade (\%)} & G1    & 1 (0.7) & 9 (4.0) & <0.001 \\
            & G2    & 28 (20.6) & 108 (47.6) &  \\
            & G3    & 107 (78.7) & 110 (48.5) &  \\
    \textbf{T stage (\%)} & T1    & 2 (1.6) & 17 (7.3) & 0.009 \\
            & T2    & 24 (18.6) & 42 (17.9) &  \\
            & T3    & 38 (29.5) & 91 (38.9) &  \\
            & T4    & 65 (50.4) & 84 (35.9) &  \\
    \textbf{N stage (\%)} & N0    & 37 (29.1) & 76 (33.2) & 0.504 \\
            & Np    & 90 (70.9) & 153 (66.8) &  \\
    \textbf{M stage (\%)} & M0    & 121 (92.4) & 206 (93.2) & 0.933 \\
            & M1    & 10 (7.6) & 15 (6.8) &  \\
    \textbf{Stage (\%)} & I     & 10 (8.3) & 37 (17.1) & 0.171 \\
            & II    & 40 (33.3) & 66 (30.6) &  \\
            & III   & 60 (50.0) & 98 (45.4) &  \\
            & IV    & 10 (8.3) & 15 (6.9) &  \\
    \textbf{MSI status (\%)} & MSI-H & 21 (21.0) & 26 (19.0) & 0.001 \\
            & MSI-L & 6 (6.0) & 32 (23.4) &  \\
            & MSS   & 73 (73.0) & 79 (57.7) &  \\
    \textbf{EBV infection (\%)} & Positive & 18 (18.0) & 5 (3.6) & 0.001 \\
            & Negative & 82 (82.0) & 132 (96.4) &  \\
    \textbf{Purity (mean (SD))} &       & 0.40 (0.17) & 0.57 (0.19) & <0.001 \\
    \textbf{Ploidy (mean (SD))} &       & 2.37 (0.67) & 2.69 (0.90) & 0.001 \\
    \bottomrule
    \end{tabular}%
    \captionsetup{font={footnotesize,stretch=1.25},justification=raggedright}
    \label{table1}%
\end{table}%

\subsubsection{拟解决的关键科学问题}
% 建议:2–4 条;每条尽量“可检验/可量化/可证伪”,并与后续研究内容一一对应。
\begin{enumerate}
  \item \itemtitlefont{关键科学问题 1:}\par\NSFCBlankPara
  \item \itemtitlefont{关键科学问题 2:}\par\NSFCBlankPara
  \item \itemtitlefont{关键科学问题 3:}\par\NSFCBlankPara
\end{enumerate}

\subsubsection{本项目研究思路与预期贡献}
% 建议:1–2 段。总体思路/核心假说 → 预期贡献(理论/方法/数据资源/平台原型/转化前景)。
\NSFCBlankPara
\NSFCBlankPara

\endgroup
