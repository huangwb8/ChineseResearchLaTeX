\begingroup
\justifying
\setlength{\parindent}{2em}% 首行缩进 2 字符

\subsubsection{研究背景与科学意义}
% 建议:1–2 段。需求/痛点 → 科学问题缺口 → 本项目意义(科学价值/方法学价值/潜在应用)。
示例:参考文献引用展示 \cite{Smith1900},多项研究 \cite{Smith1900,Piter1992,John1997} 证实了该方向的重要性。

\textbf{【示例内容】}
随着科学技术的快速发展,该领域面临着前所未有的机遇和挑战。现有研究表明 \cite{Smith1900},在该领域的理论研究和应用实践中仍存在显著差距。多项研究 \cite{Smith1900,Piter1992,John1997} 证实了开展此项研究的重要性和紧迫性。从科学价值来看,本项目的实施将填补理论空白,为相关领域提供新的研究视角和方法学支撑;在应用层面,研究成果有望转化为实际生产力,推动相关产业的发展和技术进步。\NSFCBlankPara

\subsubsection{国内外研究现状与不足}
% 建议:2–3 段。现状梳理(可分方向/代表性工作)→ 核心不足(方法/数据/理论/转化/工程)。

\textbf{【示例内容】}

\textbf{国内研究现状:}近年来,国内学者在该领域取得了显著进展。研究团队A在方法学上提出了创新性框架,研究团队B在应用层面进行了有益探索。然而,现有研究主要集中在理论分析层面,缺乏大规模实验验证和工程化应用。

\textbf{国际研究现状:}国际上,该领域的研究呈现出多学科交叉融合的趋势。代表性工作包括:Smith等人 \cite{Smith1900} 提出的经典理论框架,John团队 \cite{John1997} 在方法学上的突破,以及Piter等人 \cite{Piter1992} 在应用实践中的创新尝试。

\textbf{存在的主要不足:}
\begin{enumerate}
  \item \textbf{理论层面:}现有理论模型存在简化假设,难以描述复杂系统的真实行为;
  \item \textbf{方法层面:}传统方法的计算效率和精度难以兼顾,限制了实际应用;
  \item \textbf{数据层面:}缺乏高质量、标准化的数据集,影响了算法的客观评价;
  \item \textbf{转化层面:}研究成果与产业需求脱节,缺乏有效的转化机制。
\end{enumerate}

【示例:图片插入】图 \ref{fig:research-status} 展示了研究现状的梳理结果。

\begin{figure}[!ht]
    \begin{center}
        \includegraphics[width=0.7\linewidth]{figures/zzmx-115.jpg}
        \caption{研究现状示意图。本图展示了该领域的发展历程和主要研究方向。从图中可以看出,现有研究主要集中在方法A和方法B两个方向,但都存在一定局限性。}
        \label{fig:research-status}
    \end{center}
\end{figure}

【示例:表格插入】表 \ref{table1} 展示了临床特征数据的统计分析结果。

\begin{table}[htbp]
    \centering
    \small
    \caption{\textbf{临床特征统计分析表}}
    \begin{tabular}{lllll}
    \toprule
            & \textbf{Characteristics} & \textbf{High (n=137)} & \textbf{Low (n=235)} & \textbf{\textit{P} value} \\
    \midrule
    \textbf{Gender (\%)} & Male  & 85 (62.0) & 154 (65.5) & 0.572 \\
            & Female & 52 (38.0) & 81 (34.5) &  \\
    \textbf{Age (mean (SD))} &       & 65.50 (10.22) & 66.06 (10.95) & 0.628 \\
    \textbf{Race (\%)} & White & 98 (81.0) & 139 (69.5) & 0.054 \\
            & Asian & 21 (17.4) & 51 (25.5) &  \\
            & Other & 2 (1.7) & 10 (5.0) &  \\
    \textbf{Tumor position (\%)} & GEJ   & 17 (12.9) & 27 (11.9) & 0.03 \\
            & Cardia & 11 (8.3) & 36 (15.9) &  \\
            & Fundus & 48 (36.4) & 54 (23.8) &  \\
            & Body  & 7 (5.3) & 23 (10.1) &  \\
            & Antrum & 49 (37.1) & 87 (38.3) &  \\
    \textbf{Pathology (\%)} & Intestinal & 65 (52.4) & 167 (78.8) & <0.001 \\
            & Diffuse & 52 (41.9) & 35 (16.5) &  \\
            & Mixed & 7 (5.6) & 10 (4.7) &  \\
    \textbf{Grade (\%)} & G1    & 1 (0.7) & 9 (4.0) & <0.001 \\
            & G2    & 28 (20.6) & 108 (47.6) &  \\
            & G3    & 107 (78.7) & 110 (48.5) &  \\
    \textbf{T stage (\%)} & T1    & 2 (1.6) & 17 (7.3) & 0.009 \\
            & T2    & 24 (18.6) & 42 (17.9) &  \\
            & T3    & 38 (29.5) & 91 (38.9) &  \\
            & T4    & 65 (50.4) & 84 (35.9) &  \\
    \textbf{N stage (\%)} & N0    & 37 (29.1) & 76 (33.2) & 0.504 \\
            & Np    & 90 (70.9) & 153 (66.8) &  \\
    \textbf{M stage (\%)} & M0    & 121 (92.4) & 206 (93.2) & 0.933 \\
            & M1    & 10 (7.6) & 15 (6.8) &  \\
    \textbf{Stage (\%)} & I     & 10 (8.3) & 37 (17.1) & 0.171 \\
            & II    & 40 (33.3) & 66 (30.6) &  \\
            & III   & 60 (50.0) & 98 (45.4) &  \\
            & IV    & 10 (8.3) & 15 (6.9) &  \\
    \textbf{MSI status (\%)} & MSI-H & 21 (21.0) & 26 (19.0) & 0.001 \\
            & MSI-L & 6 (6.0) & 32 (23.4) &  \\
            & MSS   & 73 (73.0) & 79 (57.7) &  \\
    \textbf{EBV infection (\%)} & Positive & 18 (18.0) & 5 (3.6) & 0.001 \\
            & Negative & 82 (82.0) & 132 (96.4) &  \\
    \textbf{Purity (mean (SD))} &       & 0.40 (0.17) & 0.57 (0.19) & <0.001 \\
    \textbf{Ploidy (mean (SD))} &       & 2.37 (0.67) & 2.69 (0.90) & 0.001 \\
    \bottomrule
    \end{tabular}%
    \label{table1}%
\end{table}%

\subsubsection{拟解决的关键科学问题}
% 建议:2–4 条;每条尽量"可检验/可量化/可证伪",并与后续研究内容一一对应。

\textbf{【示例内容】}
基于上述研究现状和不足分析,本项目拟解决以下关键科学问题:

\begin{enumerate}
  \item \itemtitlefont{关键科学问题 1:如何建立高精度、高效率的理论模型?}\par
  该问题涉及理论框架的创新和数学方法的改进。拟通过引入新的建模方法,突破传统模型的局限性,实现对复杂系统的精确描述。预期成果包括:建立完整理论体系、发表高水平学术论文 2-3 篇。

  \item \itemtitlefont{关键科学问题 2:如何设计兼顾效率和精度的算法框架?}\par
  该问题关注算法层面的创新。拟通过优化算法结构和并行化策略,实现计算效率和精度的平衡。预期成果包括:开源算法工具包、申请软件著作权 1 项。

  \item \itemtitlefont{关键科学问题 3:如何构建标准化、可复用的数据资源?}\par
  该问题面向数据层面的突破。拟通过制定数据标准和建立数据共享平台,推动领域数据的标准化和规范化。预期成果包括:建立标准数据集、发表数据论文 1 篇。
\end{enumerate}

\subsubsection{本项目研究思路与预期贡献}
% 建议:1–2 段。总体思路/核心假说 → 预期贡献(理论/方法/数据资源/平台原型/转化前景)。

\textbf{【示例内容】}

\textbf{总体研究思路:}本项目采用"理论创新—方法突破—应用验证"三位一体的研究思路。核心假说是:通过引入新的理论建模方法和算法优化策略,可以在保持高精度的同时显著提升计算效率,从而解决现有方法在实用化方面的瓶颈问题。

\textbf{预期贡献:}\textit{理论贡献:}建立新的理论体系,填补该领域的理论空白;\textit{方法学贡献:}提出创新的算法框架,兼顾效率和精度;\textit{数据贡献:}构建标准化数据集,推动数据共享和复用;\textit{平台贡献:}开发开源工具平台,降低研究门槛;\textit{转化前景:}研究成果可直接应用于实际生产,推动相关产业的技术升级,产生显著的经济和社会效益。\NSFCBlankPara

\endgroup
