\begingroup
\justifying
\setlength{\parindent}{2em}% 首行缩进 2 字符

\subsubsection{研究背景}

\subsubsubsection{案例概述}
佐佐木希(Nozomi Sasaki,1988年生)作为日本演艺产业代表性人物,其职业生涯从2002年持续至今,跨越模特、演员、歌手等多领域。2002年获全日本国民美少女比赛模特组冠军出道,初期活跃于《Pinky》《Non-no》等时尚杂志。2008年起向影视转型,出演《热血街区》等作品。2010年代拓展至综艺、音乐领域。2020年面对丈夫出轨风波,采取"沉默-原谅-重启"策略修复公众形象。2024年开设个人YouTube频道,开启新媒体转型。

\subsubsubsection{研究价值}
佐佐木希的职业发展路径为理解东亚娱乐产业变迁、女性艺人转型策略及跨媒介形象管理提供了极具价值的案例。其职业生涯跨越日本娱乐产业从传统媒体向新媒体转型的关键期(2005-2025)。作为女性艺人,她面临的性别化挑战具有典型性。2020年个人危机处理案例为研究品牌韧性机制提供了珍贵实证。

\subsubsubsection{研究问题}
本研究旨在通过系统分析佐佐木希近20年职业轨迹,探讨:(1)艺人职业转型期关键节点识别;(2)跨媒介形象构建的动态机制;(3)个人品牌在不同生命周期阶段的价值维持策略。\NSFCBlankPara

\subsubsection{国内外研究现状}

\subsubsubsection{职业转型理论}
目前研究主要集中三个方向:Smith(2020)提出"职业锚理论"\cite{Smith2020};Super(1957)划分职业发展五阶段\cite{Super1957},但缺乏东亚语境研究。

\subsubsubsection{形象管理研究}
Johnson(2019)提出"建构-维护-修复"三阶段\cite{Johnson2019};Goffman(1959)的"自我呈现"理论\cite{Goffman1959},但新媒体环境研究不足。

\subsubsubsection{产业生态分析}
Tanaka(2021)比较日韩艺人培养体系\cite{Tanaka2021};Jin(2016)分析韩国产业全球化\cite{Jin2016},但个体定位策略研究匮乏。\NSFCBlankPara

\subsubsection{现有研究的局限性}

\subsubsubsection{理论层面}
现有研究存在以下不足:(1)缺乏完整职业生命周期的纵向案例研究;(2)对个人品牌韧性机制探讨不足;(3)忽视女性艺人的特殊挑战。

\subsubsubsection{方法层面}
(4)未能充分整合跨学科视角;(5)新媒体环境下艺人-粉丝关系重构研究不足。\NSFCBlankPara

\subsubsection{研究切入点}

\subsubsubsection{案例独特性}
佐佐木希的职业轨迹具有独特研究价值:(1)转型涵盖"模特→演员→歌手→综艺艺人→YouTuber"完整链条;(2)经历2020年个人危机的形象修复过程;(3)职业发展与产业变迁高度同步。

\subsubsubsection{研究代表性}
(4)从地方到全国性的典型路径;(5)女性艺人的性别化挑战具有代表性。\NSFCBlankPara

\endgroup
