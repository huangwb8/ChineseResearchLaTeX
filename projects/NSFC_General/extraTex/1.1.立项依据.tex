\begingroup
\justifying
\setlength{\parindent}{2em}% 首行缩进 2 字符

\subsubsection{研究背景}

\subsubsubsection{案例概述}
刘亦菲(1987年生)作为中国演艺产业代表性人物,其职业生涯从2002年持续至今,跨越童星、电视剧演员、电影演员、国际巨星等多个阶段。2002年出演《金粉世家》进入演艺圈,2005年凭借《仙剑奇侠传》中"赵灵儿"一角成为现象级偶像。2006-2008年主演《神雕侠侣》《天龙八部》等金庸武侠剧,奠定"古装女神"地位。2012年起向电影转型,主演《铜雀台》《露水红颜》等作品。2020年主演迪士尼真人版《花木兰》,成为首位担纲好莱坞A级制作女主角的华人演员,实现国际化突破。

\subsubsubsection{研究价值}
刘亦菲的职业发展路径为理解中国娱乐产业变迁、女性艺人国际化策略及跨文化形象管理提供了极具价值的案例。其职业生涯跨越中国影视产业从电视剧主导向电影主导、从国内市场向国际市场转型的关键期(2005-2025)。作为女性艺人,她在"古装女神"到"国际巨星"的转型中面临的挑战具有典型性。《花木兰》项目为研究华人演员好莱坞突破机制提供了珍贵实证。

\subsubsubsection{研究问题}
本研究旨在通过系统分析刘亦菲近20年职业轨迹,探讨:(1)从童星到成熟演员的转型策略;(2)从国内市场到好莱坞的国际化路径;(3)个人品牌形象在不同文化语境中的构建与维护策略。\NSFCBlankPara

\subsubsection{国内外研究现状}

\subsubsubsection{职业转型理论}
目前研究主要集中三个方向:Smith(2020)提出"职业锚理论"\cite{Smith2020};Super(1957)划分职业发展五阶段\cite{Super1957},但缺乏中国演艺产业语境研究。

\subsubsubsection{国际化路径研究}
Johnson(2019)提出"本土-区域-全球"三阶段国际化模型\cite{Johnson2019};Lee(2018)分析亚洲演员好莱坞突破策略\cite{Lee2018},但对华人女演员的文化适应机制研究不足。

\subsubsubsection{品牌形象管理}
Goffman(1959)的"自我呈现"理论\cite{Goffman1959}为形象管理提供理论基础;Zhang(2021)研究中国明星跨文化形象构建\cite{Zhang2021},但对"古装女神"到"国际巨星"的转型路径研究匮乏。\NSFCBlankPara

\subsubsection{现有研究的局限性}

\subsubsubsection{理论层面}
现有研究存在以下不足:(1)缺乏完整职业生命周期的纵向案例研究;(2)对华人演员好莱坞突破机制探讨不足;(3)忽视女性艺人在跨文化转型中的特殊挑战。

\subsubsubsection{方法层面}
(4)未能充分整合跨学科视角;(5)社交媒体时代艺人-粉丝关系重构研究不足;(6)缺乏对"古装女神"标签化与去标签化的动态分析。\NSFCBlankPara

\subsubsection{研究切入点}

\subsubsubsection{案例独特性}
刘亦菲的职业轨迹具有独特研究价值:(1)转型涵盖"童星→古装女神→电影演员→国际巨星"完整链条;(2)成功突破好莱坞A级制作,成为华人女演员国际化标杆;(3)职业发展与中国影视产业国际化进程高度同步。

\subsubsubsection{研究代表性}
(4)从国内市场到国际市场的典型路径;(5)女性艺人在跨文化语境中的形象管理具有代表性;(6)"古装女神"标签的双刃剑效应为研究类型化与去类型化提供案例。\NSFCBlankPara

\endgroup
