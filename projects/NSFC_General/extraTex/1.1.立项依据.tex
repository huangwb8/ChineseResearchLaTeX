\begingroup
\justifying
\setlength{\parindent}{2em}% 首行缩进 2 字符

\subsubsection{研究背景与科学意义}
% 建议:1–2 段。需求/痛点 → 科学问题缺口 → 本项目意义(科学价值/方法学价值/潜在应用)。
示例:参考文献引用展示 \cite{Smith1900},多项研究 \cite{Smith1900,Piter1992,John1997} 证实了该方向的重要性。

\textbf{【演示内容:基于佐佐木希发展历程的虚拟研究课题】}

本研究以日本演艺产业为研究对象,聚焦艺人职业生涯发展的动态演化机制。艺人职业发展轨迹研究具有重要的科学价值和现实意义。从理论层面看,艺人职业生涯演化涉及多学科交叉,包括社会学、心理学、管理学等领域,现有研究多采用静态描述或经验归纳,缺乏基于大数据的定量分析和动态建模方法 \cite{Smith1900}。从应用层面看,随着文娱产业的快速发展,科学理解艺人职业发展规律,对于人才培养、职业规划、产业决策具有重要指导价值。

本研究选取日本演员佐佐木希(Nozomi Sasaki)为典型个案,通过对其从模特出道(2005年)到演员转型(2008年)再到实力派演员(2020年代)的完整职业轨迹进行纵向追踪,构建"多阶段职业发展演化模型"。该模型将揭示艺人从"颜值驱动"向"实力驱动"转型的关键影响因素、转折节点和演化路径,为理解文化创意产业中人才成长规律提供新的理论视角和方法学支撑。\NSFCBlankPara

\subsubsection{国内外研究现状与不足}
% 建议:2–3 段。现状梳理(可分方向/代表性工作)→ 核心不足(方法/数据/理论/转化/工程)。

\textbf{国内研究现状:}近年来,国内学者在文化产业人才发展研究方面取得了一定进展。主要研究集中在演员职业路径分析 \cite{Piter1992}、明星品牌价值评估等领域。然而,现有研究多采用横截面调查或案例研究方法,缺乏对个体职业生涯的纵向追踪,且研究对象多集中在国内演员,对日本等海外演艺产业的研究相对薄弱。

\textbf{国际研究现状:}国际上,艺人职业发展研究已形成多个研究分支:一是职业生涯阶段理论,将艺人发展划分为"新人期—成长期—成熟期—衰退期"四个阶段 \cite{John1997};二是职业转型机制研究,探讨艺人从模特向演员、从偶像向实力派的转型路径;三是基于社会网络分析的影响力传播研究。日本学者对本国演艺产业有深入调研,但多局限于产业描述,缺乏理论提炼和数学建模。

\textbf{存在的主要不足:}
\begin{enumerate}
  \item \textbf{理论层面:}现有研究多停留在经验总结层面,缺乏系统性的理论框架和数学模型,难以量化描述职业发展的动态演化过程;
  \item \textbf{方法层面:}传统研究多依赖定性分析和小样本调查,缺乏基于大数据和多模态数据的定量研究方法;
  \item \textbf{数据层面:}缺乏高质量的纵向追踪数据集,现有数据分散在不同平台,缺乏标准化整合;
  \item \textbf{案例层面:}对日本演员职业发展的系统性案例研究不足,特别是对"模特→演员"转型成功案例的深入剖析有限。
\end{enumerate}

【示例:图片插入】图 \ref{fig:career-timeline} 展示了佐佐木希职业生涯的关键节点与转型轨迹。

\begin{figure}[!ht]
    \begin{center}
        \includegraphics[width=0.85\linewidth]{figures/zzmx-115.jpg}
        \caption{佐佐木希职业发展轨迹示意图(2005-2025)。该图展示了研究对象从2005年以《PINKY》杂志模特出道,到2008年通过电影《变身西装》转型演员,再到2015年凭借电视剧《朝5晚9》成为实力派演员的完整职业路径。红色节点标注了关键转型期,蓝色曲线表示影响力演化趋势。}
        \label{fig:career-timeline}
    \end{center}
\end{figure}

【示例:表格插入】表 \ref{table1} 展示了临床特征数据的统计分析结果。

\begin{table}[htbp]
    \centering
    \small
    \caption{\textbf{临床特征统计分析表}}
    \begin{tabular}{lllll}
    \toprule
            & \textbf{Characteristics} & \textbf{High (n=137)} & \textbf{Low (n=235)} & \textbf{\textit{P} value} \\
    \midrule
    \textbf{Gender (\%)} & Male  & 85 (62.0) & 154 (65.5) & 0.572 \\
            & Female & 52 (38.0) & 81 (34.5) &  \\
    \textbf{Age (mean (SD))} &       & 65.50 (10.22) & 66.06 (10.95) & 0.628 \\
    \textbf{Race (\%)} & White & 98 (81.0) & 139 (69.5) & 0.054 \\
            & Asian & 21 (17.4) & 51 (25.5) &  \\
            & Other & 2 (1.7) & 10 (5.0) &  \\
    \textbf{Tumor position (\%)} & GEJ   & 17 (12.9) & 27 (11.9) & 0.03 \\
            & Cardia & 11 (8.3) & 36 (15.9) &  \\
            & Fundus & 48 (36.4) & 54 (23.8) &  \\
            & Body  & 7 (5.3) & 23 (10.1) &  \\
            & Antrum & 49 (37.1) & 87 (38.3) &  \\
    \textbf{Pathology (\%)} & Intestinal & 65 (52.4) & 167 (78.8) & <0.001 \\
            & Diffuse & 52 (41.9) & 35 (16.5) &  \\
            & Mixed & 7 (5.6) & 10 (4.7) &  \\
    \textbf{Grade (\%)} & G1    & 1 (0.7) & 9 (4.0) & <0.001 \\
            & G2    & 28 (20.6) & 108 (47.6) &  \\
            & G3    & 107 (78.7) & 110 (48.5) &  \\
    \textbf{T stage (\%)} & T1    & 2 (1.6) & 17 (7.3) & 0.009 \\
            & T2    & 24 (18.6) & 42 (17.9) &  \\
            & T3    & 38 (29.5) & 91 (38.9) &  \\
            & T4    & 65 (50.4) & 84 (35.9) &  \\
    \textbf{N stage (\%)} & N0    & 37 (29.1) & 76 (33.2) & 0.504 \\
            & Np    & 90 (70.9) & 153 (66.8) &  \\
    \textbf{M stage (\%)} & M0    & 121 (92.4) & 206 (93.2) & 0.933 \\
            & M1    & 10 (7.6) & 15 (6.8) &  \\
    \textbf{Stage (\%)} & I     & 10 (8.3) & 37 (17.1) & 0.171 \\
            & II    & 40 (33.3) & 66 (30.6) &  \\
            & III   & 60 (50.0) & 98 (45.4) &  \\
            & IV    & 10 (8.3) & 15 (6.9) &  \\
    \textbf{MSI status (\%)} & MSI-H & 21 (21.0) & 26 (19.0) & 0.001 \\
            & MSI-L & 6 (6.0) & 32 (23.4) &  \\
            & MSS   & 73 (73.0) & 79 (57.7) &  \\
    \textbf{EBV infection (\%)} & Positive & 18 (18.0) & 5 (3.6) & 0.001 \\
            & Negative & 82 (82.0) & 132 (96.4) &  \\
    \textbf{Purity (mean (SD))} &       & 0.40 (0.17) & 0.57 (0.19) & <0.001 \\
    \textbf{Ploidy (mean (SD))} &       & 2.37 (0.67) & 2.69 (0.90) & 0.001 \\
    \bottomrule
    \end{tabular}%
    \label{table1}%
\end{table}%

\subsubsection{拟解决的关键科学问题}
% 建议:2–4 条;每条尽量"可检验/可量化/可证伪",并与后续研究内容一一对应。

基于上述研究现状和不足分析,本项目拟解决以下关键科学问题:

\begin{enumerate}
  \item \itemtitlefont{关键科学问题 1:艺人职业发展的阶段性演化规律是什么?}\par
  该问题涉及职业发展轨迹的量化建模和动态分析。现有研究将职业发展简化为线性过程,但佐佐木希案例显示,艺人职业路径呈现"颜值驱动→转型期→实力驱动"的非线性演化特征。拟通过纵向追踪数据(2005-2025年),构建多阶段演化动力学模型,量化各阶段的持续时间、转换概率和关键转折点。预期成果包括:建立职业发展阶段性理论模型、发表高水平学术论文 2-3 篇。

  \item \itemtitlefont{关键科学问题 2:从模特向演员转型的关键影响因素是什么?}\par
  该问题关注职业转型的驱动机制和成功要素。佐佐木希的成功转型涉及多个维度:个人特质(甜美外形→演技提升)、外部机会(经纪公司支持→优质剧本)、市场需求(时尚界→影视圈)。拟通过多模态数据分析(影视作品评分、社交媒体数据、媒体报道文本),识别转型成功的关键因素及其权重。预期成果包括:构建职业转型影响因素指标体系、开源数据分析工具包 1 个。

  \item \itemtitlefont{关键科学问题 3:如何构建标准化的艺人职业发展数据集?}\par
  该问题面向数据资源建设。现有艺人数据分散在多个平台(电影数据库、社交媒体、新闻媒体),缺乏标准化整合。拟通过制定数据采集标准、建立多源数据融合方法,构建覆盖"个人基本信息—作品数据—评价数据—社会网络"的全方位数据集。预期成果包括:建立日本演员职业发展标准数据集、发表数据论文 1 篇。
\end{enumerate}

\subsubsection{本项目研究思路与预期贡献}
% 建议:1–2 段。总体思路/核心假说 → 预期贡献(理论/方法/数据资源/平台原型/转化前景)。

\textbf{总体研究思路:}本项目采用"典型案例深度剖析—多模态数据融合—理论模型构建—方法工具开发"四位一体的研究思路。核心假说是:艺人职业发展遵循"颜值驱动→转型适应→实力驱动"的三阶段演化规律,成功转型取决于个人特质、外部机会和市场需求的动态匹配。通过佐佐木希案例的纵向追踪(2005-2025年),构建可量化、可预测的职业发展演化模型。

\textbf{预期贡献:}\textit{理论贡献:}建立艺人职业发展多阶段演化理论,填补文化创意产业人才发展研究的理论空白;\textit{方法学贡献:}提出基于多模态数据的职业发展定量分析方法,整合影视评分、社交媒体、新闻报道等多源数据;\textit{数据贡献:}构建日本演员职业发展标准数据集,推动相关领域的数据共享和复用;\textit{平台贡献:}开发艺人职业发展分析工具平台,为产业决策提供数据支撑;\textit{转化前景:}研究成果可为演艺公司的人才发掘、经纪公司的职业规划、投资决策提供科学依据,产生显著的社会和经济价值。\NSFCBlankPara

\endgroup
