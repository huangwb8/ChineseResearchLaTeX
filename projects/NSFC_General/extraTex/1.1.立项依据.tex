\begingroup
\justifying
\setlength{\parindent}{2em}% 首行缩进 2 字符

\subsubsection{研究背景与科学意义}
% 建议:1–2 段。需求/痛点 → 科学问题缺口 → 本项目意义(科学价值/方法学价值/潜在应用)。
\NSFCBlankPara
\NSFCBlankPara

\subsubsection{国内外研究现状与不足}
% 建议:2–3 段。现状梳理(可分方向/代表性工作)→ 核心不足(方法/数据/理论/转化/工程)。
\NSFCBlankPara
\NSFCBlankPara
\NSFCBlankPara

\subsubsection{拟解决的关键科学问题}
% 建议:2–4 条;每条尽量“可检验/可量化/可证伪”,并与后续研究内容一一对应。
\begin{enumerate}
  \item \itemtitlefont{关键科学问题 1:}\par\NSFCBlankPara
  \item \itemtitlefont{关键科学问题 2:}\par\NSFCBlankPara
  \item \itemtitlefont{关键科学问题 3:}\par\NSFCBlankPara
\end{enumerate}

\subsubsection{本项目研究思路与预期贡献}
% 建议:1–2 段。总体思路/核心假说 → 预期贡献(理论/方法/数据资源/平台原型/转化前景)。
\NSFCBlankPara
\NSFCBlankPara

\endgroup
