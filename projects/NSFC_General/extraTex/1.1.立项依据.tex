\begingroup
\justifying
\setlength{\parindent}{2em}% 首行缩进 2 字符

\subsubsection{研究背景与科学意义}
% 建议:1–2 段。需求/痛点 → 科学问题缺口 → 本项目意义(科学价值/方法学价值/潜在应用)。
示例:参考文献引用展示 \cite{Smith1900},多项研究 \cite{Smith1900,Piter1992,John1997} 证实了该方向的重要性。
\NSFCBlankPara

\subsubsection{国内外研究现状与不足}
% 建议:2–3 段。现状梳理(可分方向/代表性工作)→ 核心不足(方法/数据/理论/转化/工程)。

【示例:图片插入】图 \ref{fig:research-status} 展示了研究现状的梳理结果。

\begin{figure}[!th]
    \begin{center}
        \includegraphics[width=0.8\linewidth]{figures/zzmx-115.jpg}
        \caption{研究现状示意图。\\
        \raggedright \justifying \noindent
        本图展示了该领域的发展历程和主要研究方向。从图中可以看出,现有研究主要集中在方法A和方法B两个方向,但都存在一定局限性。}
        \label{fig:research-status}
    \end{center}
\end{figure}

\clearpage

【示例:表格插入】表 \ref{tab:method-comparison} 对比了不同方法的优缺点。

\begin{table}[htbp]
    \centering
    \caption{\textbf{研究现状对比表}}
    \begin{tabular}{llll}
    \toprule
    方法 & 优势 & 不足 & 适用场景 \\
    \midrule
    方法A & 计算效率高 & 精度有限 & 大规模数据处理 \\
    方法B & 精度高 & 计算复杂度高 & 高精度要求场景 \\
    方法C & 通用性强 & 特定场景性能一般 & 多种应用场景 \\
    \bottomrule
    \end{tabular}
    \label{tab:method-comparison}
\end{table}

\subsubsection{拟解决的关键科学问题}
% 建议:2–4 条;每条尽量“可检验/可量化/可证伪”,并与后续研究内容一一对应。
\begin{enumerate}
  \item \itemtitlefont{关键科学问题 1:}\par\NSFCBlankPara
  \item \itemtitlefont{关键科学问题 2:}\par\NSFCBlankPara
  \item \itemtitlefont{关键科学问题 3:}\par\NSFCBlankPara
\end{enumerate}

\subsubsection{本项目研究思路与预期贡献}
% 建议:1–2 段。总体思路/核心假说 → 预期贡献(理论/方法/数据资源/平台原型/转化前景)。
\NSFCBlankPara
\NSFCBlankPara

\endgroup
