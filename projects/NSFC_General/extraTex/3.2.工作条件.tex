\begingroup
\justifying
\setlength{\parindent}{2em}% 首行缩进 2 字符

\subsubsection{已具备的实验条件}

\subsubsubsection{硬件设施}
\indent 依托单位文化产业研究中心配备完善科研设施:(1)高性能计算集群:NVIDIA A100 GPU 4张(每张40GB显存),总内存512GB,支持大规模跨语言模型训练;(2)大容量数据存储系统:总容量500TB,支持多用户并发访问和自动备份;(3)专业级影像处理工作站:高色域显示器(99\% Adobe RGB)和专业级图形卡;(4)专业录音与视频编辑设备。

\subsubsubsection{软件资源}
\indent 实验室部署完整跨语言数据分析工具链:(1)自然语言处理:NLTK、spaCy、jieba、HanLP、多语言BERT等;(2)深度学习框架:TensorFlow、PyTorch、Keras,支持多语言预训练模型;(3)计算机视觉:OpenCV、PIL、scikit-image、detectron2;(4)数据可视化:Tableau、D3.js、Matplotlib、Seaborn;(5)社会科学统计:SPSS、Stata、R;(6)项目管理:Git、GitHub、Slack、Trello。

\subsubsubsection{数据库资源}
\indent 依托单位图书馆购买丰富学术数据库:(1)外文数据库:Web of Science、Scopus、JSTOR、SpringerLink、ScienceDirect、EBSCO、SAGE、Wiley等;(2)中文数据库:中国知网(CNKI)、万方数据、维普资讯等;(3)专业数据库:IMDb、Box Office Mojo、豆瓣电影等;(4)统计年鉴:中国统计年鉴、中国文化及相关产业统计年鉴等。\NSFCBlankPara

\subsubsection{尚缺少的实验条件及拟解决途径}

\subsubsubsection{缺少的实验条件}
\indent (1)好莱坞内部资料和北美社交媒体数据因地域限制和平台政策无法完整获取;(2)迪士尼等制片公司一手资料属商业机密;(3)欧美娱乐产业数据资源分散;(4)大规模跨语言视频数据存储与处理需更专业设备;(5)中英文跨文化情感分析缺乏高质量标注语料。

\subsubsubsection{拟解决途径}
\indent (1)国际合作:与USC安纳伯格传播学院、UCLA亚太研究中心、哥伦比亚大学东亚研究所、北京电影学院、中国传媒大学等建立合作关系;(2)数据采购:通过Brandwatch、Talkwalker、Cision等获取社交媒体历史数据(预算3-5万元/年);(3)公开数据集:利用Twitter Academic Research Product Track、IMDb数据集等;(4)自主研发:已开发跨语言网络爬虫工具和多语言情感分析工具原型;(5)语料库建设:与合作机构联合建设中英文跨文化情感标注语料库(约10万条);(6)设备升级:申请经费升级视频存储和处理设备(预算5-8万元)。\NSFCBlankPara

\subsubsection{利用国家重点实验室等研究基地的计划}

\subsubsubsection{合作单位}
\indent 申请团队计划与国内高水平研究基地建立紧密合作关系:中国人民大学新闻学院社会发展研究基地、清华大学新闻与传播学院新媒体研究中心、北京电影学院电影学系、中国传媒大学文化产业管理学院、文化和旅游部文化产业研究中心、国家广播电视总局发展研究中心。

\subsubsubsection{合作方式}
\indent 学者互访(每年邀请合作基地知名学者来校讲学1-2次,项目负责人赴合作基地交流访问1-2次);联合研究(在华人演员国际化等重大理论问题上开展联合研究,共同发表高水平论文);资源共享(共享部分研究数据和设备资源);人才培养(共同指导研究生,开展学生访学交流)。\NSFCBlankPara

\endgroup
