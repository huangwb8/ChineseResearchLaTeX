\begingroup
\justifying
\setlength{\parindent}{2em}% 首行缩进 2 字符

\subsubsection{研究方法}
% 写清:总体方法框架、关键步骤、对照/消融、统计学/计算验证策略。
\NSFCBlankPara
\NSFCBlankPara

\subsubsection{技术路线}
% 建议:用 1 段文字 +(可选)流程图;强调“输入→处理→输出”的闭环。
\NSFCBlankPara
\NSFCBlankPara

\subsubsection{关键技术与关键环节}
% 建议:列出 2–4 个“关键点”,说明为什么关键、怎么攻关、预期达到什么水平。
\begin{enumerate}
  \item \itemtitlefont{关键技术 1:}\par\NSFCBlankPara
  \item \itemtitlefont{关键技术 2:}\par\NSFCBlankPara
  \item \itemtitlefont{关键技术 3:}\par\NSFCBlankPara
\end{enumerate}

\subsubsection{可行性分析与风险对策}
% 建议:按“理论/技术/团队/条件”四维写可行性;并给出 2–3 条风险与备选方案。
\begin{enumerate}
  \item \itemtitlefont{可行性:}\par\NSFCBlankPara\NSFCBlankPara
  \item \itemtitlefont{主要风险与对策:}\par\NSFCBlankPara\NSFCBlankPara
\end{enumerate}

\endgroup
