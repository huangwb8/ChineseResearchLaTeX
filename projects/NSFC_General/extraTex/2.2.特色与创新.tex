\begingroup
\justifying
\setlength{\parindent}{2em}% 首行缩进 2 字符

\subsubsection{学术思想的创新}

\subsubsubsection{理论模型创新}
\indent 本项目核心创新在于提出"华人演员国际化三阶段动态模型"(Local Accumulation-Regional Expansion-Global Breakthrough),突破传统理论对单向线性路径的假设,强调本土积累与国际突破的辩证关系。

\subsubsubsection{概念引入创新}
\indent 首次将"跨文化形象一致性"概念引入华人演员国际化研究,提出"标签化-去标签化动态平衡"理论框架:(1)标签资产化机制-将"古装女神"转化为职业资本;(2)标签突破机制-通过类型多元化实现去标签化;(3)文化适应机制-在保持东方特质的同时融入国际语境。\NSFCBlankPara

\subsubsection{研究方法的创新}

\subsubsubsection{多模态数据融合分析}
\indent 首次将中英文文本、图像、视频、票房数据整合于统一框架,开发基于注意力机制的跨语言多模态特征融合算法,实现对华人演员国际形象的全方位量化评估。

\subsubsubsection{跨文化比较分析}
\indent 将生命历程研究方法与跨文化比较技术结合,提出"国际化转型状态转移矩阵"概念,识别从"本土明星"到"国际巨星"的关键跃迁点。

\subsubsubsection{混合研究方法}
\indent 采用"定量引导-定性深化"设计,开发"华人演员国际化轨迹编码手册",结合社交媒体情感分析与深度访谈,构建立体化研究框架。\NSFCBlankPara

\subsubsection{研究视角的创新}

\subsubsubsection{跨学科整合}
\indent 本项目独特之处在于跨学科研究视角,整合传播学、管理学、文化研究、计算机科学理论方法,构建华人演员国际化的多维分析框架。

\subsubsubsection{性别与文化视角}
\indent 关注女性艺人在跨文化转型中的特殊挑战,对比中西方文化语境下演员发展模式差异,将个体置于全球娱乐产业生态系统中考察,揭示"东方古典美"标签在国际化进程中的双刃剑效应。\NSFCBlankPara

\subsubsection{应用价值的创新}

\subsubsubsection{产业应用}
\indent 研究成果将为经纪公司提供"华人演员国际化路径规划决策支持工具",为演员本人提供"跨文化个人品牌管理指南",为影视制作公司提供"国际化项目选角策略参考"。

\subsubsubsection{政策支持}
\indent 为政府文化部门制定中国影视产业国际化政策提供实证依据,为"文化走出去"战略提供微观层面的实践路径参考。\NSFCBlankPara

\endgroup
