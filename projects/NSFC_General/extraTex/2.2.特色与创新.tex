\justifying
\NSFCBodyText


\subsubsection{理论模型创新}
\indent 本项目核心创新在于提出"华人演员国际化三阶段动态模型"(Local Accumulation-Regional Expansion-Global Breakthrough),突破传统理论对单向线性路径的假设,强调本土积累与国际突破的辩证关系。

\subsubsection{研究方法的创新}

\subsubsubsection{多模态数据融合分析}
\indent 首次将中英文文本、图像、视频、票房数据整合于统一框架,开发基于注意力机制的跨语言多模态特征融合算法,实现对华人演员国际形象的全方位量化评估。

\subsubsubsection{跨文化比较分析}
\indent 将生命历程研究方法与跨文化比较技术结合,提出"国际化转型状态转移矩阵"概念,识别从"本土明星"到"国际巨星"的关键跃迁点。

\subsubsubsection{混合研究方法}

\subsubsection{研究视角的创新}

\subsubsubsection{跨学科整合}
\indent 本项目独特之处在于跨学科研究视角,整合传播学、管理学、文化研究、计算机科学理论方法,构建华人演员国际化的多维分析框架。

\subsubsubsection{性别与文化视角}

\subsubsection{应用价值的创新}

\subsubsubsection{产业应用}
\indent 研究成果将为经纪公司提供"华人演员国际化路径规划决策支持工具",为演员本人提供"跨文化个人品牌管理指南",为影视制作公司提供"国际化项目选角策略参考"。

\subsubsubsection{政策支持}

