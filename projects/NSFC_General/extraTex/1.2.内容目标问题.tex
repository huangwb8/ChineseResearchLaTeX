\begingroup
\justifying
\setlength{\parindent}{2em}% 首行缩进 2 字符

\subsubsection{研究内容}
% 建议:拆成 3–4 条"研究内容",每条都写清:任务→方法→数据/实验→输出/里程碑。

\textbf{【演示内容:基于佐佐木希案例的虚拟研究课题】}

\begin{enumerate}
  \item \itemtitlefont{研究内容 1:佐佐木希职业生涯纵向追踪与阶段划分}\par
  \textbf{任务:}系统梳理佐佐木希从2005年模特出道到2025年的完整职业轨迹,识别职业发展的关键节点和阶段特征。\par
  \textbf{方法:}采用历史文献分析法、纵向案例研究法和时间序列分析法,系统收集其影视作品、媒体报道、社交媒体数据。\par
  \textbf{数据/实验:}收集2005-2025年间的影视作品数据库(日本电影数据库、电视剧评分网站)、社交媒体数据(Twitter/X、Instagram)、媒体报道(新闻文本、杂志采访),共计约5000条数据记录。通过时间序列分析识别职业发展的三个阶段:颜值驱动期(2005-2008)、转型适应期(2008-2015)、实力驱动期(2015-2025)。\par
  \textbf{输出/里程碑:}第1年末完成纵向追踪数据集构建和阶段划分模型,发表案例研究论文1篇。\NSFCBlankPara

  \item \itemtitlefont{研究内容 2:职业转型影响因素的多模态数据分析}\par
  \textbf{任务:}识别并量化从模特向演员转型的关键影响因素,构建职业转型成功要素指标体系。\par
  \textbf{方法:}采用多模态数据融合、机器学习(随机森林、梯度提升树)和文本挖掘(情感分析、主题模型)方法。\par
  \textbf{数据/实验:}整合影视作品评分(豆瓣、IMDb)、社交媒体互动数据(点赞数、转发数、评论情感)、媒体报道主题分布。通过特征工程提取30个候选影响因素,使用机器学习模型识别关键特征并计算权重。\par
  \textbf{输出/里程碑:}第2年中完成影响因素指标体系构建,开发开源数据分析工具包1个,发表方法学论文1-2篇。\NSFCBlankPara

  \item \itemtitlefont{研究内容 3:职业发展演化动力学模型构建与验证}\par
  \textbf{任务:}基于佐佐木希案例,构建艺人职业发展的三阶段演化动力学模型,并通过其他日本演员案例进行验证。\par
  \textbf{方法:}采用微分方程建模、系统动力学和元分析相结合的方法。\par
  \textbf{数据/实验:}构建"影响力—时间"微分方程模型,刻画颜值驱动期和实力驱动期的不同增长曲线。收集20位日本演员的辅助案例数据进行模型验证,计算模型的预测准确率和泛化能力。\par
  \textbf{输出/里程碑:}第3年末完成理论模型构建和验证,发表理论论文1-2篇,申请软件著作权1项。\NSFCBlankPara

  \item \itemtitlefont{研究内容 4:日本演员职业发展标准数据集构建}\par
  \textbf{任务:}制定数据采集标准,整合多源数据,构建可复用的日本演员职业发展标准数据集。\par
  \textbf{方法:}采用数据标准化、实体对齐和多源数据融合技术。\par
  \textbf{数据/实验:}整合日本电影数据库、电视剧数据库、社交媒体API、新闻媒体档案等6个数据源。制定数据标注规范(作品类型、角色重要性、影响力指标等),构建包含50位演员、覆盖1970-2025年的标准化数据集。\par
  \textbf{输出/里程碑:}第3年末完成数据集构建并公开发布,发表数据论文1篇。\NSFCBlankPara
\end{enumerate}

【示例:图片插入】图 \ref{fig:framework-design} 展示了职业发展演化模型的设计思路。

\begin{figure}[!ht]
    \begin{center}
        \includegraphics[width=0.75\linewidth]{figures/zzmx-mobile-105.jpg}
        \caption{职业发展演化动力学模型示意图。本图展示了研究内容3中三阶段演化模型的构建思路,包括颜值驱动期、转型适应期、实力驱动期的动力学方程、关键状态变量(影响力、作品数量、观众认可度)以及阶段转换的临界条件。红色箭头表示阶段转换方向,蓝色曲线表示影响力演化路径。}
        \label{fig:framework-design}
    \end{center}
\end{figure}

\subsubsection{研究目标与可验证指标}
% 建议:目标要"可度量",指标要"可验证",并与研究内容对应。

\textbf{【示例内容】}

\begin{enumerate}
  \item \itemtitlefont{总体目标:}\par
  建立全新的理论体系,突破现有方法的性能瓶颈,构建可复用的研究平台,推动该领域的科学进步和技术发展。

  \item \itemtitlefont{关键指标:}\par
  \textbf{理论层面:}建立完整理论框架,关键理论指标优于现有方法30\%以上;发表高水平SCI论文3-5篇,其中TOP期刊论文不少于2篇。\par
  \textbf{方法层面:}算法计算效率提升50\%以上,精度保持或优于现有方法;开源工具包获得社区广泛使用。\par
  \textbf{应用层面:}成果在2-3个实际场景中得到验证和应用;申请发明专利2-3项,软件著作权1-2项。\NSFCBlankPara

  \item \itemtitlefont{预期产出:}\par
  \textbf{学术论文:}SCI论文3-5篇,其中国际TOP期刊2-3篇。\par
  \textbf{知识产权:}发明专利2-3项,软件著作权1-2项。\par
  \textbf{开源工具:}算法工具包1个,数据分析平台1个,标准数据集1-2个。\par
  \textbf{人才培养:}培养博士研究生1-2名,硕士研究生2-3名。\par
  \textbf{学术交流:}参加国际学术会议2-3次,做特邀报告1-2次。
\end{enumerate}

\subsubsection{拟解决的关键问题(与立项依据保持一致)}

\textbf{【示例内容】}
本项目拟解决的关键问题与立项依据中提出的问题一一对应:

\begin{enumerate}
  \item \itemtitlefont{问题 1:如何建立高精度、高效率的理论模型?}\par
  针对该问题,本项目将引入创新的数学建模方法,突破传统理论模型的假设限制。具体而言,将采用XXX理论作为基础,结合YYY方法,构建能够准确描述复杂系统行为的理论框架。该问题的解决将为后续算法设计和应用验证奠定坚实的理论基础。\NSFCBlankPara

  \item \itemtitlefont{问题 2:如何设计兼顾效率和精度的算法框架?}\par
  针对该问题,本项目将设计全新的算法架构,通过优化数据结构和计算流程,实现效率和精度的双重提升。具体策略包括:采用并行计算技术提升效率,引入自适应机制保持精度,利用GPU加速等方法突破性能瓶颈。\NSFCBlankPara

  \item \itemtitlefont{问题 3:如何构建标准化、可复用的数据资源?}\par
  针对该问题,本项目将制定数据标准化规范,建立数据质量评估体系,构建可复用的数据共享平台。具体措施包括:制定数据采集和预处理标准,建立数据标注和质量控制流程,开发数据管理和共享平台。\NSFCBlankPara
\end{enumerate}

\endgroup
