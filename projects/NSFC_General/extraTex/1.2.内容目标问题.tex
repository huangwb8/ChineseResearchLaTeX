\begingroup
\justifying
\setlength{\parindent}{2em}% 首行缩进 2 字符

\subsubsection{研究内容}
% 建议:拆成 3–4 条"研究内容",每条都写清:任务→方法→数据/实验→输出/里程碑。

\textbf{【示例内容】}

\begin{enumerate}
  \item \itemtitlefont{研究内容 1:理论框架创新与模型构建}\par
  \textbf{任务:}分析现有理论模型的局限性,提出创新性的建模方法,建立完整的理论框架。\par
  \textbf{方法:}采用数学建模、理论分析和数值仿真相结合的方法。\par
  \textbf{数据/实验:}收集基准数据集,设计理论验证实验,包括边界条件测试和敏感性分析(如图 \ref{fig:framework-design} 所示)。\par
  \textbf{输出/里程碑:}第1年末完成理论框架搭建,发表学术论文1-2篇。\NSFCBlankPara

  \item \itemtitlefont{研究内容 2:高效算法设计与实现}\par
  \textbf{任务:}基于创新理论框架,设计高精度、高效率的算法。\par
  \textbf{方法:}采用算法优化、并行计算和工程化实现相结合的策略。\par
  \textbf{数据/实验:}构建标准测试数据集,设计对比实验,与现有方法进行全面的性能评估。\par
  \textbf{输出/里程碑:}第2年末完成算法设计和优化,开源算法工具包1个。\NSFCBlankPara

  \item \itemtitlefont{研究内容 3:应用验证与平台构建}\par
  \textbf{任务:}将研究成果应用于实际问题,构建开源工具平台。\par
  \textbf{方法:}采用原型开发、用户测试和迭代优化相结合的方式。\par
  \textbf{数据/实验:}收集实际应用场景的数据,进行实地测试和用户反馈收集。\par
  \textbf{输出/里程碑:}第3年末完成平台开发和部署,发表应用论文1-2篇。\NSFCBlankPara
\end{enumerate}

【示例:图片插入】图 \ref{fig:framework-design} 展示了理论框架的设计思路。

\begin{figure}[!ht]
    \begin{center}
        \includegraphics[width=0.7\linewidth]{figures/zzmx-mobile-105.jpg}
        \caption{理论框架设计示意图。本图展示了研究内容1中理论框架的构建思路,包括模型层次结构、关键模块划分以及数据流向。}
        \label{fig:framework-design}
    \end{center}
\end{figure}

\subsubsection{研究目标与可验证指标}
% 建议:目标要"可度量",指标要"可验证",并与研究内容对应。

\textbf{【示例内容】}

\begin{enumerate}
  \item \itemtitlefont{总体目标:}\par
  建立全新的理论体系,突破现有方法的性能瓶颈,构建可复用的研究平台,推动该领域的科学进步和技术发展。

  \item \itemtitlefont{关键指标:}\par
  \textbf{理论层面:}建立完整理论框架,关键理论指标优于现有方法30\%以上;发表高水平SCI论文3-5篇,其中TOP期刊论文不少于2篇。\par
  \textbf{方法层面:}算法计算效率提升50\%以上,精度保持或优于现有方法;开源工具包获得社区广泛使用。\par
  \textbf{应用层面:}成果在2-3个实际场景中得到验证和应用;申请发明专利2-3项,软件著作权1-2项。\NSFCBlankPara

  \item \itemtitlefont{预期产出:}\par
  \textbf{学术论文:}SCI论文3-5篇,其中国际TOP期刊2-3篇。\par
  \textbf{知识产权:}发明专利2-3项,软件著作权1-2项。\par
  \textbf{开源工具:}算法工具包1个,数据分析平台1个,标准数据集1-2个。\par
  \textbf{人才培养:}培养博士研究生1-2名,硕士研究生2-3名。\par
  \textbf{学术交流:}参加国际学术会议2-3次,做特邀报告1-2次。
\end{enumerate}

\subsubsection{拟解决的关键问题(与立项依据保持一致)}

\textbf{【示例内容】}
本项目拟解决的关键问题与立项依据中提出的问题一一对应:

\begin{enumerate}
  \item \itemtitlefont{问题 1:如何建立高精度、高效率的理论模型?}\par
  针对该问题,本项目将引入创新的数学建模方法,突破传统理论模型的假设限制。具体而言,将采用XXX理论作为基础,结合YYY方法,构建能够准确描述复杂系统行为的理论框架。该问题的解决将为后续算法设计和应用验证奠定坚实的理论基础。\NSFCBlankPara

  \item \itemtitlefont{问题 2:如何设计兼顾效率和精度的算法框架?}\par
  针对该问题,本项目将设计全新的算法架构,通过优化数据结构和计算流程,实现效率和精度的双重提升。具体策略包括:采用并行计算技术提升效率,引入自适应机制保持精度,利用GPU加速等方法突破性能瓶颈。\NSFCBlankPara

  \item \itemtitlefont{问题 3:如何构建标准化、可复用的数据资源?}\par
  针对该问题,本项目将制定数据标准化规范,建立数据质量评估体系,构建可复用的数据共享平台。具体措施包括:制定数据采集和预处理标准,建立数据标注和质量控制流程,开发数据管理和共享平台。\NSFCBlankPara
\end{enumerate}

\endgroup
