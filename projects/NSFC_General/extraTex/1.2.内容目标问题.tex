\begingroup
\justifying
\setlength{\parindent}{2em}% 首行缩进 2 字符

\subsubsection{研究内容}
% 建议:拆成 3–4 条“研究内容”,每条都写清:任务→方法→数据/实验→输出/里程碑。
\begin{enumerate}
  \item \itemtitlefont{研究内容 1:}\par\NSFCBlankPara\NSFCBlankPara
  \item \itemtitlefont{研究内容 2:}\par\NSFCBlankPara\NSFCBlankPara
  \item \itemtitlefont{研究内容 3:}\par\NSFCBlankPara\NSFCBlankPara
\end{enumerate}

\subsubsection{研究目标与可验证指标}
% 建议:目标要“可度量”,指标要“可验证”,并与研究内容对应。
\begin{enumerate}
  \item \itemtitlefont{总体目标:}\par\NSFCBlankPara
  \item \itemtitlefont{关键指标:}\par\NSFCBlankPara\NSFCBlankPara
  \item \itemtitlefont{预期产出:}\par\NSFCBlankPara
\end{enumerate}

\subsubsection{拟解决的关键问题(与立项依据保持一致)}
\begin{enumerate}
  \item \itemtitlefont{问题 1:}\par\NSFCBlankPara
  \item \itemtitlefont{问题 2:}\par\NSFCBlankPara
  \item \itemtitlefont{问题 3:}\par\NSFCBlankPara
\end{enumerate}

\endgroup
