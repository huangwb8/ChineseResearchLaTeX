\begingroup
\justifying
\setlength{\parindent}{2em}% 首行缩进 2 字符

\subsubsection{研究方法}
% 写清:总体方法框架、关键步骤、对照/消融、统计学/计算验证策略。
\NSFCBlankPara
\NSFCBlankPara

\subsubsection{技术路线}
% 建议:用 1 段文字 +(可选)流程图;强调“输入→处理→输出”的闭环。
\NSFCBlankPara
\NSFCBlankPara

\subsubsection{关键技术与关键环节}
% 建议:列出 2–4 个"关键点",说明为什么关键、怎么攻关、预期达到什么水平。
\begin{enumerate}
  \item \itemtitlefont{关键技术 1:}\par\NSFCBlankPara
  \item \itemtitlefont{关键技术 2:}\par\NSFCBlankPara
  \item \itemtitlefont{关键技术 3:}\par\NSFCBlankPara
\end{enumerate}

\clearpage

【示例:代码插入】以下展示关键算法的实现代码(引用 \texttt{code/test.sh}):

\subsubsection{关键算法实现}
\lstinputlisting[style=codestyle01, language=Bash]{code/test.sh}

\endgroup
