\begingroup
\justifying
\setlength{\parindent}{2em}% 首行缩进 2 字符

\subsubsection{研究方法}
% 写清:总体方法框架、关键步骤、对照/消融、统计学/计算验证策略。

\textbf{【演示内容】}

本研究采用"纵向案例研究+多模态数据分析+数学建模+计算验证"的综合方法框架。

\textbf{(1)纵向案例研究方法:}对佐佐木希的职业生涯进行20年纵向追踪(2005-2025年),采用历史文献分析法系统收集影视作品记录、媒体报道档案、社交媒体数据,构建时间序列数据集。通过关键事件分析法识别职业发展的转折节点(如2008年首次主演电影、2015年《朝5晚9》热播)。

\textbf{(2)多模态数据融合方法:}整合结构化数据(影视作品数据库、评分数据)和非结构化数据(社交媒体文本、新闻报道),采用自然语言处理(NLP)技术进行文本挖掘和情感分析。通过多模态特征融合技术,构建包含个人特质、作品表现、社会网络、市场反馈的综合特征空间。

\textbf{(3)数学建模方法:}基于系统动力学理论,构建职业发展演化的微分方程模型。设影响力指数 $I(t)$ 为状态变量,建立三阶段动力学方程:
\begin{itemize}
  \item \textit{颜值驱动期(2005-2008):} $\frac{dI}{dt} = \alpha_1 A(t) - \beta_1 I(t)$,其中 $A(t)$ 为外貌优势指标
  \item \textit{转型适应期(2008-2015):} $\frac{dI}{dt} = \alpha_2 S(t) + \alpha_3 A(t) - \beta_2 I(t)$,引入演技提升 $S(t)$
  \item \textit{实力驱动期(2015-2025):} $\frac{dI}{dt} = \alpha_4 S(t) + \alpha_5 C(t) - \beta_3 I(t)$,$C(t)$ 为观众认可度
\end{itemize}

\textbf{(4)对照验证策略:}收集20位同期日本演员的对比案例,其中10位成功转型(模特→演员),10位转型失败或未转型。通过元分析方法,验证佐佐木希案例的普适性和特殊性。采用k-fold交叉验证评估模型的预测准确率,目标达到 $R^2 > 0.75$。\NSFCBlankPara
\NSFCBlankPara

\subsubsection{技术路线}
% 建议:用 1 段文字 +(可选)流程图;强调"输入→处理→输出"的闭环。

本项目的技术路线遵循"数据采集→特征提取→模型构建→验证应用"的闭环流程:

\textbf{阶段1(数据采集):}通过API接口爬取日本电影数据库、豆瓣电影、Twitter/X、Instagram等平台数据,清洗并整合为标准数据集。对非结构化文本(媒体报道、社交媒体评论)进行分词、去停用词、实体识别等预处理。

\textbf{阶段2(特征提取):}从多模态数据中提取特征向量,包括:(1)作品特征:作品数量、类型分布、角色重要性、评分变化;(2)个人特征:年龄、形象变化、技能提升;(3)社会网络特征:合作演员、导演、经纪公司变化;(4)市场反馈特征:评分趋势、搜索量、社交媒体互动量。

\textbf{阶段3(模型构建):}首先基于时间序列分析划分职业发展阶段,然后采用随机森林算法识别关键影响因素,最后构建系统动力学微分方程模型刻画演化规律。

\textbf{阶段4(验证应用):}使用对比案例验证模型泛化能力,开发可视化分析平台,为演艺产业决策提供支持。\NSFCBlankPara
\NSFCBlankPara

\subsubsection{关键技术与关键环节}
% 建议:列出 2–4 个"关键点",说明为什么关键、怎么攻关、预期达到什么水平。
\begin{enumerate}
  \item \itemtitlefont{关键技术 1:跨语言多源数据融合技术}\par
  该技术是本研究的基础,涉及日语、中文、英语多语言数据的整合。关键技术难点包括:(1)实体对齐:同一演员在不同数据库中的名称变体识别(如"佐佐木希"、"Nozomi Sasaki"、"佐々木希");(2)数据标准化:不同平台评分体系的归一化处理;(3)时间对齐:多源数据的时间戳统一和缺失值插补。拟采用基于深度学习的实体识别模型(BERT多语言预训练模型)和图神经网络(GNN)进行数据融合,预期实体识别准确率 $>95\%$。\NSFCBlankPara

  \item \itemtitlefont{关键技术 2:职业发展阶段识别算法}\par
  该技术用于从时间序列数据中自动识别职业发展的阶段转换点。传统方法依赖人工标注,效率低且主观性强。拟基于变化点检测(Change Point Detection)算法,结合贝叶斯推断和隐马尔可夫模型(HMM),自动识别佐佐木希职业轨迹中的关键转换节点。该技术的创新在于将影响力指数、作品质量、观众认可度等多维时间序列融合,提升阶段识别的鲁棒性。预期阶段识别准确率 $>90\%$,与专家人工标注的Kappa一致性系数 $>0.85$。\NSFCBlankPara

  \item \itemtitlefont{关键技术 3:系统动力学模型参数估计}\par
  该技术用于估计三阶段演化动力学模型中的参数($\alpha_i, \beta_i$),涉及微分方程的数值求解和参数拟合。关键技术难点包括:(1)模型辨识:从离散的时间序列数据中估计连续微分方程的参数;(2)约束优化:参数需满足经济学和社会学的合理性约束(如增长率非负、衰减率有界)。拟采用最大似然估计(MLE)结合马尔可夫链蒙特卡洛(MCMC)采样方法,进行贝叶斯推断。预期参数估计的置信区间覆盖率 $>90\%$,模型预测的均方根误差(RMSE)相比基准模型降低30\%以上。\NSFCBlankPara
\end{enumerate}

【示例:代码插入】以下展示关键算法的实现代码(引用 \texttt{code/test.sh}):

\subsubsection{关键算法实现}
\begin{figure}[!ht]
\centering
\lstinputlisting[style=codestyle01, language=Bash, basicstyle=\tiny\ttfamily]{code/test.sh}
\caption{关键算法实现代码示例}
\end{figure}

\endgroup
