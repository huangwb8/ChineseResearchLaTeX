\begingroup
\justifying
\setlength{\parindent}{2em}% 首行缩进 2 字符

\subsubsection{第一年研究计划}

\subsubsubsection{研究重点}
\indent 数据收集与预处理、国际化轨迹初步建模。

\subsubsubsection{主要工作}
\indent 完成刘亦菲职业发展档案数据库建立(中英文媒体报道、影视作品、商业代言、票房数据、社交媒体舆情数据),开发跨语言数据清洗与预处理流程。完成相关研究文献综述,构建"华人演员国际化三阶段动态模型"理论框架,开发"华人演员国际化轨迹编码手册"初稿。

\subsubsubsection{年度预期成果}
\indent 建立完整跨语言研究数据库;完成文献综述论文1-2篇;构建理论模型框架。\NSFCBlankPara

\subsubsection{第二年研究计划}

\subsubsubsection{研究重点}
\indent 深度数据分析、跨文化模型构建与验证。

\subsubsubsection{主要工作}
\indent 应用序列分析方法识别国际化发展关键转型节点与典型路径,构建"国际化转型状态转移矩阵",开发"国际化路径预测模型"原型。采用深度学习方法对中英文影像资料进行特征提取,运用跨语言NLP技术分析文本资料中的形象建构策略,构建"跨文化形象一致性测量指标体系"。重点分析《花木兰》项目的跨文化传播效果。

\subsubsubsection{年度预期成果}
\indent 完成国际化轨迹建模分析;发表实证研究论文2-3篇;开发模型原型系统。\NSFCBlankPara

\subsubsection{第三年研究计划}

\subsubsubsection{研究重点}
\indent 模型验证、案例比较、成果总结与转化。

\subsubsubsection{主要工作}
\indent 通过交叉验证和敏感性分析评估模型稳健性,选取同时期其他华人女演员(章子怡、巩俐、杨紫琼等)作为对照案例检验外部效度。撰写研究总报告,开发"华人演员国际化路径规划决策支持工具"原型,组织学术研讨会,撰写专著《华人演员国际化的动态机制:基于刘亦菲案例的纵向研究》。

\subsubsubsection{年度预期成果}
\indent 完成模型验证与优化;发表高水平研究论文3-4篇;完成研究总报告;开发决策支持工具原型。\NSFCBlankPara

\subsubsection{预期研究结果}

\subsubsubsection{理论贡献}
\indent 提出"华人演员国际化三阶段动态模型";建立"跨文化形象一致性"理论框架;构建"标签化-去标签化动态平衡测量指标体系";开发"华人演员国际化轨迹编码手册";提出跨语言多模态数据融合分析框架。

\subsubsubsection{实践价值}
\indent 为经纪公司提供"华人演员国际化路径规划决策支持工具";为演员本人提供"跨文化个人品牌管理指南";为政府文化部门制定中国影视产业国际化政策提供实证依据。

\subsubsubsection{学术产出}
\indent 计划发表高水平学术论文6-8篇(SSCI/CSSCI不少于4篇),目标期刊包括International Journal of Cultural Studies、Asian Journal of Communication、《新闻与传播研究》《现代传播》等;撰写专著1部;参加国际学术会议2-3次;组织专题学术研讨会1-2次。\NSFCBlankPara

\endgroup
