\begingroup
\justifying
\setlength{\parindent}{2em}% 首行缩进 2 字符

\subsubsection{第一年研究计划}

\textbf{【示例内容】}

\textbf{研究重点:数据收集与预处理、职业轨迹初步建模}

\begin{enumerate}
  \item \itemtitlefont{数据收集阶段(第1-6个月):}\par
  (1)建立佐佐木希职业发展档案数据库,收集2002-2025年的媒体报道、影视作品目录、商业代言记录、杂志封面数据等。\par
  (2)收集2020年出轨风波期间的社交媒体舆情数据(Twitter、Instagram、5ch等),建立时间序列数据集。\par
  (3)开发数据清洗与预处理流程,实现多源异构数据的标准化存储。\NSFCBlankPara

  \item \itemtitlefont{文献综述与理论构建(第7-12个月):}\par
  (1)完成国内外相关研究的系统文献综述,撰写文献综述报告。\par
  (2)构建"艺人职业发展三阶段动态模型"的理论框架,撰写理论模型论文。\par
  (3)开发"艺人职业轨迹编码手册"(Career Trajectory Coding Manual)初稿。\NSFCBlankPara
\end{enumerate}

\textbf{年度预期成果:}建立完整的研究数据库;完成文献综述论文1-2篇;构建理论模型框架。

\subsubsection{第二年研究计划}

\textbf{研究重点:深度数据分析、模型构建与验证}

\begin{enumerate}
  \item \itemtitlefont{职业轨迹建模阶段(第13-18个月):}\par
  (1)应用序列分析方法(Sequence Analysis)识别佐佐木希职业发展的关键转型节点与典型路径。\par
  (2)构建"职业状态转移矩阵"(Career State Transition Matrix),量化不同职业状态之间的转移概率。\par
  (3)开发"职业转型预测模型"原型,实现转型时机与方向的初步预测。\NSFCBlankPara

  \item \itemtitlefont{形象分析阶段(第19-24个月):}\par
  (1)采用深度学习方法(CNN、Vision Transformer)对佐佐木希的影像资料进行特征提取,量化形象呈现的演变规律。\par
  (2)运用自然语言处理技术(BERT、GPT)分析文本资料中的形象建构策略,识别关键词汇与主题演变。\par
  (3)构建"跨媒介形象一致性测量指标体系",实现形象一致性的量化评估。\NSFCBlankPara
\end{enumerate}

\textbf{年度预期成果:}完成职业轨迹建模分析;发表实证研究论文2-3篇;开发模型原型系统。

\subsubsection{第三年研究计划}

\textbf{研究重点:模型验证、案例比较、成果总结与转化}

\begin{enumerate}
  \item \itemtitlefont{模型验证与优化阶段(第25-30个月):}\par
  (1)通过交叉验证(Cross-Validation)和敏感性分析(Sensitivity Analysis)评估"职业转型预测模型"的稳健性。\par
  (2)选取同时期其他日本女性艺人(如坛蜜、木下优树菜、水原希子等)作为对照案例,检验模型的外部效度。\par
  (3)根据验证结果优化模型参数,提高预测准确性。\NSFCBlankPara

  \item \itemtitlefont{成果总结与转化阶段(第31-36个月):}\par
  (1)撰写研究总报告,系统总结研究发现与理论贡献。\par
  (2)开发"艺人职业规划决策支持工具"原型,进行用户测试与优化。\par
  (3)组织学术研讨会,向娱乐产业从业者、政策制定者、学术界传播研究成果。\par
  (4)撰写专著《艺人职业发展的动态机制:基于佐佐木希案例的纵向研究》。\NSFCBlankPara
\end{enumerate}

\textbf{年度预期成果:}完成模型验证与优化;发表高水平研究论文3-4篇;完成研究总报告;开发决策支持工具原型。

\subsubsection{预期研究结果}

\textbf{理论贡献:}

\begin{enumerate}
  \item \itemtitlefont{理论创新:}\par
  (1)提出"艺人职业发展三阶段动态模型"(Exploration-Stabilization-Reinvention),为职业发展理论提供新的理论视角。\par
  (2)建立"品牌韧性"(Brand Resilience)理论框架,揭示负面事件后品牌修复的内在机制。\par
  (3)构建"跨媒介形象一致性测量指标体系",实现形象管理的量化评估。\NSFCBlankPara

  \item \itemtitlefont{方法学贡献:}\par
  (1)开发"艺人职业轨迹编码手册",为相关研究提供标准化的研究工具。\par
  (2)提出多模态数据融合分析框架,推动文化研究方法的数字化革新。\par
  (3)建立"舆情事件影响评估模型",为品牌危机管理提供量化评估工具。\NSFCBlankPara
\end{enumerate}

\textbf{实践价值:}

\begin{enumerate}
  \item \itemtitlefont{对娱乐产业:}\par
  (1)为经纪公司提供"艺人职业规划决策支持工具",提高艺人管理的科学性。\par
  (2)为艺人本人提供"个人品牌管理指南",增强职业发展的可持续性。\NSFCBlankPara

  \item \itemtitlefont{对文化产业政策:}\par
  (1)为政府文化部门制定艺人保护政策、完善娱乐产业规范提供实证依据。\par
  (2)为文化产业的可持续发展提供理论指导。\NSFCBlankPara
\end{enumerate}

\textbf{学术产出:}

\begin{enumerate}
  \item \itemtitlefont{论文发表:}\par
  (1)计划发表高水平学术论文6-8篇,其中SSCI/CSSCI检索论文不少于4篇。\par
  (2)目标期刊包括《Journal of Business Research》、《Tourism Management》、《新闻与传播研究》、《现代传播》等。\NSFCBlankPara

  \item \itemtitlefont{学术专著:}\par
  (1)撰写专著1部,系统呈现研究成果。\NSFCBlankPara

  \item \itemtitlefont{学术交流:}\par
  (1)参加国际学术会议2-3次,包括ICA、IAMCR等顶级会议。\par
  (2)组织专题学术研讨会1-2次,推动学术交流与合作。\NSFCBlankPara
\end{enumerate}

\endgroup
