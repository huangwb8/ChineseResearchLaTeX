\begingroup
\justifying
\setlength{\parindent}{2em}% 首行缩进 2 字符

% "研究基础与可行性分析"(2026 版)建议写作顺序:
% 与本项目相关的研究工作积累 → 已取得的研究工作成绩 → 研究风险的应对措施

\subsubsection{研究基础}
% 建议:梳理前期工作,说明本项目延续和深化的基础。
\NSFCBlankPara
\NSFCBlankPara

\subsubsection{已取得的研究工作成绩}
% 建议:列出相关论文、专利、预实验结果等,证明团队能力。
\NSFCBlankPara
\NSFCBlankPara

\subsubsection{可行性分析}
% 建议:按"理论/技术/团队/条件"四维写可行性;并给出 2–3 条风险与备选方案。
\begin{enumerate}
  \item \itemtitlefont{理论可行性:}\par\NSFCBlankPara
  \item \itemtitlefont{技术可行性:}\par\NSFCBlankPara
  \item \itemtitlefont{团队与条件可行性:}\par\NSFCBlankPara
  \item \itemtitlefont{主要风险与对策:}\par\NSFCBlankPara
\end{enumerate}

\endgroup
