\begingroup
\justifying
\setlength{\parindent}{2em}% 首行缩进 2 字符

% "研究基础与可行性分析"(2026 版)建议写作顺序:
% 与本项目相关的研究工作积累 → 已取得的研究工作成绩 → 研究风险的应对措施

\subsubsection{研究基础}
% 建议:梳理前期工作,说明本项目延续和深化的基础。

\textbf{【示例内容】}

申请团队长期致力于该领域的基础研究和应用探索,积累了丰富的研究经验。近年来,团队主持完成了国家自然科学基金项目"XXX理论及其应用研究"(批准号:XXXXXXXX,资助金额:XX万元,起止年月:20XX年XX月—20XX年XX月),在该项目的支持下,团队在理论建模、算法设计和系统实现三个方面取得了系统性进展,为本项目的开展奠定了坚实基础。

同时,团队还参与了国家重点研发计划"YYY关键技术研究"(项目编号:YYYYYYYY),负责其中的算法优化模块。通过该项目,团队积累了大规模工程化实践经验,培养了技术攻关能力,建立了完整的研发流程和质量管理体系。

在学术交流方面,团队与美国XXX大学、英国YYY研究所等国际一流研究机构建立了长期合作关系,定期开展学术访问和联合研究,为本项目的国际化视野和前沿性提供了保障。

\NSFCBlankPara

\subsubsection{已取得的研究工作成绩}
% 建议:列出相关论文、专利、预实验结果等,证明团队能力。

\textbf{【示例内容】}

\textbf{学术论文:}团队在该领域顶级期刊和会议上发表了系列研究成果。近5年来,发表SCI收录论文XX篇,其中TOP期刊论文XX篇,包括IEEE Transactions on XXX(2篇)、ACM XXX(1篇)、中国科学(1篇)等。这些论文被国内外同行广泛引用(Google Scholar引用次数:XXX次),产生了重要的学术影响。

\textbf{发明专利:}申请国家发明专利XX项,其中已授权XX项。核心专利"一种基于XXX的高效算法"(专利号:ZLXXXXXXXXXX.X)已成功转化应用,产生了显著的经济效益。

\textbf{软件著作权:}获得软件著作权XX项,开发的"XXX数据分析系统"已在多家单位和科研机构得到应用。

\textbf{获奖情况:}研究成果获得XXX科学技术奖一等奖(20XX年)、YYY协会优秀论文奖(20XX年)等多项奖励。

\textbf{预实验结果:}团队已完成初步的概念验证实验,实验结果表明新方法在效率上较现有方法提升XX%,精度提升XX%,验证了核心思路的可行性和有效性。

\NSFCBlankPara

\subsubsection{可行性分析}
% 建议:按"理论/技术/团队/条件"四维写可行性;并给出 2–3 条风险与备选方案。

\textbf{【示例内容】}

\begin{enumerate}
  \item \itemtitlefont{理论可行性:}\par
  本项目提出的理论框架建立在坚实的数学基础之上,相关理论已经过严格的数学推导和证明。团队前期的理论研究表明,该框架在理论上是完备的,能够覆盖现有理论的特殊情况,并在关键假设上有所突破。通过大量的文献调研和专家咨询,我们确认该理论方向是可行的,具有明确的研究价值和应用前景。

  \item \itemtitlefont{技术可行性:}\par
  团队在相关技术领域积累了丰富的经验,掌握了核心算法的关键技术。前期的预实验结果表明,技术路线是可行的,主要技术难点已经找到解决方案。团队配备了先进的实验设备和计算资源,包括高性能计算集群(XX核XX内存)、GPU加速卡(XX块)等,能够满足项目的技术需求。

  \item \itemtitlefont{团队与条件可行性:}\par
  \textbf{团队构成:}项目负责人为XXX教授/研究员,长期从事该领域研究,发表SCI论文XX篇,主持国家级项目XX项。研究团队包括教授/研究员X名,副教授/副研究员X名,博士/硕士研究生X名,形成了合理的人才梯队。团队成员在理论、算法、系统等方面各有所长,能够协同攻关。\par
  \textbf{实验条件:}依托XXX国家重点实验室/XXX教育部重点实验室,拥有完备的实验设备和先进的计算平台,能够满足项目的研究需求。\par
  \textbf{合作基础:}与国内外多个顶尖研究机构建立了长期合作关系,能够提供必要的技术支持和学术交流。

  \item \itemtitlefont{主要风险与对策:}\par
  \textbf{风险1:}理论模型可能存在未预见的局限性。\par
  \textbf{对策:}采用渐进式研究策略,先在简化模型上验证核心思想,再逐步扩展到一般情况;建立定期专家咨询机制,及时发现和解决问题。\par

  \textbf{风险2:}算法性能可能达不到预期指标。\par
  \textbf{对策:}设计多套备选方案,在不同技术路线并行推进;引入最新的优化技术,定期进行性能评估和调整。\par

  \textbf{风险3:}实验数据获取可能存在困难。\par
  \textbf{对策:}提前与数据提供方建立合作关系,制定数据共享协议;开发数据合成和增强技术,减少对真实数据的依赖。
\end{enumerate}

\endgroup
