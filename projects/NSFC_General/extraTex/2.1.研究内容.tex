\justifying
\NSFCBodyText

\subsubsection{研究内容}

\subsubsubsection{职业发展轨迹纵向研究}
本研究以刘亦菲2002-2025年职业发展为案例,梳理职业路径,划分为五阶段(2002-2005童星期、2005-2008古装剧巅峰期、2008-2012转型探索期、2012-2020电影发展期、2020-2025国际化突破期),识别转型触发点与关键决策。

\subsubsubsection{跨文化形象构建机制}
分析不同文化语境下形象呈现策略,探讨“古装女神”标签的建构、固化与突破过程,研究从“东方古典美”到“国际动作巨星”的形象转型策略。

\subsubsubsection{国际化路径机制}
以《花木兰》项目为关键事件,分析华人女演员突破好莱坞的策略选择、文化适应机制及市场定位策略,探讨“中国元素”在国际化进程中的作用。

\subsubsubsection{社交媒体时代粉丝经营策略}

\subsubsection{研究目标}

\subsubsubsection{理论目标}
本项目总体目标是构建基于中国文化语境的艺人国际化发展动态分析框架。提出“华人演员国际化三阶段模型”(本土积累期-区域拓展期-全球突破期),建立“跨文化形象一致性测量指标体系”。

\subsubsubsection{方法学目标}
开发“艺人国际化轨迹编码手册”,构建“跨文化形象转型评估模型”,建立“好莱坞突破影响因素分析框架”。

\subsubsubsection{应用目标}

\subsubsection{拟解决的关键科学问题}

\subsubsubsection{华人演员国际化的临界点识别问题}
如何科学识别从区域市场向全球市场跨越的关键节点?哪些内外部因素起决定性作用?《花木兰》项目的成功是偶然还是必然?

\subsubsubsection{跨文化形象的一致性维持问题}
如何在保持“东方古典美”核心标签的同时实现“国际动作巨星”的形象转型?如何平衡文化特殊性与全球普遍性?

\subsubsubsection{标签化与去标签化的动态平衡问题}

\subsubsection{研究方法}

\subsubsubsection{纵向案例研究法}
本研究采用混合研究方法。收集完整职业档案(媒体报道约8000篇、影视作品约40部、商业代言约60个、社交媒体数据等),采用时序分析识别关键转折点,重点关注2005年《仙剑奇侠传》、2020年《花木兰》等关键节点。

\subsubsubsection{内容分析法}
对影视作品约40部进行编码分析,提炼角色类型演变规律;对国内外媒体报道进行跨文化比较分析。编码工作由两名研究者独立完成,采用Cohen's Kappa系数评估信度(K>0.8)。

\subsubsubsection{跨文化比较法}
收集《花木兰》项目期间中美两国社交媒体舆情数据约20万条,运用情感分析和主题模型评估跨文化受众反应差异。采用LDA模型进行主题分析,使用BERT预训练模型进行情感分类(准确率目标85\%以上)。

\subsubsubsection{比较研究法}

\subsubsection{技术路线}

\subsubsubsection{数据收集与预处理}
本研究技术路线分四阶段(图\ref{fig:techroute})。建立多源异构数据库,包括影视作品数据库、媒体报道数据库、社交媒体数据库、票房与市场数据库,并将数据清洗、结构化存储与可追溯日志记录纳入同一流程。

\lstinputlisting[
  language=bash,
  firstline=1,
  lastline=14,
  caption={数据处理流水线脚本示例(code/test.sh)},
  label={code:data-pipeline}
]{code/test.sh}

\subsubsubsection{职业轨迹建模}
应用序列分析识别职业发展典型路径,构建“国际化转型状态转移矩阵”,识别从“本土明星”到“国际巨星”的关键跃迁点。

\subsubsubsection{形象分析}
采用深度学习方法对影像资料进行特征提取,运用NLP技术分析中英文文本资料,构建“跨文化形象语义网络”。

\subsubsubsection{模型验证与应用}

\begin{figure}[htbp]
  \centering
  \includegraphics[width=0.95\textwidth]{figures/lyf-001.jpg}
  \caption{关键模型与跨平台链路示意(projects/NSFC\_General/figures/)。}
  \label{fig:techroute}
\end{figure}

\subsubsection{关键技术}

\subsubsubsection{多模态数据融合}
整合文本、图像、视频等多模态数据,构建统一知识图谱。

\subsubsubsection{职业轨迹序列挖掘}
应用频繁序列模式挖掘算法识别职业发展典型模式。

\subsubsubsection{舆情情感分析}

\subsubsection{可行性分析}

\subsubsubsection{理论可行}
建立在成熟的职业发展理论、品牌管理理论、传播学理论基础之上。

\subsubsubsection{数据可行}
刘亦菲公开资料丰富,IMDb、Box Office Mojo、主流媒体数据库及社交平台数据为研究提供了充足数据源。

\subsubsubsection{技术可行}
