\begingroup
\justifying
\setlength{\parindent}{2em}% 首行缩进 2 字符

\subsubsection{研究内容}
% 建议:拆成 3–4 条"研究内容",每条都写清:任务→方法→数据/实验→输出/里程碑。

\textbf{【演示内容:基于佐佐木希案例的虚拟研究课题】}

\begin{enumerate}
  \item \itemtitlefont{研究内容 1:佐佐木希职业生涯纵向追踪与阶段划分}\par
  \textbf{任务:}系统梳理佐佐木希从2005年模特出道到2025年的完整职业轨迹,识别职业发展的关键节点和阶段特征。\par
  \textbf{方法:}采用历史文献分析法、纵向案例研究法和时间序列分析法,系统收集其影视作品、媒体报道、社交媒体数据。\par
  \textbf{数据/实验:}收集2005-2025年间的影视作品数据库(日本电影数据库、电视剧评分网站)、社交媒体数据(Twitter/X、Instagram)、媒体报道(新闻文本、杂志采访),共计约5000条数据记录。通过时间序列分析识别职业发展的三个阶段:颜值驱动期(2005-2008)、转型适应期(2008-2015)、实力驱动期(2015-2025)。\par
  \textbf{输出/里程碑:}第1年末完成纵向追踪数据集构建和阶段划分模型,发表案例研究论文1篇。\NSFCBlankPara

  \item \itemtitlefont{研究内容 2:职业转型影响因素的多模态数据分析}\par
  \textbf{任务:}识别并量化从模特向演员转型的关键影响因素,构建职业转型成功要素指标体系。\par
  \textbf{方法:}采用多模态数据融合、机器学习(随机森林、梯度提升树)和文本挖掘(情感分析、主题模型)方法。\par
  \textbf{数据/实验:}整合影视作品评分(豆瓣、IMDb)、社交媒体互动数据(点赞数、转发数、评论情感)、媒体报道主题分布。通过特征工程提取30个候选影响因素,使用机器学习模型识别关键特征并计算权重。\par
  \textbf{输出/里程碑:}第2年中完成影响因素指标体系构建,开发开源数据分析工具包1个,发表方法学论文1-2篇。\NSFCBlankPara

  \item \itemtitlefont{研究内容 3:职业发展演化动力学模型构建与验证}\par
  \textbf{任务:}基于佐佐木希案例,构建艺人职业发展的三阶段演化动力学模型,并通过其他日本演员案例进行验证。\par
  \textbf{方法:}采用微分方程建模、系统动力学和元分析相结合的方法。\par
  \textbf{数据/实验:}构建"影响力—时间"微分方程模型,刻画颜值驱动期和实力驱动期的不同增长曲线。收集20位日本演员的辅助案例数据进行模型验证,计算模型的预测准确率和泛化能力。\par
  \textbf{输出/里程碑:}第3年末完成理论模型构建和验证,发表理论论文1-2篇,申请软件著作权1项。\NSFCBlankPara

  \item \itemtitlefont{研究内容 4:日本演员职业发展标准数据集构建}\par
  \textbf{任务:}制定数据采集标准,整合多源数据,构建可复用的日本演员职业发展标准数据集。\par
  \textbf{方法:}采用数据标准化、实体对齐和多源数据融合技术。\par
  \textbf{数据/实验:}整合日本电影数据库、电视剧数据库、社交媒体API、新闻媒体档案等6个数据源。制定数据标注规范(作品类型、角色重要性、影响力指标等),构建包含50位演员、覆盖1970-2025年的标准化数据集。\par
  \textbf{输出/里程碑:}第3年末完成数据集构建并公开发布,发表数据论文1篇。\NSFCBlankPara
\end{enumerate}

【示例:图片插入】图 \ref{fig:framework-design} 展示了职业发展演化模型的设计思路。

\begin{figure}[!ht]
    \begin{center}
        \includegraphics[width=0.75\linewidth]{figures/zzmx-mobile-105.jpg}
        \caption{职业发展演化动力学模型示意图。本图展示了研究内容3中三阶段演化模型的构建思路,包括颜值驱动期、转型适应期、实力驱动期的动力学方程、关键状态变量(影响力、作品数量、观众认可度)以及阶段转换的临界条件。红色箭头表示阶段转换方向,蓝色曲线表示影响力演化路径。}
        \label{fig:framework-design}
    \end{center}
\end{figure}

\subsubsection{研究目标与可验证指标}
% 建议:目标要"可度量",指标要"可验证",并与研究内容对应。

\textbf{【示例内容】}

\begin{enumerate}
  \item \itemtitlefont{总体目标:}\par
  建立全新的理论体系,突破现有方法的性能瓶颈,构建可复用的研究平台,推动该领域的科学进步和技术发展。

  \item \itemtitlefont{关键指标:}\par
  \textbf{理论层面:}建立完整理论框架,关键理论指标优于现有方法30\%以上;发表高水平SCI论文3-5篇,其中TOP期刊论文不少于2篇。\par
  \textbf{方法层面:}算法计算效率提升50\%以上,精度保持或优于现有方法;开源工具包获得社区广泛使用。\par
  \textbf{应用层面:}成果在2-3个实际场景中得到验证和应用;申请发明专利2-3项,软件著作权1-2项。\NSFCBlankPara

  \item \itemtitlefont{预期产出:}\par
  \textbf{学术论文:}SCI论文3-5篇,其中国际TOP期刊2-3篇。\par
  \textbf{知识产权:}发明专利2-3项,软件著作权1-2项。\par
  \textbf{开源工具:}算法工具包1个,数据分析平台1个,标准数据集1-2个。\par
  \textbf{人才培养:}培养博士研究生1-2名,硕士研究生2-3名。\par
  \textbf{学术交流:}参加国际学术会议2-3次,做特邀报告1-2次。
\end{enumerate}

\subsubsection{拟解决的关键科学问题(与立项依据保持一致)}

\textbf{【示例内容】}
本项目拟解决的关键问题与立项依据中提出的问题一一对应:

\begin{enumerate}
  \item \itemtitlefont{问题 1:如何建立高精度、高效率的理论模型?}\par
  针对该问题,本项目将引入创新的数学建模方法,突破传统理论模型的假设限制。具体而言,将采用XXX理论作为基础,结合YYY方法,构建能够准确描述复杂系统行为的理论框架。该问题的解决将为后续算法设计和应用验证奠定坚实的理论基础。\NSFCBlankPara

  \item \itemtitlefont{问题 2:如何设计兼顾效率和精度的算法框架?}\par
  针对该问题,本项目将设计全新的算法架构,通过优化数据结构和计算流程,实现效率和精度的双重提升。具体策略包括:采用并行计算技术提升效率,引入自适应机制保持精度,利用GPU加速等方法突破性能瓶颈。\NSFCBlankPara

  \item \itemtitlefont{问题 3:如何构建标准化、可复用的数据资源?}\par
  针对该问题,本项目将制定数据标准化规范,建立数据质量评估体系,构建可复用的数据共享平台。具体措施包括:制定数据采集和预处理标准,建立数据标注和质量控制流程,开发数据管理和共享平台。\NSFCBlankPara
\end{enumerate}

\subsubsection{研究方法}

\textbf{【演示内容】}

本研究采用"纵向案例研究+多模态数据分析+数学建模+计算验证"的综合方法框架。

\textbf{(1)纵向案例研究方法:}对佐佐木希的职业生涯进行20年纵向追踪(2005-2025年),采用历史文献分析法系统收集影视作品记录、媒体报道档案、社交媒体数据,构建时间序列数据集。通过关键事件分析法识别职业发展的转折节点(如2008年首次主演电影、2015年《朝5晚9》热播)。

\textbf{(2)多模态数据融合方法:}整合结构化数据(影视作品数据库、评分数据)和非结构化数据(社交媒体文本、新闻报道),采用自然语言处理(NLP)技术进行文本挖掘和情感分析。通过多模态特征融合技术,构建包含个人特质、作品表现、社会网络、市场反馈的综合特征空间。

\textbf{(3)数学建模方法:}基于系统动力学理论,构建职业发展演化的微分方程模型。设影响力指数 $I(t)$ 为状态变量,建立三阶段动力学方程:
\begin{itemize}
  \item \textit{颜值驱动期(2005-2008):} $\frac{dI}{dt} = \alpha_1 A(t) - \beta_1 I(t)$,其中 $A(t)$ 为外貌优势指标
  \item \textit{转型适应期(2008-2015):} $\frac{dI}{dt} = \alpha_2 S(t) + \alpha_3 A(t) - \beta_2 I(t)$,引入演技提升 $S(t)$
  \item \textit{实力驱动期(2015-2025):} $\frac{dI}{dt} = \alpha_4 S(t) + \alpha_5 C(t) - \beta_3 I(t)$,$C(t)$ 为观众认可度
\end{itemize}

\textbf{(4)对照验证策略:}收集20位同期日本演员的对比案例,其中10位成功转型(模特→演员),10位转型失败或未转型。通过元分析方法,验证佐佐木希案例的普适性和特殊性。采用k-fold交叉验证评估模型的预测准确率,目标达到 $R^2 > 0.75$。\NSFCBlankPara
\NSFCBlankPara

\subsubsection{技术路线}
% 建议:用 1 段文字 +(可选)流程图;强调"输入→处理→输出"的闭环。

本项目的技术路线遵循"数据采集→特征提取→模型构建→验证应用"的闭环流程:

\textbf{阶段1(数据采集):}通过API接口爬取日本电影数据库、豆瓣电影、Twitter/X、Instagram等平台数据,清洗并整合为标准数据集。对非结构化文本(媒体报道、社交媒体评论)进行分词、去停用词、实体识别等预处理。

\textbf{阶段2(特征提取):}从多模态数据中提取特征向量,包括:(1)作品特征:作品数量、类型分布、角色重要性、评分变化;(2)个人特征:年龄、形象变化、技能提升;(3)社会网络特征:合作演员、导演、经纪公司变化;(4)市场反馈特征:评分趋势、搜索量、社交媒体互动量。

\textbf{阶段3(模型构建):}首先基于时间序列分析划分职业发展阶段,然后采用随机森林算法识别关键影响因素,最后构建系统动力学微分方程模型刻画演化规律。

\textbf{阶段4(验证应用):}使用对比案例验证模型泛化能力,开发可视化分析平台,为演艺产业决策提供支持。\NSFCBlankPara
\NSFCBlankPara

\subsubsection{关键技术与关键环节}
% 建议:列出 2–4 个"关键点",说明为什么关键、怎么攻关、预期达到什么水平。
\begin{enumerate}
  \item \itemtitlefont{关键技术 1:跨语言多源数据融合技术}\par
  该技术是本研究的基础,涉及日语、中文、英语多语言数据的整合。关键技术难点包括:(1)实体对齐:同一演员在不同数据库中的名称变体识别(如"佐佐木希"、"Nozomi Sasaki"、"佐々木希");(2)数据标准化:不同平台评分体系的归一化处理;(3)时间对齐:多源数据的时间戳统一和缺失值插补。拟采用基于深度学习的实体识别模型(BERT多语言预训练模型)和图神经网络(GNN)进行数据融合,预期实体识别准确率 $>95\%$。\NSFCBlankPara

  \item \itemtitlefont{关键技术 2:职业发展阶段识别算法}\par
  该技术用于从时间序列数据中自动识别职业发展的阶段转换点。传统方法依赖人工标注,效率低且主观性强。拟基于变化点检测(Change Point Detection)算法,结合贝叶斯推断和隐马尔可夫模型(HMM),自动识别佐佐木希职业轨迹中的关键转换节点。该技术的创新在于将影响力指数、作品质量、观众认可度等多维时间序列融合,提升阶段识别的鲁棒性。预期阶段识别准确率 $>90\%$,与专家人工标注的Kappa一致性系数 $>0.85$。\NSFCBlankPara

  \item \itemtitlefont{关键技术 3:系统动力学模型参数估计}\par
  该技术用于估计三阶段演化动力学模型中的参数($\alpha_i, \beta_i$),涉及微分方程的数值求解和参数拟合。关键技术难点包括:(1)模型辨识:从离散的时间序列数据中估计连续微分方程的参数;(2)约束优化:参数需满足经济学和社会学的合理性约束(如增长率非负、衰减率有界)。拟采用最大似然估计(MLE)结合马尔可夫链蒙特卡洛(MCMC)采样方法,进行贝叶斯推断。预期参数估计的置信区间覆盖率 $>90\%$,模型预测的均方根误差(RMSE)相比基准模型降低30\%以上。\NSFCBlankPara
\end{enumerate}

【示例:代码插入】以下展示关键算法的实现代码(引用 \texttt{code/test.sh}):

\subsubsection{关键算法实现}
\begin{figure}[!ht]
\centering
\lstinputlisting[style=codestyle01, language=Bash, basicstyle=\tiny\ttfamily]{code/test.sh}
\caption{关键算法实现代码示例}
\end{figure}

\endgroup
