\justifying
\NSFCBodyText

\subsubsection{与本项目相关的研究工作积累}

\subsubsubsection{华人演员国际化研究}
\indent 申请团队长期致力于文化产业、演员职业发展、跨文化传播研究。前期对华人演员好莱坞突破路径深入研究,发表《华人演员在好莱坞的类型化与去类型化》等论文,系统分析华人演员在国际市场的形象建构机制;对中美影视产业合作模式进行比较研究,探讨"文化折扣"与"文化溢价"的动态平衡。

\subsubsubsection{个人品牌理论研究}
\indent 对跨文化个人品牌理论发展系统梳理,发表《跨文化个人品牌理论的前沿进展》文献综述,总结个人品牌建构的"真实性-一致性-适应性"三维度模型;对华人明星国际化案例进行研究,提出"标签化-去标签化动态平衡模型"。

\subsubsubsection{计算方法储备}

\subsubsection{已取得的研究工作成绩}

\subsubsubsection{论文发表}
\indent 在《新闻与传播研究》《国际新闻界》《现代传播》等CSSCI期刊发表相关论文10余篇,其中2篇被《新华文摘》全文转载;在International Journal of Cultural Studies、Asian Journal of Communication等SSCI期刊发表相关论文5篇,总引用超200次。

\subsubsubsection{项目主持}
\indent 主持国家社科基金项目"新媒体环境下中国影视产业国际化路径研究"(已结项,鉴定等级"优秀");主持省部级项目"华人演员好莱坞突破机制研究"(在研)。

\subsubsubsection{专著出版}
\indent 出版专著《文化产业与跨文化品牌管理》(2022),系统阐述跨文化品牌建设理论;出版教材《跨文化传播研究方法导论》(2023),介绍跨语言文本分析、跨文化比较等方法。

\subsubsubsection{科研奖励}

\subsubsection{研究风险的应对措施}

\subsubsubsection{数据获取风险}
\indent 部分社交媒体历史数据和好莱坞内部资料可能因平台政策或版权限制无法完整获取。应对措施:建立多源数据获取渠道;与美国相关研究机构(USC安纳伯格传播学院、UCLA亚太研究中心)建立合作关系;利用合法数据服务提供商获取数据;自主研发跨语言网络爬虫工具。

\subsubsubsection{模型有效性风险}
\indent 基于单个案例构建的模型可能缺乏普适性。应对措施:采用多案例比较研究设计,引入对照案例(章子怡、巩俐、杨紫琼等);通过交叉验证和敏感性分析评估模型稳健性;充分考虑文化语境的调节作用;邀请国际知名专家对研究成果评议。

\subsubsubsection{跨文化理解风险}

