
\justifying

\indent\setlength{\parindent}{2em}%首行缩进4字符

申请人与本项目相关的研究工作积累和已取得的研究工作成绩主要包括以下几个方面:

\subsubsubsection{明星研究领域的研究基础}
申请人长期从事明星研究(Star Studies)、文化产业研究,已在《文艺研究》《当代电影》《电影艺术》等CSSCI期刊发表相关论文5篇。其中,《日本女演员的符号意义与文化价值》一文系统分析了日本女演员在东亚流行文化中的地位和作用;《从童星到成人演员:青年演员的转型困境与出路》一文探讨了童星转型期的理论问题,为本项目的"童星转型期"概念模型奠定了理论基础。

申请人已完成的省部级项目"日本影视产业的人才培养机制研究"(项目编号:XXX-20XXX,资助金额:8万元),系统调研了日本演艺产业的人才培养体系,收集了大量一手资料,包括:经纪公司的培训课程、表演学校的课程设置、演员的职业规划案例等。这些前期研究为本项目提供了坚实的资料基础和理论准备。

\subsubsubsection{跨文化比较研究的能力}
申请人具有丰富的跨文化研究经验,曾于2018-2019年在日本东京大学访问学习一年,期间系统学习了日本电影史、日本文化产业论等课程,建立了广泛的学术联系网络。申请人已与日本某大学签署合作协议,可通过该校图书馆访问《电影旬报》《日刊スポーツ》等数据库,为本项目的数据收集提供了便利条件。

申请人精通日语,具备阅读日文资料和进行日语访谈的能力,确保研究过程中能够准确理解日本语境下的文化现象和产业逻辑。

\subsubsubsection{研究团队的学术积累}
研究团队成员包括:1名副教授(负责人,负责理论构建和总体设计)、2名博士生(负责数据收集和分析)、3名硕士生(负责文献整理和编码)。团队成员具备影视学、社会学、统计学等多学科背景,能够胜任跨学科研究任务。

团队成员在前期研究中已熟练掌握NVivo qualitative data analysis software、Stata等分析工具,并完成了预调研(10个访谈、100份问卷),验证了研究设计的可行性。团队成员已发表相关学术论文10余篇,其中CSSCI期刊论文5篇,具有扎实的学术基础和研究能力。

\subsubsubsection{前期研究成果的转化}
申请人已完成的相关研究为本项目提供了重要的理论和方法支持。例如,前期项目中的"日本演员培养机制数据库"可直接用于本项目的案例分析;前期研究中建立的"中日观众评价比较模型"可用于本项目的跨文化比较研究;前期研究中开发的"表演分析编码手册"可为本项目的微观表演分析提供标准。

这些前期研究成果的转化,将大大提高本研究的效率和质量,确保项目按期完成并取得高水平成果。

\clearpage
