\justifying

\subsubsection{研究计划}

\indent\setlength{\parindent}{2em}%首行缩进4字符

第 1 年:完成数据规范化与标注流程;搭建基线 CNN(U-Net/ResNet)与训练管线,系统评估常规增强(翻转、旋转、裁剪、强度扰动)。
第 2 年:面向小病灶与多尺度结构,研发注意力与多尺度融合模块;引入组合增强(RandAugment、MixUp/CutMix)与一致性训练,开展消融实验。
第 3 年:开展跨中心泛化与迁移验证,形成可复现的模型/增强策略推荐方案;完善不确定性评估与失败案例分析,输出原型系统与论文/专利。

\subsubsection{预期研究结果}

\indent\setlength{\parindent}{2em}%首行缩进4字符

预期形成:
(1)一套面向医疗影像的 CNN 架构与训练策略(含模块化实现与参数配置建议);
(2)一套可复用的数据增强与鲁棒训练基准(含增强强度标定与迁移规则);
(3)公开可复现的实验报告与对比基线,并在典型任务(分割/分类)上取得稳定性能提升。

\clearpage
