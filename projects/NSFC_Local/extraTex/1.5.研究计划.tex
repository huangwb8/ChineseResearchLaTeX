\justifying
\NSFCBodyText

\subsubsection{研究计划}

\subsubsubsection{第 1 年:文献梳理与数据收集}
完成明星研究、职业生涯发展理论、文化社会学等领域的文献梳理,构建理论分析框架。提出"童星转型期"概念模型,明确研究假设和分析维度。通过日本合作大学图书馆,收集滨边美波2011-2025年所有影视作品资料,建立作品数据库。同时,收集日本主流媒体关于滨边美波的报道(约500篇),完成预编码工作。申请并通过伦理审查(IRB)。参加1-2次国内学术会议,进行初步研究成果交流。

\subsubsubsection{第 2 年:数据分析与深度访谈}
完成影视作品的微观表演分析,采用双人独立编码方式,编码一致性检验达到0.8以上。对媒体报道和观众评论进行话语分析和情感分析,识别核心主题和话语框架。选取日本演艺产业从业者10-15人进行深度访谈,获取产业内部的一手资料。设计中日观众调查问卷,完成预调查(100份)并修正问卷。开展大规模问卷调查(日本500份,中国500份)。完成初步数据分析,撰写2篇学术论文并投稿CSSCI期刊。

\subsubsubsection{第 3 年:案例分析与理论验证}
基于统计数据和质性资料,对滨边美波的职业生涯进行系统性分析,验证"童星转型期"理论模型的解释力。开展跨案例分析(选取2-3位同时期童星进行对比研究),提炼青年演员职业发展的共性规律。完成中日观众问卷数据的跨文化比较分析,识别两国在青年演员培养、观众期待等方面的异同。组织专家研讨会,邀请中国演艺产业从业者、政策制定者、学者共同讨论日本经验的适用性。撰写3-4篇学术论文(其中至少1篇投稿SSCI期刊),完成政策建议报告。

\subsubsection{预期研究结果}

\subsubsubsection{理论成果}
(1)构建"童星转型期"(Child Star Transition Phase)理论模型,发表高水平学术论文3-5篇(其中CSSCI期刊2-3篇,SSCI期刊1-2篇);(2)出版学术专著1部《日本青年演员职业发展研究:以滨边美波为例》;(3)形成"个人-产业-文化"三维分析框架,为明星研究领域提供新的方法论工具。

\subsubsubsection{实践成果}
(1)完成《中国青年演员培养政策建议报告》,提交文化和旅游部、中国电影家协会等相关部门;(2)建立"日本演艺产业案例库",包含滨边美波及其他童星的作品资料、访谈记录、媒体报道等,为后续研究提供数据基础;(3)举办1次"中日青年演员培养研讨会",促进两国文化产业交流与合作。

\subsubsubsection{人才培养}
(1)培养博士生2名,硕士生3名,其中至少1名博士生的学位论文获得省级优秀学位论文;(2)组织研究团队赴日本合作大学交流访问1-2次,提升国际学术视野;(3)开设"明星研究与文化产业"研究生课程,将研究成果转化为教学资源。

