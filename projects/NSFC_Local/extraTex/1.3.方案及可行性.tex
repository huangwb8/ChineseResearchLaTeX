\justifying
\NSFCBodyText

\subsubsection{研究方法}

\subsubsubsection{纵向案例研究法}
本研究采用纵向案例研究法,以滨边美波2011-2025年的职业生涯为时间轴,系统梳理其影视作品、媒体报道、业内评价等多维度数据。通过时间序列分析,识别其职业发展的关键节点、转型期特征和风格演变轨迹。该方法适合于探究"如何"和"为什么"类型的因果问题,能够深入揭示青年演员职业发展的动态过程。

\subsubsubsection{文本分析法}
对影视作品进行微观表演分析(情感表达、台词处理、肢体语言等),并对主流媒体报道进行话语分析,以刻画形象建构策略与舆论框架。

\subsubsubsection{深度访谈法}
选取经纪公司、制作方与表演训练体系相关从业者进行半结构化访谈,补充公开资料难以覆盖的培养机制、作品选择与行业评价标准等信息。

\subsubsection{技术路线}

\subsubsubsection{第一阶段:文献梳理与理论构建}
系统梳理明星研究(Star Studies)、职业生涯发展理论(Career Development Theory)、文化社会学等领域的经典文献,构建本研究的理论分析框架。提出"童星转型期"(Child Star Transition Phase)的概念模型,明确关键变量和分析维度。完成研究设计和伦理审查申请。

\subsubsubsection{第二阶段:数据收集与数据库建立}
收集滨边美波2011-2025年所有影视作品(电影、电视剧、舞台剧、配音作品),建立包含以下字段的作品数据库:作品名称、类型、播出年份、角色类型、制作公司、收视率/票房、获奖情况等。同时,收集媒体报道(约500篇)、观众评论(网络爬虫,约2000条)、业内访谈(约20人)等多模态数据。

\subsubsubsection{第三阶段:作品分析与编码训练}
对影视作品进行微观表演分析,采用双人独立编码方式,编码一致性检验(Cohen's Kappa)需达到0.8以上。编码维度包括:情感表达强度、台词处理方式、肢体语言特征、角色类型(清纯/叛逆/成熟等)。对媒体报道和观众评论进行开放式编码和轴心编码,识别核心主题和话语框架。

\subsubsubsection{第四阶段:案例分析与理论验证}
基于统计数据和质性资料,对滨边美波的职业生涯进行阶段性分析,验证"童星转型期"理论模型的解释力。通过跨案例分析(与同时期其他童星对比),提炼青年演员职业发展的共性规律和个性特征。撰写研究报告初稿,并进行同行评议。

\subsubsubsection{第五阶段:比较研究与政策建议}
基于中日两国观众问卷数据开展跨文化比较,提炼青年演员培养与观众期待的差异与共性,并据此形成面向产业与教育环节的政策建议要点,为中国青年演员培养提供参考。

\subsubsection{关键技术}

\subsubsubsection{多模态数据整合技术}
本研究涉及影视作品(视频)、媒体报道(文本)、观众评论(文本+社交媒体)、访谈录音(音频)等多模态数据。采用NVivo qualitative data analysis software进行数据管理和编码,通过交叉引用和矩阵查询功能,实现不同数据类型之间的关联分析。对于社交媒体大数据,采用Python爬虫技术和自然语言处理(NLP)方法进行情感分析和主题建模。图\ref{fig:data-pipeline}展示了多模态数据整合的技术路线。

\subsubsubsection{纵向案例分析技术}
采用时间序列分析和事件史分析(Event History Analysis)方法,识别滨边美波职业发展中的关键节点(如首次主演电影、首次转型成人角色、首次获奖等)。通过生存分析(Survival Analysis)方法,估计不同转型策略对职业 longevity 的影响。使用Stata或R软件进行统计分析。

\subsubsubsection{跨文化比较研究技术}
采用多组群结构方程模型(Multi-group SEM)方法,检验中日两国观众对青年演员评价的心理机制差异。通过测量等值性(Measurement Invariance)检验,确保跨文化比较的有效性。使用Mplus或AMOS软件进行分析。图\ref{fig:comparison-framework}展示了跨文化比较研究的分析框架。
\begin{figure}[htbp]
  \centering
  \includegraphics[width=0.85\textwidth]{figures/bbmb-80.jpg}
  \caption{多模态数据整合技术路线示意图(projects/NSFC\_Local/figures/)。}
  \label{fig:data-pipeline}
\end{figure}

\begin{figure}[htbp]
  \centering
  \includegraphics[width=0.85\textwidth]{figures/bbmb-mobile-129.jpg}
  \caption{跨文化比较研究分析框架示意图(projects/NSFC\_Local/figures/)。}
  \label{fig:comparison-framework}
\end{figure}


\subsubsubsection{质性-量化混合研究设计}
本研究采用解释性序列设计(Explanatory Sequential Design),先进行量化分析(作品统计与问卷调查),再进行质性分析(深度访谈与媒体话语分析),最后整合两类证据进行三角验证(Triangulation),以提高结论的可信度与可解释性。

\subsubsection{可行性分析}

\subsubsubsection{研究对象的可获得性}
滨边美波作为公开公众人物,其影视作品与媒体资料可通过合法渠道获取;研究团队已建立日文资料获取与核对流程,并具备必要的日语资料阅读与访谈能力。

\subsubsubsection{研究团队的学术基础}
申请人长期从事明星研究与文化产业研究,团队具备影视分析与社会调查等方法储备,能够胜任跨学科数据采集与模型分析任务。

\subsubsubsection{研究方法的成熟性}
本研究采用的纵向案例研究、文本分析、深度访谈与问卷调查均为成熟范式;团队已熟练掌握 NVivo、Stata 等工具,并通过预调研验证了流程可执行性。
