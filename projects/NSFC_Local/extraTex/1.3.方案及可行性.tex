\justifying

\subsubsection{研究方法}

\subsubsubsection{数据层}
\indent\setlength{\parindent}{2em}%首行缩进4字符
本项目采用"数据驱动 + 结构归纳偏置"相结合的方法,围绕 CNN 在医学影像分割/检出任务中的鲁棒学习展开。对多中心 CT/MRI 数据进行统一重采样、强度归一化与 ROI 裁剪,配合类别重采样与难例挖掘缓解类不均衡。

\subsubsubsection{模型层}
以 U-Net/nnU-Net 编码器—解码器为主干,结合残差块与注意力门控提升小病灶边界表征,并引入多尺度融合(FPN/ASPP)覆盖不同尺度结构。

\subsubsubsection{训练层}
构建可控强度的数据增强策略(几何/强度/伪影扰动),并结合 MixUp/CutMix 与一致性正则化提升跨设备泛化。

\subsubsubsection{评测层}
采用 Dice、Hausdorff95、AUC 等指标与分层交叉验证,开展消融与泛化对照实验,量化"结构设计/增强强度/数据规模"对性能的贡献。

\subsubsection{技术路线}

\subsubsubsection{数据准备}
\indent\setlength{\parindent}{2em}%首行缩进4字符
技术路线按"数据—模型—训练—验证—落地"五个环节推进。建立多中心数据接入与质量控制流程,完成标注规范与一致性检查,并构建训练/验证/外部测试集。

\subsubsubsection{基线构建}
实现可复现实验管线(预处理、训练、推理、评估),以 U-Net/ResNet 为基线对比。

\subsubsubsection{结构迭代}
围绕编码器主干、注意力门控与多尺度模块进行模块化替换与组合搜索,形成面向不同任务的 CNN 结构推荐。

\subsubsubsection{增强与鲁棒训练}
从常规增强扩展到 RandAugment、MixUp/CutMix、随机伪影与强度扰动,并引入一致性训练/伪标签策略提升域泛化。

\subsubsubsection{临床验证}
在外部中心数据上进行泛化验证与失败案例分析,结合不确定性估计输出风险提示,形成原型系统与可交付报告。

\subsubsection{关键技术}

\subsubsubsection{CNN 架构设计}
\indent\setlength{\parindent}{2em}%首行缩进4字符
编码器残差化、解码器跳连注意力门控、多尺度特征融合与深监督,使模型兼顾全局语义与边界细节。

\subsubsubsection{数据增强与强度标定}
建立"增强算子—强度—任务指标"的映射关系,避免过强增强引入伪影偏差。

\subsubsubsection{跨域泛化训练}
采用一致性正则化与分布扰动模拟,提升跨设备/跨中心的鲁棒性,并结合自适应阈值的伪标签迭代。

\subsubsubsection{不确定性与失败样本闭环}
基于 MC Dropout/深度集成等估计不确定性,驱动难例再训练与可解释风险提示。

\subsubsection{可行性分析}

\subsubsubsection{数据与任务可获得}
\indent\setlength{\parindent}{2em}%首行缩进4字符
可利用公开医学影像数据集(如分割/检出基准)并结合合作单位的真实临床数据,覆盖多中心分布差异。

\subsubsubsection{算法与工具链成熟}
基于 PyTorch、MONAI/nnU-Net 等生态可快速搭建基线与复现实验,便于开展结构与增强的系统化消融。

\subsubsubsection{算力与工程保障}
具备 GPU 训练环境与规范化实验管理(日志、可复现配置、版本控制),支撑大规模对照实验。

\subsubsubsection{风险可控}
针对小样本/噪声标注/域偏移等风险,预设数据清洗、弱监督与外部验证方案,并通过不确定性驱动的失败分析实现闭环改进。

\clearpage
