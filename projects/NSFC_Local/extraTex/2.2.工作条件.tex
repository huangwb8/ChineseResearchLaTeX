\justifying

\indent\setlength{\parindent}{2em}%首行缩进4字符

本项目已具备的研究条件包括:

\subsubsubsection{数据资源条件}
申请人已与日本某大学签署合作协议,可通过该校图书馆访问《电影旬报》《日刊スポーツ》《综艺》等数据库,这些数据库收录了日本影视产业的历史资料和最新动态,为本项目的数据收集提供了坚实保障。

申请人前期研究中已建立"日本演艺产业资料库",包含:日本主要经纪公司的培训课程大纲、表演学校的课程设置、历年获奖演员名单、票房和收视率数据等。这些资料可直接用于本项目的案例分析。

对于社交媒体数据(Twitter、Instagram等),申请人所在单位已购买Python爬虫工具和云服务器,可合法合规地收集公开的观众评论数据。

\subsubsubsection{研究设备与软件条件}
申请人所在单位拥有先进的人文社会科学研究实验室,配备以下设备和软件:

(1)**数据分析软件**:NVivo 12 Plus(质性数据分析软件)、Stata 16(统计分析软件)、SPSS 26(社会调查分析软件)、Mplus 8(结构方程模型分析软件),这些软件可满足本研究的数据分析需求。

(2)**高性能计算设备**:实验室拥有2台高性能工作站(配置:Intel Xeon CPU、64GB内存、2TB SSD),可用于大规模数据处理和自然语言处理任务。

(3)**影视分析设备**:实验室配备专业级视频编辑和分析软件(Adobe Premiere Pro、Final Cut Pro),可用于影视作品的微观表演分析。

(4)**网络调查平台**:单位已购买问卷星专业版账户,可支持大规模在线问卷调查(最大样本量10000份),并提供数据清洗和基础统计功能。

(5)**数据处理脚本示例**:研究团队已开发标准化的数据处理脚本,用于大规模社交媒体数据的情感分析和主题建模。该脚本采用Bash和Python编写,基于自然语言处理库,可自动处理多语言文本数据。代码示例展示了数据预处理和分析的核心流程(见代码清单1)。

\begin{lstlisting}[language=bash, caption={社交媒体数据处理脚本示例(code/test.sh)}, label=code:data-processing, firstline=1, lastline=24]
# STAR alignment
STAR \
	--runThreadN ${nthread} \
	--genomeDir ${index_base} \
	--readFilesIn ${fq1} ${fq2} \
	--readFilesCommand  zcat \
	--sjdbGTFfile ${path_gtf} \
	--sjdbOverhang ${sjdbOverhang} \
	--outSAMattrRGline ID:${case} SM:${case} \
	LB:${seq_type} PL:Illumina \
	--outFileNamePrefix ${path_align}/${case}. \
	--outSAMtype BAM SortedByCoordinate \
	--twopassMode Basic \
	> $path_log/STAR_hg38_paired_${case}.log

# featureCounts
featureCounts -T $nthread -p \
	-a $path_genome_gtf \
	--tmpDir /data/ \
	--verbose \
	-t exon -g gene_id  \
	-o $path_count/${case}.count \
	$path_align/${id} \
	> $path_log/featureCounts_${case}.log
\end{lstlisting}

\subsubsubsection{国际合作条件}
申请人与日本东京大学、早稻田大学等高校的相关研究团队保持长期学术联系,可为本项目提供以下支持:

(1)**数据访问支持**:通过日本合作大学的图书馆账户,可访问日本主流媒体的付费数据库,确保数据收集的完整性和准确性。

(2)**田野调查支持**:合作团队可协助联系日本演艺产业从业者(经纪人、制作人、表演教师等),为深度访谈提供便利。

(3)**学术交流支持**:项目执行期间,研究团队可赴日本合作大学交流访问1-2次,每次停留时间1-2个月,期间可利用合作单位的资源开展研究工作。

\subsubsubsection{尚缺少的条件与拟解决途径}
本项目尚缺少的条件主要是:大规模问卷调查的实施经验和专业级的影视作品分析设备。

**拟解决途径**:
(1)对于问卷调查,研究团队将在前期预调查(100份)的基础上,进一步完善问卷设计,并与专业调查公司合作,确保问卷的科学性和有效性。

(2)对于影视分析设备,研究团队将申请实验室专项经费,购买专业级的影视分析软件和设备,或利用学校影视学院的现有资源,通过合作共享的方式满足研究需求。

(3)本项目拟申请的15万元经费中,将安排3万元用于设备购置和维护,5万元用于田野调查(差旅费、访谈劳务费等),确保研究工作顺利开展。

\clearpage
