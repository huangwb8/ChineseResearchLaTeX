\justifying
\NSFCBodyText

\subsubsection{工作条件}

本项目已具备的研究条件包括数据资源、研究设备与软件条件以及国际合作基础,能够支撑资料采集、编码分析与跨媒介数据建模等工作。

\subsubsubsection{数据资源条件}
申请人已与日本某大学签署合作协议,可通过该校图书馆访问《电影旬报》《日刊スポーツ》《综艺》等数据库,这些数据库收录了日本影视产业的历史资料和最新动态,为本项目的数据收集提供了坚实保障。

申请人前期研究已整理形成“日本演艺产业资料库”(培训课程、课程设置、获奖名单、票房与收视率等),并具备合规采集公开社交媒体评论数据的技术与算力条件。

\subsubsubsection{研究设备与软件条件}
申请人所在单位拥有先进的人文社会科学研究实验室,配备以下设备和软件:

(1)\textbf{数据分析软件}:NVivo 12 Plus(质性数据分析软件)、Stata 16(统计分析软件)、SPSS 26(社会调查分析软件)、Mplus 8(结构方程模型分析软件),这些软件可满足本研究的数据分析需求。

(2)\textbf{高性能计算设备}:实验室拥有2台高性能工作站(配置:Intel Xeon CPU、64GB内存、2TB SSD),可用于大规模数据处理和自然语言处理任务。

(3)\textbf{影视分析设备}:实验室配备专业级视频编辑和分析软件(Adobe Premiere Pro、Final Cut Pro),可用于影视作品的微观表演分析。

(4)\textbf{数据处理脚本示例}:研究团队已开发标准化的数据处理脚本,用于大规模社交媒体数据的情感分析和主题建模。该脚本采用 Bash 和 Python 编写,基于自然语言处理库,可自动处理多语言文本数据。代码清单\ref{code:data-processing}展示了数据预处理与分析流程的关键片段。

\lstinputlisting[
  language=bash,
  firstline=1,
  lastline=10,
  caption={社交媒体数据处理脚本示例(code/test.sh)},
  label={code:data-processing}
]{code/test.sh}

\subsubsubsection{国际合作条件}
申请人与日本东京大学、早稻田大学等高校的相关研究团队保持长期学术联系,可为本项目提供以下支持:

(1)\textbf{数据访问支持}:通过日本合作大学的图书馆账户,可访问日本主流媒体的付费数据库,确保数据收集的完整性和准确性。

(2)\textbf{田野调查支持}:合作团队可协助联系日本演艺产业从业者(经纪人、制作人、表演教师等),为深度访谈提供便利。

(3)\textbf{学术交流支持}:项目执行期间,研究团队可赴日本合作大学交流访问1-2次,每次停留时间1-2个月,期间可利用合作单位的资源开展研究工作。

\subsubsubsection{尚缺少的条件与拟解决途径}
本项目尚缺少的条件主要是:大规模问卷调查的实施经验和专业级的影视作品分析设备。

\textbf{拟解决途径}:
(1)对于问卷调查,研究团队将在前期预调查(100份)的基础上,进一步完善问卷设计,并与专业调查公司合作,确保问卷的科学性和有效性。

(2)对于影视分析设备,研究团队将申请实验室专项经费,购买专业级的影视分析软件和设备,或利用学校影视学院的现有资源,通过合作共享的方式满足研究需求。
