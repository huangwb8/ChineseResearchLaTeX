\justifying

\subsubsection{研究内容}

\subsubsubsection{网络架构设计}
\indent\setlength{\parindent}{2em}%首行缩进4字符
围绕医疗影像(如 CT/MRI/超声)中"病灶分割—检出—分型"的临床需求,本项目构建端到端卷积神经网络(CNN)分析框架:以 U-Net/nnU-Net 风格的编码器—解码器结构为主干,通过残差/密集连接提升梯度传播,并在跳连处加入注意力门控以抑制背景噪声。

\subsubsubsection{多尺度特征融合}
针对多尺度病灶与器官边界,引入多分辨率特征融合(FPN/ASPP)并结合深监督,增强对细小结构的敏感性。

\subsubsubsection{泛化能力提升}
考虑跨设备与跨中心分布偏移,采用强数据增强与一致性正则化(如 RandAugment、MixUp/CutMix、随机伪影与强度扰动),提升泛化能力。上述方法将以"分割精度 + 诊断判别"双指标联合优化,形成可复用的医疗影像智能分析流程。

\subsubsection{研究目标}

\subsubsubsection{总体目标}
\indent\setlength{\parindent}{2em}%首行缩进4字符
研究目标是形成可在多中心数据上稳定工作的 CNN 医疗影像分析模型与训练策略。

\subsubsubsection{具体目标}
(1)建立具备可解释性的网络结构设计准则(主干选择、注意力/多尺度模块配置、深监督策略);(2)构建覆盖真实成像扰动的增强策略库,并给出增强强度—性能收益的定量关系;(3)在公开数据与自建数据上验证鲁棒性与可迁移性,力争在 Dice、AUC 等指标上达到国内同类方法领先水平。

\subsubsection{拟解决的关键问题}

\subsubsubsection{架构设计问题}
\indent\setlength{\parindent}{2em}%首行缩进4字符
CNN 架构在医疗影像小样本、类别不均衡场景下的有效表征学习机制与结构选型。

\subsubsubsection{数据增强问题}
数据增强在"提高泛化"与"引入伪影偏差"之间的平衡机理,以及增强策略的可迁移性。

\subsubsubsection{域偏移问题}
跨中心域偏移下的稳定训练与不确定性评估方法,确保模型输出可用于临床决策支持。

\clearpage
