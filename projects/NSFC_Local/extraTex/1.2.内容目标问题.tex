\justifying

\subsubsection{研究内容}

\subsubsubsection{滨边美波职业生涯阶段划分}
\indent\setlength{\parindent}{2em}%首行缩进4字符
本研究将滨边美波的职业生涯划分为四个阶段:童星期(2011-2015)、转型探索期(2016-2019)、实力确立期(2020-2022)、多元化发展期(2023至今)。通过对每个时期的代表作品、角色类型、媒体曝光度、业内评价等维度进行系统分析,揭示青年演员职业发展的阶段性特征和转型关键点。

\subsubsubsection{表演风格演变与技能建构}
重点分析滨边美波在不同类型角色中的表演策略,包括情感表达方式、台词处理技巧、肢体语言运用等微观表演层面。通过对比其在《狂赌之渊》中的夸张表演与《我想被高中生杀死》中的内敛演技,探讨演员如何在商业作品与艺术作品间进行风格调适。同时,考察她在舞台剧(如《悲惨世界》)中获得的技能训练如何反哺影视表演。

\subsubsubsection{产业生态系统分析}
从经纪公司、制作方、观众、媒体等多主体互动角度,分析日本演艺生态系统中青年演员的生存环境。具体包括:经纪公司的培养机制(如东宝艺能学校的训练体系)、作品选择策略、媒体报道框架(如"清纯派"形象的建构与解构)、粉丝社群的文化实践等。通过访谈、问卷调查、网络民族志等方法,获取产业内部的一手资料。

\subsubsection{研究目标}

\subsubsubsection{总体目标}
\indent\setlength{\parindent}{2em}%首行缩进4字符
构建日本青年演员职业发展的理论分析框架,以滨边美波为典型案例,揭示日本演艺生态系统的人才培养机制和转型规律,为中国青年演员的职业发展提供可借鉴的经验。

\subsubsubsection{具体目标}
(1)完成滨边美波2011-2025年所有影视作品的系统梳理和分类编码,建立演员作品数据库;(2)提出"童星转型期"(Child Star Transition Phase)的理论模型,明确转型期的关键节点、风险因素和成功要素;(3)基于多主体访谈和行业数据,绘制日本演艺生态系统运作机制图谱;(4)在比较研究视野下,总结中日两国在青年演员培养方面的异同,提出政策建议。

\subsubsection{拟解决的关键科学问题}

\subsubsubsection{青年演员职业发展的阶段性特征问题}
\indent\setlength{\parindent}{2em}%首行缩进4字符
如何科学划分青年演员的职业发展阶段?各阶段之间有何本质区别?转型期的触发机制和成功要素是什么?本研究将通过纵向案例分析,结合职业生涯发展理论,构建青年演员职业发展的阶段性模型。

\subsubsubsection{表演风格的习得与调适问题}
青年演员如何在商业作品与艺术作品间进行表演风格的调适?不同表演技能(影视、舞台、配音)之间如何相互迁移和促进?本研究将通过微观表演分析,结合技能习得理论,揭示演员表演风格的建构机制。

\subsubsubsection{演艺生态系统的结构功能问题}
日本演艺生态系统中各主体(经纪公司、制作方、观众、媒体)如何相互作用以形塑演员的职业轨迹?这种生态系统与中国相比有何异同?本研究将通过社会网络分析和制度比较研究,揭示东亚语境下演艺产业的运作逻辑。

\clearpage
