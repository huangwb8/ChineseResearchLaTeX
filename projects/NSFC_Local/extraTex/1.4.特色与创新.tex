\justifying
\NSFCBodyText

本项目的特色与创新点主要体现在:

(1)\textbf{理论创新}:提出"童星转型期"(Child Star Transition Phase)概念模型,填补青年演员职业发展理论的空白。现有研究多聚焦于成人明星或独立研究童星现象,缺乏对童星向成人演员转型过程的系统性理论阐释。本研究基于滨边美波的纵向案例,结合职业生涯发展理论和角色转型理论,构建包含"触发因素-转型策略-成功要素-风险控制"四维度的分析框架,为理解青年演员成长规律提供新的理论工具。

(2)\textbf{方法创新}:采用"个人-产业-文化"三维分析框架,打破传统明星研究中"去语境化"的局限。现有研究多从单一视角(如文本分析、受众研究)出发,难以全面把握演员职业发展的复杂性。本研究将微观层面的表演风格分析、中观层面的产业生态系统分析、宏观层面的跨文化比较研究相结合,形成多层次、多方法的研究设计,提高研究结论的解释力和普适性。

(3)\textbf{技术创新}:建立多模态数据整合分析平台,实现影视作品(视频)、媒体报道(文本)、观众评论(社交媒体)、深度访谈(音频)的关联分析。采用自然语言处理(NLP)方法对大规模观众评论进行情感分析和主题建模,结合社会网络分析(SNA)方法揭示演艺产业内部的结构关系。这些技术的应用为明星研究提供了新的方法论工具。

(4)\textbf{跨学科创新}:融合文化社会学、媒体研究、表演艺术理论、职业生涯发展理论等多学科视角,形成跨学科的研究范式。现有研究多局限于单一学科内部(如电影研究、社会学),缺乏跨学科对话。本研究通过理论和方法上的创新性整合,为明星研究领域的跨学科发展提供示范。

(5)\textbf{实践应用创新}:基于日本经验,提出中国青年演员培养的政策建议。现有研究多停留在理论层面,缺乏对实践的关注。本研究通过比较中日两国在人才培养机制、产业制度、观众期待等方面的差异,提炼可供中国借鉴的日本经验,为中国演艺产业的可持续发展提供智力支持。
