%!TEX program = xelatex

% 国家自然科学基金NSFC-青年项目-申请书正文模板(2024年版)

% 测试环境:[使用VSCode编写LaTeX - 知乎](https://zhuanlan.zhihu.com/p/38178015)

% 编译顺序: xelatex -> bibtex -> xelatex -> xelatex

% 参考 & 鸣谢:
% [Ruzim/NSFC-application-template-latex: 国家自然科学基金申请书正文(面上项目)LaTeX 模板(非官方)](https://github.com/Ruzim/NSFC-application-template-latex)
% [Readon/NSFC-application-template-latex: 国家自然科学基金申请书正文(面上项目)模板(非官方)](https://github.com/Readon/NSFC-application-template-latex)
% [Aligning equations with amsmath - Overleaf, Online LaTeX Editor](https://www.overleaf.com/learn/latex/Aligning_equations_with_amsmath):各种格式的公式写法


\documentclass[12pt,UTF8,AutoFakeBold=3,a4paper]{ctexart} %默认小四号字。允许楷体粗体。

%%%%=================================系统设置=================================%%%%

\usepackage[english]{babel} %支持混合语言
\usepackage{xcolor}
\usepackage{graphicx} 
\usepackage{amsmath} %更多数学符号
\usepackage{wasysym}
\usepackage{geometry} %改尺寸
\usepackage{fontspec} %字体包
\usepackage{setspace}
\usepackage{amsfonts} % 数学公式增强
\usepackage{xeCJK} % 调整字间距
\usepackage{titlesec} 
\usepackage{ragged2e} % 正文两端对齐所需的包。在需要两端对齐的文字前面添加\justifying即可。


%%%————————————————————————————————边距
% \CJKsetecglue命令定义了CJK字符与其他字符之间的空白,不只限于数字字符,也适用于字母等其他非CJK字符。如果你想要更精细的控制,可能需要借助其它工具或者宏包。
% \hskip0.15em plus0.05em minus 0.05em表示基础间距为0.15em,可增加至0.2em,也可减少至0.1em以便满足排版需求。
% 页边距和字距的某个组合应该可以完美复制Word。但目前没调出来。
\geometry{left=3.20cm,right=3.14cm,top=2.54cm,bottom=2.54cm} % 页边距
\CJKsetecglue{\hskip 0.3em plus 0.05em minus 0.1em} % 数字与CJK字符的间距
% left=3.20cm,right=3.13-3.14cm

%%%————————————————————————————————颜色
\definecolor{MsBlue}{RGB}{0,112,192} %Ms Word 的蓝色和latex xcolor包预定义的蓝色不一样。通过屏幕取色得到。


%%%————————————————————————————————对象名
\addto\captionsenglish{
    \renewcommand{\contentsname}{目录}
    \renewcommand{\listfigurename}{插图目录}
    \renewcommand{\listtablename}{表格}
    %\renewcommand{\refname}{\sihao 参考文献}
    \renewcommand{\refname}{\sihao \kaishu \leftline{参考文献}} %这几个字默认字号稍大,改成四号字,楷书,居左(默认居中) 根据喜好自行修改,官方模板未作要求
    \renewcommand{\abstractname}{摘要}
    \renewcommand{\indexname}{索引}
    \renewcommand{\tablename}{表}
    \renewcommand{\figurename}{图}
} %把Figure改成‘图’,reference改成‘参考文献’。如此处理是为了避免和babel包冲突。


%%%————————————————————————————————字体
\newcommand{\chuhao}{\fontsize{42pt}{\baselineskip}\selectfont}
\newcommand{\xiaochuhao}{\fontsize{36pt}{\baselineskip}\selectfont}
\newcommand{\yihao}{\fontsize{26pt}{\baselineskip}\selectfont}
\newcommand{\erhao}{\fontsize{22pt}{\baselineskip}\selectfont}
\newcommand{\xiaoerhao}{\fontsize{18pt}{\baselineskip}\selectfont}
\newcommand{\sanhao}{\fontsize{16pt}{\baselineskip}\selectfont}
\newcommand{\sihao}{\fontsize{14pt}{\baselineskip}\selectfont}
\newcommand{\xiaosihao}{\fontsize{12pt}{\baselineskip}\selectfont}
\newcommand{\yemeizihao}{\fontsize{11pt}{\baselineskip}\selectfont}
\newcommand{\wuhao}{\fontsize{10.5pt}{\baselineskip}\selectfont}
\newcommand{\xiaowuhao}{\fontsize{9pt}{\baselineskip}\selectfont}
\newcommand{\liuhao}{\fontsize{7.875pt}{\baselineskip}\selectfont}
\newcommand{\qihao}{\fontsize{5.25pt}{\baselineskip}\selectfont}
%字号对照表
%二号 21pt
%四号 14
%小四 12
%五号 10.5

%解决阿拉伯数字字体问题
\newfontfamily\kaitichar{KaiTi}[AutoFakeBold] %定义楷体英文和数字。
\NewDocumentCommand \templatefont   { } { \kaishu \kaitichar } %使用模板字体代替直接使用楷书。
\setmainfont{Times New Roman}


%%%————————————————————————————————行距
%设置行距 1.5倍
\renewcommand{\baselinestretch}{1.5}
\XeTeXlinebreaklocale "zh" % 中文断行


%%%————————————————————————————————图片与表格

% 对于表格,推荐使用Excel的插件Excel2LaTeX自动生成LaTeX代码。详见: https://www.zhihu.com/question/307970489/answer/2305355098

%图片
\usepackage{graphicx} %插图宏包
\usepackage{float} %设置图片浮动位置的宏包
\usepackage{subfigure} %插入多图时用子图显示的宏包
\usepackage[section]{placeins} %避免浮动体跨过\section
\usepackage{enumerate} %设置列表环境的包
\usepackage{enumitem} % 增强自定义Bullet(即各种List前的那个标记)
\setlist[enumerate]{
  label={\templatefont \bfseries \hspace{1em} \color{MsBlue}(\arabic*)},
  leftmargin=0em, % 列表的左边界
  itemindent=4em, % item首行缩进
  itemsep=0em, % 列表项之间的垂直间距。
  labelsep=0.1pt,  % label与content的距离
  parsep=0em,  % 两个段落之间的垂直间距。效果上与itemsep类似。
  topsep=0em % 列表与上一段对象的垂直间距。
  % after=\vspace{5pt}
}
\newcommand{\itemtitlefont}[1]{\textbf{\color{MsBlue} #1}} % 定义item小标题字体;
\usepackage{caption}
\captionsetup[figure]{name={图},font={footnotesize,stretch=1.25},labelsep=period,labelfont=bf, singlelinecheck=off, justification=centering} 

%表格
\usepackage{booktabs} %表格
\usepackage{tabularx} %表格宽度
\usepackage{multirow} %多行合并
\usepackage{longtable} % latex 多页显示同一表格
\usepackage{tabu} % 使用longtabu(tabularx + longtable)将长文本在单元格多行显示。
\captionsetup[table]{name={表},font={footnotesize,stretch=1.25},labelsep=period,labelfont=bf, singlelinecheck=off, justification=centering} 


%%%————————————————————————————————超链接
\usepackage[
	colorlinks,
	urlcolor=MsBlue, % 超链接的颜色。默认是芭比粉
	linkcolor=black,
	anchorcolor=black,
	citecolor=black,
	CJKbookmarks=True
]{hyperref}


%%%————————————————————————————————标题计数

% [latex 标题、段落及行距 - 简书](https://www.jianshu.com/p/d7848f815e5f/)
% [LaTeX 中的行间距](https://yinguobing.com/linespace-in-latex/)

\usepackage{titlesec}

\setcounter{secnumdepth}{4} %显示3级目录,然后再具体定制不同的目录
% \setcounter{tocdepth}{4}

% Define section font
\newcommand{\sectionzihao}{\fontsize{14pt}{\baselineskip}\selectfont}
\newcommand{\subsectionzihao}{\fontsize{14pt}{14pt}\selectfont}
\newcommand{\subsubsectionzihao}{\fontsize{13.5pt}{20pt}\selectfont}

% Custom format for section
\titleformat{\section}
  {\color{MsBlue} \sectionzihao \templatefont} % format
  {\hspace{1.45em}  } % label
  {0pt} % separation
  {}   % before-code

% Custom format for subsection
\titleformat{\subsection}
  {\color{MsBlue} \subsectionzihao \templatefont \linespread{1}} % format
  {} % label
  {0pt} % separation
  {}   % before-code

% Custom format for subsubsection
\titleformat{\subsubsection}
  {\color{MsBlue} \subsubsectionzihao \templatefont \bfseries} % format
  {\hspace{1.2em}  \textnormal{\templatefont \arabic{subsection}.\arabic{subsubsection}}} % label
  {0.5em} % separation
  {}   % before-code

% Custom format for subsubsubsection
\titleclass{\subsubsubsection}{straight}[\subsubsection]
\newcounter{subsubsubsection}[subsubsection]
\renewcommand\thesubsubsubsection{(\arabic{subsubsection})}
\titleformat{\subsubsubsection}
  {\templatefont \bfseries} % format \color{MsBlue} 
  {\hspace{1em} (\arabic{subsubsubsection})} % label
  {0.5pt} % separation
  {}   % before-code
\makeatletter
\def\toclevel@subsubsubsection{4}
\def\l@subsubsubsection{\@dottedtocline{4}{7.0em}{4em}}
\makeatother

% Define line stretch for section
\titlespacing*{\section}
  {0pt} % Left indentation
  {0pt plus 0pt minus 0pt} % Space before the title
  {0pt plus 0pt minus 0pt} % Space after the title

% Define line stretch for subsection
\titlespacing*{\subsection}
  {0pt} % Left indentation
  {0pt plus 0pt minus 0pt} % Space before the title
  {0pt plus 0pt minus 0pt} % Space after the title

% Define line stretch for subsubsection
\titlespacing*{\subsubsection}
  {0pt} % Left indentation
  {0pt plus 0pt minus 0pt} % Space before the title
  {0pt plus 0pt minus 0pt} % Space after the title

% Define line stretch for subsubsection
\titlespacing*{\subsubsubsection}
  {0pt} % Left indentation
  {0pt plus 0pt minus 0pt} % Space before the title
  {0pt plus 0pt minus 0pt} % Space after the title


%%%————————————————————————————————代码
\usepackage{listings} %代码块
\usepackage{xcolor}%代码块
\renewcommand{\lstlistingname}{Code}

% 参考:
% https://www.youtube.com/watch?v=qxkQgG1Y0bY
% https://zhuanlan.zhihu.com/p/65441079

% \lstset{
% 	% language=bash,  %代码语言使用的是matlab
% 	frame=shadowbox, %把代码用带有阴影的框圈起来
% 	rulesepcolor=\color{red!20!green!20!blue!20},%代码块边框为淡青色
% 	keywordstyle=\color{blue!90}\bfseries, %代码关键字的颜色为蓝色,粗体
% 	commentstyle=\color{red!10!green!70}\textit,% 设置代码注释的颜色
% 	showstringspaces=true,%不显示代码字符串中间的空格标记
% 	numbers=left, % 显示行号
% 	numberstyle=\tiny,    % 行号字体
% 	stringstyle=\ttfamily, % 代码字符串的特殊格式
% 	breaklines=false, %对过长的代码自动换行
% 	extendedchars=false,  %解决代码跨页时,章节标题,页眉等汉字不显示的问题
% 	escapebegin=\begin{CJK*},escapeend=\end{CJK*},% 代码中出现中文必须加上,否则报错
% 	texcl=true
% }

\lstdefinestyle{codestyle01}{
	backgroundcolor=\color{gray!12},
	%basicstyle=\ttfamily\small,
	basicstyle=\zihao{-5}\ttfamily,
	commentstyle=\color{green!60!black},
	keywordstyle=\color{magenta},
	stringstyle=\color{blue!50!red},
	showstringspaces=false,
	numbers=none,
	numberstyle=\footnotesize\color{gray},
	numbersep=10pt,
	stepnumber=2, %左边数字的步幅
	tabsize=4,
	frame=TB %顶部和底部有两条竖线。
	% frame=tblr
	% frame=L,
	% framerule=1pt,
	% rulecolor=\color{red}
}


%%%————————————————————————————————其它设置

% 将每页的图片置顶展示(而不是居中)
\makeatletter	
	\setlength{\@fptop}{0pt}
\makeatother

% 去除页码
\pagestyle{empty}



%%%%============================正  文=============================%%%%
\begin{document}

% \setcounter{page}{4} % 手动设置第一页的页码为4

\begin{center}
{\sanhao \kaishu \bfseries 报告正文}
\end{center}

{\sihao \kaishu 参照以下提纲撰写,要求内容翔实、清晰,层次分明,标题突出。{\color{MsBlue} \bfseries 请勿删除或改动下述提纲标题及括号中的文字。}}
\vskip -5mm

\section{{\bfseries(一)立项依据与研究内容}(建议8000字以内):} %2024

\subsection{\hspace{1.45em} 1.~{\bfseries 项目的立项依据}(研究意义、国内外研究现状及发展动态分析,需结合科学研究发展趋势来论述科学意义;或结合国民经济和社会发展中迫切需要解决的关键科技问题来论述其应用前景。附主要参考文献目录);}

{\justifying
\NSFCBodyText

\subsubsection{研究背景}

\subsubsubsection{案例概述}
\indent 刘亦菲(1987年生)作为中国演艺产业代表性人物,其职业生涯从2002年持续至今,跨越童星、电视剧演员、电影演员、国际巨星等多个阶段。2002年出演《金粉世家》进入演艺圈,2005年凭借《仙剑奇侠传》中"赵灵儿"一角成为现象级偶像。2006-2008年主演《神雕侠侣》《天龙八部》等金庸武侠剧,奠定"古装女神"地位。2012年起向电影转型,主演《铜雀台》《露水红颜》等作品。2020年主演迪士尼真人版《花木兰》,成为首位担纲好莱坞A级制作女主角的华人演员,实现国际化突破。

\subsubsubsection{研究价值}
\indent 刘亦菲的职业发展路径为理解中国娱乐产业变迁、女性艺人国际化策略及跨文化形象管理提供了极具价值的案例。其职业生涯跨越中国影视产业从电视剧主导向电影主导、从国内市场向国际市场转型的关键期(2005-2025)。作为女性艺人,她在"古装女神"到"国际巨星"的转型中面临的挑战具有典型性。《花木兰》项目为研究华人演员好莱坞突破机制提供了珍贵实证。

\subsubsubsection{研究问题}

\subsubsection{国内外研究现状}

\subsubsubsection{职业转型理论}
\indent 目前研究主要集中三个方向:Smith(2020)提出"职业锚理论"\cite{Smith2020};Super(1957)划分职业发展五阶段\cite{Super1957},但缺乏中国演艺产业语境研究。

\subsubsubsection{国际化路径研究}
\indent Johnson(2019)提出"本土-区域-全球"三阶段国际化模型\cite{Johnson2019};Lee(2018)分析亚洲演员好莱坞突破策略\cite{Lee2018},但对华人女演员的文化适应机制研究不足。

\subsubsubsection{品牌形象管理}

\subsubsection{现有研究的局限性}

\subsubsubsection{理论层面}
\indent 现有研究存在以下不足:
\begin{enumerate}[label=\ssssubtitle{\arabic*}, leftmargin=*, itemsep=0.2em]
  \item 缺乏完整职业生命周期的纵向案例研究。
  \item 对华人演员好莱坞突破机制探讨不足。
  \item 忽视女性艺人在跨文化转型中的特殊挑战。
\end{enumerate}

\subsubsubsection{方法层面}

\subsubsection{研究切入点}

\subsubsubsection{案例独特性}
\indent 刘亦菲的职业轨迹具有独特研究价值:(1)转型涵盖"童星→古装女神→电影演员→国际巨星"完整链条;(2)成功突破好莱坞A级制作,成为华人女演员国际化标杆;(3)职业发展与中国影视产业国际化进程高度同步。

\subsubsubsection{研究代表性}
}

% 参考文献
{\clearpage
\bibliographystyle{bibtex-style/gbt7714-nsfc.bst}
\bibliography{references/myexample}}


\subsection{\hspace{1.45em} 2.~{\bfseries 项目的研究内容、研究目标,以及拟解决的关键科学问题}(此部分为重点阐述内容);}

{\begingroup
\justifying
\setlength{\parindent}{2em}% 首行缩进 2 字符

\subsubsection{研究内容}
% 建议:拆成 3–4 条"研究内容",每条都写清:任务→方法→数据/实验→输出/里程碑。

\textbf{【演示内容:基于佐佐木希案例的虚拟研究课题】}

\begin{enumerate}
  \item \itemtitlefont{研究内容 1:佐佐木希职业生涯纵向追踪与阶段划分}\par
  \textbf{任务:}系统梳理佐佐木希从2005年模特出道到2025年的完整职业轨迹,识别职业发展的关键节点和阶段特征。\par
  \textbf{方法:}采用历史文献分析法、纵向案例研究法和时间序列分析法,系统收集其影视作品、媒体报道、社交媒体数据。\par
  \textbf{数据/实验:}收集2005-2025年间的影视作品数据库(日本电影数据库、电视剧评分网站)、社交媒体数据(Twitter/X、Instagram)、媒体报道(新闻文本、杂志采访),共计约5000条数据记录。通过时间序列分析识别职业发展的三个阶段:颜值驱动期(2005-2008)、转型适应期(2008-2015)、实力驱动期(2015-2025)。\par
  \textbf{输出/里程碑:}第1年末完成纵向追踪数据集构建和阶段划分模型,发表案例研究论文1篇。\NSFCBlankPara

  \item \itemtitlefont{研究内容 2:职业转型影响因素的多模态数据分析}\par
  \textbf{任务:}识别并量化从模特向演员转型的关键影响因素,构建职业转型成功要素指标体系。\par
  \textbf{方法:}采用多模态数据融合、机器学习(随机森林、梯度提升树)和文本挖掘(情感分析、主题模型)方法。\par
  \textbf{数据/实验:}整合影视作品评分(豆瓣、IMDb)、社交媒体互动数据(点赞数、转发数、评论情感)、媒体报道主题分布。通过特征工程提取30个候选影响因素,使用机器学习模型识别关键特征并计算权重。\par
  \textbf{输出/里程碑:}第2年中完成影响因素指标体系构建,开发开源数据分析工具包1个,发表方法学论文1-2篇。\NSFCBlankPara

  \item \itemtitlefont{研究内容 3:职业发展演化动力学模型构建与验证}\par
  \textbf{任务:}基于佐佐木希案例,构建艺人职业发展的三阶段演化动力学模型,并通过其他日本演员案例进行验证。\par
  \textbf{方法:}采用微分方程建模、系统动力学和元分析相结合的方法。\par
  \textbf{数据/实验:}构建"影响力—时间"微分方程模型,刻画颜值驱动期和实力驱动期的不同增长曲线。收集20位日本演员的辅助案例数据进行模型验证,计算模型的预测准确率和泛化能力。\par
  \textbf{输出/里程碑:}第3年末完成理论模型构建和验证,发表理论论文1-2篇,申请软件著作权1项。\NSFCBlankPara

  \item \itemtitlefont{研究内容 4:日本演员职业发展标准数据集构建}\par
  \textbf{任务:}制定数据采集标准,整合多源数据,构建可复用的日本演员职业发展标准数据集。\par
  \textbf{方法:}采用数据标准化、实体对齐和多源数据融合技术。\par
  \textbf{数据/实验:}整合日本电影数据库、电视剧数据库、社交媒体API、新闻媒体档案等6个数据源。制定数据标注规范(作品类型、角色重要性、影响力指标等),构建包含50位演员、覆盖1970-2025年的标准化数据集。\par
  \textbf{输出/里程碑:}第3年末完成数据集构建并公开发布,发表数据论文1篇。\NSFCBlankPara
\end{enumerate}

【示例:图片插入】图 \ref{fig:framework-design} 展示了职业发展演化模型的设计思路。

\begin{figure}[!ht]
    \begin{center}
        \includegraphics[width=0.75\linewidth]{figures/zzmx-mobile-105.jpg}
        \caption{职业发展演化动力学模型示意图。本图展示了研究内容3中三阶段演化模型的构建思路,包括颜值驱动期、转型适应期、实力驱动期的动力学方程、关键状态变量(影响力、作品数量、观众认可度)以及阶段转换的临界条件。红色箭头表示阶段转换方向,蓝色曲线表示影响力演化路径。}
        \label{fig:framework-design}
    \end{center}
\end{figure}

\subsubsection{研究目标与可验证指标}
% 建议:目标要"可度量",指标要"可验证",并与研究内容对应。

\textbf{【示例内容】}

\begin{enumerate}
  \item \itemtitlefont{总体目标:}\par
  建立全新的理论体系,突破现有方法的性能瓶颈,构建可复用的研究平台,推动该领域的科学进步和技术发展。

  \item \itemtitlefont{关键指标:}\par
  \textbf{理论层面:}建立完整理论框架,关键理论指标优于现有方法30\%以上;发表高水平SCI论文3-5篇,其中TOP期刊论文不少于2篇。\par
  \textbf{方法层面:}算法计算效率提升50\%以上,精度保持或优于现有方法;开源工具包获得社区广泛使用。\par
  \textbf{应用层面:}成果在2-3个实际场景中得到验证和应用;申请发明专利2-3项,软件著作权1-2项。\NSFCBlankPara

  \item \itemtitlefont{预期产出:}\par
  \textbf{学术论文:}SCI论文3-5篇,其中国际TOP期刊2-3篇。\par
  \textbf{知识产权:}发明专利2-3项,软件著作权1-2项。\par
  \textbf{开源工具:}算法工具包1个,数据分析平台1个,标准数据集1-2个。\par
  \textbf{人才培养:}培养博士研究生1-2名,硕士研究生2-3名。\par
  \textbf{学术交流:}参加国际学术会议2-3次,做特邀报告1-2次。
\end{enumerate}

\subsubsection{拟解决的关键问题(与立项依据保持一致)}

\textbf{【示例内容】}
本项目拟解决的关键问题与立项依据中提出的问题一一对应:

\begin{enumerate}
  \item \itemtitlefont{问题 1:如何建立高精度、高效率的理论模型?}\par
  针对该问题,本项目将引入创新的数学建模方法,突破传统理论模型的假设限制。具体而言,将采用XXX理论作为基础,结合YYY方法,构建能够准确描述复杂系统行为的理论框架。该问题的解决将为后续算法设计和应用验证奠定坚实的理论基础。\NSFCBlankPara

  \item \itemtitlefont{问题 2:如何设计兼顾效率和精度的算法框架?}\par
  针对该问题,本项目将设计全新的算法架构,通过优化数据结构和计算流程,实现效率和精度的双重提升。具体策略包括:采用并行计算技术提升效率,引入自适应机制保持精度,利用GPU加速等方法突破性能瓶颈。\NSFCBlankPara

  \item \itemtitlefont{问题 3:如何构建标准化、可复用的数据资源?}\par
  针对该问题,本项目将制定数据标准化规范,建立数据质量评估体系,构建可复用的数据共享平台。具体措施包括:制定数据采集和预处理标准,建立数据标注和质量控制流程,开发数据管理和共享平台。\NSFCBlankPara
\end{enumerate}

\endgroup
}

\subsection{\hspace{1.45em}  3.~{\bfseries 拟采取的研究方案及可行性分析}(包括研究方法、技术路线、实验手段、关键技术等说明);}

{\justifying

\subsubsection{研究方法}

\subsubsubsection{纵向案例研究法}
\indent\setlength{\parindent}{2em}%首行缩进4字符
本研究采用纵向案例研究法,以滨边美波2011-2025年的职业生涯为时间轴,系统梳理其影视作品、媒体报道、业内评价等多维度数据。通过时间序列分析,识别其职业发展的关键节点、转型期特征和风格演变轨迹。该方法适合于探究"如何"和"为什么"类型的因果问题,能够深入揭示青年演员职业发展的动态过程。

\subsubsubsection{文本分析法}
对滨边美波的影视作品进行微观表演分析,重点关注情感表达、台词处理、肢体语言等表演要素。同时,对日本主流媒体(如《日刊スポーツ》《综艺》《电影旬报》)关于滨边美波的报道进行话语分析,解构媒体报道框架和形象建构策略。结合符号学分析方法,探讨演员形象的文化意义和社会功能。

\subsubsubsection{深度访谈法}
选取日本演艺产业的关键从业者进行半结构化深度访谈,包括:经纪公司经纪人、影视制作方、表演指导教师、资深演员等。访谈内容涵盖:人才培养机制、作品选择策略、转型期支持体系、行业评价标准等。通过访谈获取产业内部的一手资料,弥补公开资料的不足。

\subsubsubsection{问卷调查法}
面向日本和中国影视观众设计问卷调查,了解:观众对滨边美波的认知和评价、对其表演风格的接受度、对其职业转型的态度等。通过中日对比,揭示不同文化语境下观众对青年演员的期待差异。采用分层抽样方法,确保样本代表性。

\subsubsection{技术路线}

\subsubsubsection{第一阶段:文献梳理与理论构建}
\indent\setlength{\parindent}{2em}%首行缩进4字符
系统梳理明星研究(Star Studies)、职业生涯发展理论(Career Development Theory)、文化社会学等领域的经典文献,构建本研究的理论分析框架。提出"童星转型期"(Child Star Transition Phase)的概念模型,明确关键变量和分析维度。完成研究设计和伦理审查申请。

\subsubsubsection{第二阶段:数据收集与数据库建立}
收集滨边美波2011-2025年所有影视作品(电影、电视剧、舞台剧、配音作品),建立包含以下字段的作品数据库:作品名称、类型、播出年份、角色类型、制作公司、收视率/票房、获奖情况等。同时,收集媒体报道(约500篇)、观众评论(网络爬虫,约2000条)、业内访谈(约20人)等多模态数据。

\subsubsubsection{第三阶段:作品分析与编码训练}
对影视作品进行微观表演分析,采用双人独立编码方式,编码一致性检验(Cohen's Kappa)需达到0.8以上。编码维度包括:情感表达强度、台词处理方式、肢体语言特征、角色类型(清纯/叛逆/成熟等)。对媒体报道和观众评论进行开放式编码和轴心编码,识别核心主题和话语框架。

\subsubsubsection{第四阶段:案例分析与理论验证}
基于统计数据和质性资料,对滨边美波的职业生涯进行阶段性分析,验证"童星转型期"理论模型的解释力。通过跨案例分析(与同时期其他童星对比),提炼青年演员职业发展的共性规律和个性特征。撰写研究报告初稿,并进行同行评议。

\subsubsubsection{第五阶段:比较研究与政策建议}
基于中日两国观众问卷数据,进行跨文化比较研究,分析两国在青年演员培养、观众期待、产业制度等方面的异同。组织专家研讨会,邀请中国演艺产业从业者、政策制定者、学者共同讨论日本经验的适用性。最终形成政策建议报告,为中国青年演员培养提供参考。

\subsubsection{关键技术}

\subsubsubsection{多模态数据整合技术}
\indent\setlength{\parindent}{2em}%首行缩进4字符
本研究涉及影视作品(视频)、媒体报道(文本)、观众评论(文本+社交媒体)、访谈录音(音频)等多模态数据。采用NVivo qualitative data analysis software进行数据管理和编码,通过交叉引用和矩阵查询功能,实现不同数据类型之间的关联分析。对于社交媒体大数据,采用Python爬虫技术和自然语言处理(NLP)方法进行情感分析和主题建模。图\ref{fig:data-pipeline}展示了多模态数据整合的技术路线。

\begin{figure}[htbp]
  \centering
  \includegraphics[width=0.9\textwidth]{figures/zzmx-115.jpg}
  \caption{\textbf{多模态数据整合技术路线}\\
  \raggedright \justifying \noindent
  该图展示了本研究采用的多模态数据整合流程,包括数据收集(影视作品、媒体报道、社交媒体、深度访谈)、数据预处理(编码、清洗、标注)、数据分析(情感分析、主题建模、网络分析)和结果可视化四个阶段。通过NVivo和Python工具实现多源数据的关联分析,为揭示演员职业发展规律提供数据支撑。}
  \label{fig:data-pipeline}
\end{figure}

\subsubsubsection{纵向案例分析技术}
采用时间序列分析和事件史分析(Event History Analysis)方法,识别滨边美波职业发展中的关键节点(如首次主演电影、首次转型成人角色、首次获奖等)。通过生存分析(Survival Analysis)方法,估计不同转型策略对职业 longevity 的影响。使用Stata或R软件进行统计分析。

\subsubsubsection{跨文化比较研究技术}
采用多组群结构方程模型(Multi-group SEM)方法,检验中日两国观众对青年演员评价的心理机制差异。通过测量等值性(Measurement Invariance)检验,确保跨文化比较的有效性。使用Mplus或AMOS软件进行分析。图\ref{fig:comparison-framework}展示了跨文化比较研究的分析框架。

\begin{figure}[htbp]
  \centering
  \includegraphics[width=0.85\textwidth]{figures/zzmx-mobile-105.jpg}
  \caption{\textbf{跨文化比较研究分析框架}\\
  \raggedright \justifying \noindent
  该框架展示了中日两国观众对青年演员评价的比较研究路径。左侧为自变量(文化背景、年龄、性别、观影经验),中间为中介变量(角色期待、审美标准、价值判断),右侧为因变量(演员评价、观看意愿、推荐行为)。通过多组群结构方程模型检验不同文化背景下各变量路径系数的差异,揭示文化因素对观众评价机制的调节作用。}
  \label{fig:comparison-framework}
\end{figure}

\clearpage

\subsubsubsection{质性-量化混合研究设计}
本研究采用解释性序列设计(Explanatory Sequential Design),先进行量化分析(如作品统计分析、问卷调查),再进行质性分析(如深度访谈、媒体话语分析),最后整合两种数据类型,形成三角验证(Triangulation)。该方法能够弥补单一研究方法的局限,提高研究结论的信度和效度。

\subsubsection{可行性分析}

\subsubsubsection{研究对象的可获得性}
\indent\setlength{\parindent}{2em}%首行缩进4字符
滨边美波作为公开的公众人物,其影视作品、媒体报道等资料均属于公开信息,可通过合法渠道获取。研究团队已与日本某大学建立合作关系,可通过该校图书馆访问《电影旬报》《日刊スポーツ》等数据库。同时,研究团队已学习日语,具备阅读日文资料和进行日语访谈的能力。

\subsubsubsection{研究团队的学术基础}
申请人长期从事明星研究、文化产业研究,已在CSSCI期刊发表相关论文5篇。研究团队成员包括:1名副教授(负责理论构建)、2名博士生(负责数据收集和分析)、3名硕士生(负责文献整理和编码)。团队成员具备影视分析、社会调查、统计分析等多学科背景,能够胜任跨学科研究任务。

\subsubsubsection{研究方法的成熟性}
本研究采用的纵向案例研究法、文本分析法、深度访谈法、问卷调查法均为社会科学研究的成熟方法,已有大量成功案例可供参考。研究团队在前期研究中已熟练掌握NVivo、Stata等分析工具,并完成了预调研(10个访谈、100份问卷),验证了研究设计的可行性。

\subsubsubsection{伦理审查与风险控制}
本研究涉及人类被试(访谈对象和问卷受访者),需严格遵守学术伦理。研究方案已通过所在机构伦理审查委员会审查(IRB编号:XXX-2024-XXX)。所有访谈均需获得知情同意,可采用匿名化处理保护受访者隐私。对于社交媒体数据,仅分析公开评论,不涉及个人隐私信息。

\clearpage
}

\subsection{\hspace{1.45em}  4.~{\bfseries 本项目的特色与创新之处;}}

{\justifying

\indent\setlength{\parindent}{2em}%首行缩进4字符

本项目的特色与创新点主要体现在:
(1)提出面向多尺度病灶的 CNN 结构组合策略:编码器残差化 + 解码器注意力门控 + 多尺度融合,使模型兼顾全局语义与边界细节;
(2)构建“扰动—一致性”一体化数据增强框架:将几何/强度扰动与 MixUp/CutMix 等组合增强统一到可控强度空间,显式提升跨中心泛化;
(3)引入不确定性驱动的失败样本挖掘与再训练机制,为临床可用性提供可解释的风险提示与闭环优化路径。
}


\subsection{\hspace{1.5em}  5.~{\bfseries 年度研究计划及预期研究结果}(包括拟组织的重要学术交流活动、国际合作与交流计划等)。}

{\justifying

\subsubsection{研究计划}

\indent\setlength{\parindent}{2em}%首行缩进4字符

第 1 年:完成数据规范化与标注流程;搭建基线 CNN(U-Net/ResNet)与训练管线,系统评估常规增强(翻转、旋转、裁剪、强度扰动)。
第 2 年:面向小病灶与多尺度结构,研发注意力与多尺度融合模块;引入组合增强(RandAugment、MixUp/CutMix)与一致性训练,开展消融实验。
第 3 年:开展跨中心泛化与迁移验证,形成可复现的模型/增强策略推荐方案;完善不确定性评估与失败案例分析,输出原型系统与论文/专利。

\subsubsection{预期研究结果}

\indent\setlength{\parindent}{2em}%首行缩进4字符

预期形成:
(1)一套面向医疗影像的 CNN 架构与训练策略(含模块化实现与参数配置建议);
(2)一套可复用的数据增强与鲁棒训练基准(含增强强度标定与迁移规则);
(3)公开可复现的实验报告与对比基线,并在典型任务(分割/分类)上取得稳定性能提升。

\clearpage
}


% \vskip -5mm %可以通过类似的命令微调行距以使得排版美观


\section{{\bfseries(二)研究基础与工作条件}} %2024


\subsection{\hspace{1.45em}  1.~{\bfseries 研究基础}(与本项目相关的研究工作积累和已取得的研究工作成绩);}


{
\justifying

\indent\setlength{\parindent}{2em}%首行缩进4字符

happy happy happy.

快乐 快乐 快乐。}

\subsection{\hspace{1.45em}  2.~{\bfseries 工作条件}(包括已具备的实验条件,尚缺少的实验条件和拟解决的途径,包括利用国家实验室、国家重点实验室和部门重点实验室等研究基地的计划与落实情况);}

{\justifying
\NSFCBodyText

\subsubsection{工作条件}

本项目已具备的研究条件包括数据资源、研究设备与软件条件以及国际合作基础,能够支撑资料采集、编码分析与跨媒介数据建模等工作。

\subsubsubsection{数据资源条件}
申请人已与日本某大学签署合作协议,可通过该校图书馆访问《电影旬报》《日刊スポーツ》《综艺》等数据库,这些数据库收录了日本影视产业的历史资料和最新动态,为本项目的数据收集提供了坚实保障。

申请人前期研究已整理形成“日本演艺产业资料库”(培训课程、课程设置、获奖名单、票房与收视率等),并具备合规采集公开社交媒体评论数据的技术与算力条件。

\subsubsubsection{研究设备与软件条件}
申请人所在单位拥有先进的人文社会科学研究实验室,配备以下设备和软件:

(1)\textbf{数据分析软件}:NVivo 12 Plus(质性数据分析软件)、Stata 16(统计分析软件)、SPSS 26(社会调查分析软件)、Mplus 8(结构方程模型分析软件),这些软件可满足本研究的数据分析需求。

(2)\textbf{高性能计算设备}:实验室拥有2台高性能工作站(配置:Intel Xeon CPU、64GB内存、2TB SSD),可用于大规模数据处理和自然语言处理任务。

(3)\textbf{影视分析设备}:实验室配备专业级视频编辑和分析软件(Adobe Premiere Pro、Final Cut Pro),可用于影视作品的微观表演分析。

(4)\textbf{数据处理脚本示例}:研究团队已开发标准化的数据处理脚本,用于大规模社交媒体数据的情感分析和主题建模。该脚本采用 Bash 和 Python 编写,基于自然语言处理库,可自动处理多语言文本数据。代码清单\ref{code:data-processing}展示了数据预处理与分析流程的关键片段。

\lstinputlisting[
  language=bash,
  firstline=1,
  lastline=10,
  caption={社交媒体数据处理脚本示例(code/test.sh)},
  label={code:data-processing}
]{code/test.sh}

\subsubsubsection{国际合作条件}
申请人与日本东京大学、早稻田大学等高校的相关研究团队保持长期学术联系,可为本项目提供以下支持:

(1)\textbf{数据访问支持}:通过日本合作大学的图书馆账户,可访问日本主流媒体的付费数据库,确保数据收集的完整性和准确性。

(2)\textbf{田野调查支持}:合作团队可协助联系日本演艺产业从业者(经纪人、制作人、表演教师等),为深度访谈提供便利。

(3)\textbf{学术交流支持}:项目执行期间,研究团队可赴日本合作大学交流访问1-2次,每次停留时间1-2个月,期间可利用合作单位的资源开展研究工作。

\subsubsubsection{尚缺少的条件与拟解决途径}
本项目尚缺少的条件主要是:大规模问卷调查的实施经验和专业级的影视作品分析设备。

\textbf{拟解决途径}:
(1)对于问卷调查,研究团队将在前期预调查(100份)的基础上,进一步完善问卷设计,并与专业调查公司合作,确保问卷的科学性和有效性。

(2)对于影视分析设备,研究团队将申请实验室专项经费,购买专业级的影视分析软件和设备,或利用学校影视学院的现有资源,通过合作共享的方式满足研究需求。
}

\subsection{\hspace{1.2em}  3.~{\bfseries 正在承担的与本项目相关的科研项目情况}(申请人正在承担的与本项目相关的科研项目情况,包括国家自然科学基金的项目和国家其他科技计划项目,要注明项目的资助机构、项目类别、批准号、项目名称、获资助金额、起止年月、与本项目的关系及负责的内容等);}

{\input{extraTex/2.3.承担项目.tex}}

\subsection{\hspace{1.45em}  4.~{\bfseries 完成国家自然科学基金项目情况}(对申请人负责的前一个已资助期满的科学基金项目(项目名称及批准号)完成情况、后续研究进展及与本申请项目的关系加以详细说明。另附该项目的研究工作总结摘要(限500字)和相关成果详细目录)。}

{\input{extraTex/2.4.项目完成情况.tex}}


\section{{\bfseries (三)其他需要说明的情况}} %2024

\subsection{\hspace{1.45em}  1.~申请人同年申请不同类型的国家自然科学基金项目情况(列明同年申请的其他项目的项目类型、项目名称信息,并说明与本项目之间的区别与联系)。}


{\input{extraTex/3.1.不同类型国基情况.tex}}

\subsection{\hspace{1.45em}  2.~具有高级专业技术职务(职称)的申请人是否存在同年申请或者参与申请国家自然科学基金项目的单位不一致的情况;如存在上述情况,列明所涉及人员的姓名,申请或参与申请的其他项目的项目类型、项目名称、单位名称、上述人员在该项目中是申请人还是参与者,并说明单位不一致原因。}

{\input{extraTex/3.2.同年单位不一致.tex}}

\subsection{\hspace{1.45em}  3.~具有高级专业技术职务(职称)的申请人是否存在与正在承担的国家自然科学基金项目的单位不一致的情况;如存在上述情况,列明所涉及人员的姓名,正在承担项目的批准号、项目类型、项目名称、单位名称、起止年月,并说明单位不一致原因。}

{\input{extraTex/3.3.承担中单位不一致.tex}}

\subsection{\hspace{1.45em}  4.~其他。}

{\input{extraTex/3.4.其它.tex}}

\clearpage
\end{document}