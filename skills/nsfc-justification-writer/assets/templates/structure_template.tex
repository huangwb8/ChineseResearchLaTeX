% 立项依据结构模板(与 projects/NSFC_Young/extraTex/1.1.立项依据.tex 对齐)

\subsubsection{研究背景}
\justifying\indent
% 建议写作要点:痛点/需求→影响范围或成本→为何现在必须做(1–2 段即可)

\subsubsection{国内外研究现状}
\indent
% 建议写作要点:主流路线与代表性工作→2–4 条明确不足(尽量可量化/可验证,避免口号)

\subsubsection{现有研究的局限性}
\indent
% 建议写作要点:2–4 条关键瓶颈→1–3 个科学问题(疑问句,非研究目标;约束与瓶颈一一映射)→一句科学假设(陈述句,预测性结果,不写验证方式)→验证维度(数据/指标/对照/消融)

\subsubsection{研究切入点}
\indent
% 建议写作要点:差异化切口→可交付成果与指标→最后 1 句承上启下引到 2.1 研究内容
