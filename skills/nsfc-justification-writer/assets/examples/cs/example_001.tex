% 示例(计算机):仅供结构与措辞参考

\subsubsection{研究背景}
\justifying\indent
(示例)面向真实业务系统的安全与隐私需求,数据共享受限导致模型难以获得足够多样的分布;同时在线推理与运维约束(时延、成本、可审计)使得“只追求离线精度”的方案难以落地。

\subsubsection{国内外研究现状}
\indent
(示例)现有工作主要集中在三类:其一,隐私保护学习(如联邦学习/差分隐私)缓解数据孤岛;其二,大模型与知识增强提升泛化;其三,系统侧的训练/推理优化提升效率与可用性。上述路线在跨域迁移、端侧约束与可审计性方面仍存在明显差距。

\subsubsection{现有研究的局限性}
\indent
(示例)现有方案往往在“隐私-性能-效率”三者间难以兼顾:隐私增强带来性能衰减,系统优化依赖特定硬件/框架,且缺乏面向真实故障与攻击面的稳健性评估。由此需要提出一个可证伪假说:在明确威胁模型与资源约束下,通过“算法-系统-评估”一体化设计,可在给定指标上实现可量化改进。

\subsubsection{研究切入点}
\indent
(示例)本项目拟从可审计的威胁模型出发,构建可复现基准与评测协议,提出面向资源约束的训练/推理协同方法,并以端到端指标(隐私预算、精度、时延/成本、稳健性)验证有效性,最后自然过渡到后续“研究内容与技术路线”。

