% 示例(数学/优化):仅供结构与措辞参考

\subsubsection{研究背景}
\justifying\indent
(示例)围绕某类优化/方程问题,先说明“应用或理论驱动→现有理论无法覆盖关键情形→需要新的可证明结果/可计算方法”的必要性。

\subsubsection{国内外研究现状}
\indent
(示例)按主流路线(理论收敛/复杂度分析/数值算法)分 2–3 类概括,强调核心假设与结论边界;确需引用时先核验再写 \cite{...}。

\subsubsection{现有研究的局限性}
\indent
(示例)用 2–4 条可验证不足收束:例如依赖过强假设、缺少统一框架、最坏情形界不紧、数值稳定性与误差控制不足等。

\subsubsection{研究切入点}
\indent
(示例)用“差异化切口(可证明性 + 可计算性)+ 可验收指标(定理/界/算法)+ 过渡到研究内容”结束。

