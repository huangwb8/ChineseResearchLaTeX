% 示例(生物):仅供结构与措辞参考

\subsubsection{研究背景}
\justifying\indent
(示例)围绕某疾病机制/关键通路,先用“临床或生物学重大问题→现有证据不足→迫切需要可验证的新机制/新靶点”说明必要性。

\subsubsection{国内外研究现状}
\indent
(示例)按主流技术路线(组学测量/模型系统/因果验证)分 2–3 类概括,强调代表性结论与证据类型;引用需先核验再写 \cite{...}。

\subsubsection{现有研究的局限性}
\indent
(示例)用 2–4 条可验证不足收束:例如样本异质性导致结论不稳、因果链条缺关键节点验证、跨尺度整合不足、可重复性与外部验证缺失等。

\subsubsection{研究切入点}
\indent
(示例)用“差异化切口(因果验证 + 外部验证 + 可复现实验方案)+ 量化指标 + 过渡到研究内容”结束。

