\usepackage[english]{babel} %支持混合语言
\usepackage{xcolor}
\usepackage{graphicx} 
\usepackage{amsmath} %更多数学符号
\usepackage{wasysym}
\usepackage{geometry} %改尺寸
\usepackage{fontspec} %字体包
\usepackage{setspace}
\usepackage{amsfonts} % 数学公式增强
\usepackage{xeCJK} % 调整字间距
\usepackage{titlesec} 
\usepackage{ragged2e} % 正文两端对齐所需的包。在需要两端对齐的文字前面添加\justifying即可。
% \usepackage{ctex}
\usepackage{pifont} % 带圈数字
\usepackage{ifplatform} % 检测操作系统

%%%————————————————————————————————边距
% \CJKsetecglue命令定义了CJK字符与其他字符之间的空白,不只限于数字字符,也适用于字母等其他非CJK字符。如果你想要更精细的控制,可能需要借助其它工具或者宏包。
% \hskip0.15em plus0.05em minus 0.05em表示基础间距为0.15em,可增加至0.2em,也可减少至0.1em以便满足排版需求。
% 页边距和字距的某个组合应该可以完美复制Word。
\geometry{left=3.20cm,right=3.14cm,top=2.54cm,bottom=2.54cm} % 页边距
\CJKsetecglue{\hskip 0.3em plus 0.05em minus 0.1em} % 数字与CJK字符的间距
% left=3.20cm,right=3.13-3.14cm

%%%————————————————————————————————颜色
\definecolor{MsBlue}{RGB}{0,112,192} %Ms Word 的蓝色和latex xcolor包预定义的蓝色不一样。通过屏幕取色得到。
\definecolor{headercolor}{RGB}{0,0,0} % 页眉颜色
\definecolor{footercolor}{RGB}{0,0,0} % 页脚颜色

%%%————————————————————————————————对象名
\addto\captionsenglish{
    \renewcommand{\contentsname}{目录}
    \renewcommand{\listfigurename}{插图目录}
    \renewcommand{\listtablename}{表格}
    %\renewcommand{\refname}{\sihao 参考文献}
    \renewcommand{\refname}{\sihao \kaishu \bfseries \leftline{参考文献} \vspace*{5mm}} %这几个字默认字号稍大,改成四号字,楷书,居左(默认居中) 根据喜好自行修改,官方模板未作要求
    \renewcommand{\abstractname}{摘要}
    \renewcommand{\indexname}{索引}
    \renewcommand{\tablename}{表}
    \renewcommand{\figurename}{图}
} %把Figure改成‘图’,reference改成‘参考文献’。如此处理是为了避免和babel包冲突。


%%%————————————————————————————————字号
\newcommand{\chuhao}{\fontsize{42pt}{\baselineskip}\selectfont}
\newcommand{\xiaochuhao}{\fontsize{36pt}{\baselineskip}\selectfont}
\newcommand{\yihao}{\fontsize{26pt}{\baselineskip}\selectfont}
\newcommand{\erhao}{\fontsize{22pt}{\baselineskip}\selectfont}
\newcommand{\xiaoerhao}{\fontsize{18pt}{\baselineskip}\selectfont}
\newcommand{\sanhao}{\fontsize{16pt}{\baselineskip}\selectfont}
\newcommand{\sihao}{\fontsize{14pt}{\baselineskip}\selectfont}
\newcommand{\xiaosihao}{\fontsize{12pt}{\baselineskip}\selectfont}
\newcommand{\yemeizihao}{\fontsize{11pt}{\baselineskip}\selectfont}
\newcommand{\wuhao}{\fontsize{10.5pt}{\baselineskip}\selectfont}
\newcommand{\xiaowuhao}{\fontsize{9pt}{\baselineskip}\selectfont}
\newcommand{\liuhao}{\fontsize{7.875pt}{\baselineskip}\selectfont}
\newcommand{\qihao}{\fontsize{5.25pt}{\baselineskip}\selectfont}
%字号对照表
%二号 21pt
%四号 14
%小四 12
%五号 10.5


%%%————————————————————————————————字体
\ifwindows
  % 可参考[LaTeX 中文字体配置基础指南 - 知乎](https://zhuanlan.zhihu.com/p/538459335) 选择自己喜欢的字体。 个人感觉中宋(SimSun)还可以 (~ ̄▽ ̄)~ 
  \setCJKfamilyfont{sectionzhfont}{KaiTi}[AutoFakeBold=3] % 定义模板中文字体为楷体
  \newcommand{\sectionzhfont}{\CJKfamily{sectionzhfont}} % 定义模板中文字体为新命令\sectionzhfont
  \newfontfamily\sectionenfont{KaiTi}[AutoFakeBold] % 定义模板中英文字体为楷体,并定义新命令\sectionzhfont
  \NewDocumentCommand \templatefont { } {\sectionzhfont \sectionenfont} % 模板中英文字体
  \setmainfont{Times New Roman} % 定义主体内容的英文字体为Times New Roman
  \setCJKmainfont{KaiTi}[AutoFakeBold=3]  % 定义主体内容的字体为Kaiti
  % \setCJKmainfont{SimSun}[AutoFakeBold=3]  % 定义主体内容的字体为SimSun
  \xeCJKsetup{PunctStyle=quanjiao} % 强制中文标点符号使用中文字体
\else
  \setCJKfamilyfont{sectionzhfont}[Path=./fonts/, Extension=.ttf, AutoFakeBold=3]{Kaiti} % 定义模板中文字体为楷体
  \newcommand{\sectionzhfont}{\CJKfamily{sectionzhfont}} % 定义模板中文字体为新命令\sectionzhfont
  \newfontfamily\sectionenfont[Path=./fonts/, Extension=.ttf, AutoFakeBold=3]{Kaiti} % 定义模板中英文字体为楷体,并定义新命令\sectionzhfont
  \NewDocumentCommand \templatefont {} {\sectionzhfont \sectionenfont} % 模板中英文字体
  \setmainfont[Path=./fonts/, Extension=.ttf]{TimesNewRoman} % 定义主体内容的英文字体为Times New Roman
  % \setCJKmainfont[Path=./fonts/, Extension=.ttf, AutoFakeBold=3]{SimSun} % 定义主体内容的字体为SimSun
  \setCJKmainfont[Path=./fonts/, Extension=.ttf, AutoFakeBold=3]{Kaiti} % 定义主体内容的字体为Kaiti
  \xeCJKsetup{PunctStyle=quanjiao} % 强制中文标点符号使用中文字体
\fi

%%%————————————————————————————————行距
%设置行距 1.5倍
\renewcommand{\baselinestretch}{1.5}
\XeTeXlinebreaklocale "zh" % 中文断行


%%%————————————————————————————————图片与表格

% 对于表格,推荐使用Excel的插件Excel2LaTeX自动生成LaTeX代码。详见: https://www.zhihu.com/question/307970489/answer/2305355098

%图片
\usepackage{graphicx} %插图宏包
\usepackage{float} %设置图片浮动位置的宏包
\usepackage{subfigure} %插入多图时用子图显示的宏包
\usepackage[section]{placeins} %避免浮动体跨过\section
\usepackage{enumerate} %设置列表环境的包
\usepackage{enumitem} % 增强自定义Bullet(即各种List前的那个标记)
\setlist[enumerate]{
  label={\templatefont \bfseries \hspace{1em} \color{MsBlue}(\arabic*)},
  leftmargin=0em, % 列表的左边界
  itemindent=4em, % item首行缩进
  itemsep=0em, % 列表项之间的垂直间距。
  labelsep=0.1pt,  % label与content的距离
  parsep=0em,  % 两个段落之间的垂直间距。效果上与itemsep类似。
  topsep=0em % 列表与上一段对象的垂直间距。
  % after=\vspace{5pt}
}
% \newcommand{\itemtitlefont}[1]{\textbf{\color{MsBlue} #1}} % 定义item小标题字体;
\newcommand{\itemtitlefont}[1]{{\bfseries \color{MsBlue} #1}} % 定义item小标题字体;
\usepackage{caption}
\captionsetup[figure]{name={图},font={footnotesize,stretch=1.25},labelsep=period,labelfont=bf, singlelinecheck=off, justification=centering} 

%表格
\usepackage{booktabs} %表格
\usepackage{tabularx} %表格宽度
\usepackage{multirow} %多行合并
\usepackage{longtable} % latex 多页显示同一表格
\usepackage{tabu} % 使用longtabu(tabularx + longtable)将长文本在单元格多行显示。
% \captionsetup[table]{name={表},font={footnotesize,stretch=1.25},labelsep=period,labelfont=bf, singlelinecheck=off, justification=centering} 


%%%————————————————————————————————超链接
\usepackage[
	colorlinks,
	urlcolor=MsBlue, % 超链接的颜色。默认是芭比粉
	linkcolor=black,
	anchorcolor=black,
	citecolor=black,
	CJKbookmarks=True
]{hyperref}


%%%————————————————————————————————标题计数

% [latex 标题、段落及行距 - 简书](https://www.jianshu.com/p/d7848f815e5f/)
% [LaTeX 中的行间距](https://yinguobing.com/linespace-in-latex/)

\usepackage{titlesec}

\setcounter{secnumdepth}{4} %显示3级目录,然后再具体定制不同的目录
% \setcounter{tocdepth}{4}

% Define section font
\newcommand{\sectionzihao}{\fontsize{14pt}{\baselineskip}\selectfont}
\newcommand{\subsectionzihao}{\fontsize{14pt}{14pt}\selectfont}
\newcommand{\subsubsectionzihao}{\fontsize{13.5pt}{20pt}\selectfont}


% Custom format for section
\titleformat{\section}
  {\color{MsBlue} \sectionzihao \templatefont} % format
  {\hspace{1.45em}  } % label
  {0pt} % separation
  {}   % before-code

% Custom format for subsection
\titleformat{\subsection}
  {\color{MsBlue} \subsectionzihao \templatefont \linespread{1}} % format
  {} % label
  {0pt} % separation
  {}   % before-code

% Custom format for subsubsection
\titleformat{\subsubsection}
  {\color{MsBlue} \subsubsectionzihao \templatefont \bfseries} % format
  {\hspace{1.1em}  \textnormal{\templatefont \arabic{subsection}.\arabic{subsubsection}}} % label
  {0.5em} % separation
  {}   % before-code

% Custom format for subsubsubsection
\titleclass{\subsubsubsection}{straight}[\subsubsection]
\newcounter{subsubsubsection}[subsubsection]
\renewcommand\thesubsubsubsection{(\arabic{subsubsection})}
\titleformat{\subsubsubsection}
  {\templatefont \bfseries} % format \color{MsBlue} 
  {\hspace{1em} (\arabic{subsubsubsection})} % label
  {0.5pt} % separation
  {}   % before-code
\makeatletter
\def\toclevel@subsubsubsection{4}
\def\l@subsubsubsection{\@dottedtocline{4}{7.0em}{4em}}
\makeatother

% Define line stretch for section
\titlespacing*{\section}
  {0pt} % Left indentation
  {0pt plus 0pt minus 0pt} % Space before the title
  {0pt plus 0pt minus 0pt} % Space after the title

% Define line stretch for subsection
\titlespacing*{\subsection}
  {0pt} % Left indentation
  {0pt plus 0pt minus 0pt} % Space before the title
  {0pt plus 0pt minus 0pt} % Space after the title

% Define line stretch for subsubsection
\titlespacing*{\subsubsection}
  {0pt} % Left indentation
  {0pt plus 0pt minus 0pt} % Space before the title
  {0pt plus 0pt minus 0pt} % Space after the title

% Define line stretch for subsubsection
\titlespacing*{\subsubsubsection}
  {0pt} % Left indentation
  {0pt plus 0pt minus 0pt} % Space before the title
  {0pt plus 0pt minus 0pt} % Space after the title


%%%————————————————————————————————代码
\usepackage{listings} %代码块
\usepackage{xcolor}%代码块
\renewcommand{\lstlistingname}{Code}

% 参考:
% https://www.youtube.com/watch?v=qxkQgG1Y0bY
% https://zhuanlan.zhihu.com/p/65441079

% \lstset{
% 	% language=bash,  %代码语言使用的是matlab
% 	frame=shadowbox, %把代码用带有阴影的框圈起来
% 	rulesepcolor=\color{red!20!green!20!blue!20},%代码块边框为淡青色
% 	keywordstyle=\color{blue!90}\bfseries, %代码关键字的颜色为蓝色,粗体
% 	commentstyle=\color{red!10!green!70}\textit,% 设置代码注释的颜色
% 	showstringspaces=true,%不显示代码字符串中间的空格标记
% 	numbers=left, % 显示行号
% 	numberstyle=\tiny,    % 行号字体
% 	stringstyle=\ttfamily, % 代码字符串的特殊格式
% 	breaklines=false, %对过长的代码自动换行
% 	extendedchars=false,  %解决代码跨页时,章节标题,页眉等汉字不显示的问题
% 	escapebegin=\begin{CJK*},escapeend=\end{CJK*},% 代码中出现中文必须加上,否则报错
% 	texcl=true
% }

\lstdefinestyle{codestyle01}{
	backgroundcolor=\color{gray!12},
	%basicstyle=\ttfamily\small,
	basicstyle=\zihao{-5}\ttfamily,
	commentstyle=\color{green!60!black},
	keywordstyle=\color{magenta},
	stringstyle=\color{blue!50!red},
	showstringspaces=false,
	numbers=none,
	numberstyle=\footnotesize\color{gray},
	numbersep=10pt,
	stepnumber=2, %左边数字的步幅
	tabsize=4,
	frame=TB, %顶部和底部有两条竖线。
  breaklines=true % 添加此行以启用自动换行
	% frame=tblr
	% frame=L,
	% framerule=1pt,
	% rulecolor=\color{red}
}


%%%————————————————————————————————其它设置


\makeatletter	
	\setlength{\@fptop}{0pt} % 图片置顶展示(而不是居中)
  % \setlength{\textfloatsep}{1cm plus 1.0pt minus 2.0pt} % 浮动体(如figure)之间的垂直间距
  \setlength{\intextsep}{0.5cm plus 1.0pt minus 2.0pt} % 浮动体(如figure)与文字的垂直间距
\makeatother

% 去除页码
\pagestyle{empty}

% 汉字与标点的间距;
% \punctstyle{banjiao} % 所有标点半角
% \punctstyle{kaiming} % 部分的标点半角
\punctstyle{hangmobanjiao} % 末半角式,仅行末挤压。

% 参考文献中的链接字体用的是texttt
\renewcommand{\ttdefault}{tnr} % 设置默认等宽字体为 Times New Roman

% 自定义第4级小标题
% \newcommand{\ssssubtitle}[1]{\ding{\numexpr171+#1\relax}} % 类似①
\newcommand{\ssssubtitle}[1]{\textcircled{\raisebox{-0.8pt}{\xiaosihao #1}}}
% \newcommand{\ssssubtitle}[1]{#1)} % 类似1)
% \newcommand{\blackding}[1]{\ding{\numexpr181+#1\relax}}
% \newcommand{\whitedingB}[1]{\ding{\numexpr191+#1\relax}}
% \newcommand{\blackdingB}[1]{\ding{\numexpr201+#1\relax}}